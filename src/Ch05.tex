\setcounter{chapter}{4}

\chapter{导数的应用}



% \begin{shaded}
% 	{\bf 多项式函数的拐点、极值点的判定:}设
% 	$$P(x)=(x-a_1)^{k_1}(x-a_2)^{k_2}\ldots(x-a_n)^{k_n},$$
% 	其中$a_i\in\mathbb{R}(i=1,2,\ldots,n)$(称为$P(x)$的{\it 零点},
% 	或$P(x)=0$的{\it 根}),$k_i\in\mathbb{Z}^+(i=1,2,\ldots,n)$
% 	(称为$a_i$对应的{\it 重数})。如下准则成立:
% 	\begin{itemize}
% %   	  \setlength{\itemindent}{1cm}
% 	  \item {\bf 所有二次以上的重根均为驻点,其中偶数次的为极值点,奇数次的为拐点}
% 	  \item {\bf 两个相邻零点之间有且仅有一个极值点}
% 	  \item {\bf 两个相邻驻点之间有且仅有一个拐点}
% 	\end{itemize}
% 
% 	{\bf 例:}$y=(x-1)(x-2)^2(x-3)^3$有几个极值点和拐点?
% 
% 	{\bf 解:}利用前述的判定准则:
% 	\begin{enumerate}[(1)]
%   	  \setlength{\itemindent}{1cm}
% 	  \item $x=2$为$2$重根,故为驻点兼极值点
% 	  \item $x=3$为$3$重根,故为驻点兼拐点
% 	  \item 在$(1,2)$和$(2,3)$中各有一个驻点兼极值点,记为$\xi_1,\xi_2$
% 	  \item $4$个驻点由小到大的排列为
% 	  $$\xi_1<2<\xi_2<3,$$
% 	  故在其间有且仅有$3$个拐点
% 	\end{enumerate}
% 
% 	综上,该函数共有$3$个极值点,$4$个拐点。
% 
% 	{\bf 例:}$y=2(x+1)^5(x-2)^8(x-4)^3+9$有几个极值点和拐点?
% 
% 	{\bf 解:}注意到该函数和
% 	$$y=2(x+1)^5(x-2)^8(x-4)^3,$$
% 	的极值点和拐点完全相同,故我们只需研究后者的极值点和拐点个数:
% 	\begin{enumerate}[(1)]
%   	  \setlength{\itemindent}{1cm}
% 	  \item $x=-1$为$5$重根,故为驻点兼拐点
% 	  \item $x=2$为$8$重根,故为驻点兼极值点
% 	  \item $x=4$为$3$重根,故为驻点兼拐点
% 	  \item 在$(-1,2)$和$(2,4)$中各有一个驻点兼极值点,记为$\xi_1,\xi_2$
% 	  \item $5$个驻点由小到大的排列为
% 	  $$-1<\xi_1<2<\xi_2<4,$$
% 	  故在其间有且仅有$4$个拐点
% 	\end{enumerate}
% 
% 	综上,该函数共有$3$个极值点,$6$个拐点。
% 
% \end{shaded}

\section{函数的极值与最值}





\subsection{Taylor多项式}





\section{函数的单调性、凹凸性}

\subsection{可导函数的单调性与极值}

\subsubsection{【单调性】}

{\bf 定理5.4.1:}设$f(x)$在$[a,b]$上连续,$(a,b)$内可导,且$f\,'(x)$恒大(小)于零,
则$f(x)$在$[a,b]$上严格单调递增(减)。



\subsubsection{【可导函数的极值】}

{\bf 定理5.4.2}(极值第一充分条件)
设$f(x)$在$x_0$连续,在$x_0$的去心领域内可导,且$f\,'(x)$在$x_0$两侧导数值异号
({\it $f(x)$单调性不同}),则$f(x)$在$x_0$处取极值。\ps{\b 此处不关心$f(x)$在
$x_0$是否可导!}



{\bf 定理5.4.3}(极值第二充分条件)\ps{最常用的极值判定方法}
设$f(x)$在$x_0$二阶可导,且$f\,'(x_0)=0$,则
\begin{enumerate}[(1)]
  \setlength{\itemindent}{1cm}
  \item 若$f\,''(x_0)<0$,$f(x)$在$x_0$处取极大值 
  \item 若$f\,''(x_0)>0$,$f(x)$在$x_0$处取极小值
\end{enumerate}



\subsubsection{【综合应用】}





% {\bf 例:}$f(x)\in C[a,b]$,且在$(a,b)$内$f''(x)>0$,试讨论
% $$F(x)=\df{f(x)-f(a)}{x-a}$$
% 在$(a,b)$内的单调性。
% 
% 解:对任意$x\in(a,b)$,由Lagrange中值定理,存在$\xi\in(a,x)$,
% 使得$f(x)-f(a)=f'(\xi)(x-a)$,又由已知,$f''(x)>0$,
% 故$f'(x)$在$(a,b)$上严格单调递增,从而$f'(x)>f'(\xi)$,于是
% $$F'(x)=\df{f'(x)(x-a)+f(x)-f(a)}{(x-a)^2}=\df{f'(x)+f'(\xi)}{x-a},$$



\subsection{曲线的凹凸性}







{\bf 推论}(教材-例11)严格凸(凹)函数的驻点为极值点。

\subsection{分析作图}



\section{曲率}

% {\bf 问题:}如何刻画一条平面曲线的几何特征?
% 
% \begin{itemize}
%   \setlength{\itemindent}{1cm}
%   \item {\bf 切线斜率:}一阶导数
%   \item {\bf 凹凸性:}二阶导数
%   \item {\bf 长度:}弧微分
%   \item {\bf 弯曲程度:}曲率
% \end{itemize}



\subsection{弧微分}


% {\bf 例:}求$y=x^2$对应于$x\in[0,1]$的曲线长度。

\subsection{曲率}





\section*{课后作业}
\addcontentsline{toc}{section}{课后作业}

{\bf 【必作题】}

\begin{itemize}
  \item 习题5.1:6(2,4),9,11,12,14
  \item 习题5.2:3,6,9,10,12,14,16
  \item 习题5.3:1,2,4,5,6,10,12,13,14
  \item 习题5.4:6(2,4),7,9,11,13,14(3)
  \item 习题5.5:3,5,6,7,10
\end{itemize}

\bigskip

\hrule

\bigskip
\bigskip

{\bf 【思考题】}

\begin{itemize}
  \item 习题5.1:7,13
  \item 习题5.2:4,5,7,8,11,13,15,17,18,19
  \item 习题5.3:3,8,9,16,17
  \item 计算如下极限
  	\begin{enumerate}
	  \item $\limx{0}\df{e^x-e^{-x}-2x}{x-\sin x}$ 
	  \item $\limx{0}\df{e^{-1/x^2}}{x^{100}}$ 
	  \item $\limx{a}\df{x^a-a^x}{x^x-a^a}\;(a>0)$ 
	  \item $\limx{0}\left(\df{a^x+b^x+c^x}{3}\right)^{1/x}$ 
	  \item $\limn\sqrt{n}(\sqrt[n]{n}-1)$ 
	  \item $\limn\left(n\sin\df{1}{n}\right)^{n^2}$
	\end{enumerate}	
  \item 习题5.4:4,5,8,12,15-21
  \item 习题5.5:15
\end{itemize}

\newpage



\newpage

\section*{习题5.3参考解答}
\addcontentsline{toc}{section}{习题5.3参考解答}

1. 将$f(x)=x^3+3x^2-2x+4$在$x_0=-1$处展开成Taylor公式。

解:因为
$$f(x)=[(x+1)-1]^3+3[(x+1)-2]^2-2[(x+1)-1]+4
=8-5(x+1)+(x+1)^3,$$
故$f(x)$在$x_0=-1$处的各阶带Peano余项的Taylor公式依次为
\begin{align}
	P_1(x)&=8-5(x+1)+\circ(x+1)\notag\\
	P_2(x)&=8-5(x+1)+\circ((x+1)^2)\notag\\
	P_3(x)&=8-5(x+1)+(x+1)^3+\circ((x+1)^3)\notag\\
	P_k(x)&=8-5(x+1)+(x+1)^3+\circ((x+1)^k),\;k>3.\notag
\end{align}

\bigskip

2. 已知$f(x)$是四次多项式,且$f(2)=-1,f'(2)=0,f''(2)=-2,f'''(2)=-12,
f^{(4)}(x)=24$,求$f(-1)$.

解:由已知,$f(x)$在$x=2$处的Taylor多项式的各阶系数依次为
\begin{align}
	a_0&=f(2)=-1\notag\\
	a_1&={f'(2)}=0\notag\\
	a_2&=\df{f''(2)}2=-1\notag\\
	a_3&=\df{f'''(2)}{3!}=-2\notag\\
	a_4&=\df{f^{(4)}(2)}{4!}=1\notag
\end{align}
因为$f(x)$为四阶多项式,故其四阶Taylor多项式即为其自身,从而
$$f(x)=-1-(x-2)^2-2(x-2)^3+(x-2)^4,$$
由此易得$f(-1)=125$.

\bigskip

4. 写出下列函数带Peano余项的Maclaurin公式

(1) $f(x)=\ln(2+x)$
\begin{align}
	\ln(2+x)&=\ln2+\ln\left(1+\df x2\right)\notag\\
	&=\ln2+\sum\limits_{k=1}^n\df{(-1)^{k-1}}{k}\left(
	\df{x}2\right)^k+\circ\left(\left(\df x2\right)^n\right)\notag\\
	&=\ln2+\sum\limits_{k=1}^n\df{(-1)^{k-1}}{2^kk}x^k+\circ(x^n)\notag
\end{align}

(2) $f(x)=e^{-x^2}$
\begin{align}
	e^{-x^2}&=\sum\limits_{k=0}^n\df{(-x^2)^k}{k!}+\circ((-x^2)^n)\notag\\
	&=\sum\limits_{k=0}^n\df{(-1)^k}{k!}x^{2k}+\circ(x^{2n})\notag
\end{align}

(3) $f(x)=x\sin x$
\begin{align}
	x\sin x&=x\left[\sum\limits_{k=0}^n\df{(-1)^k}{(2k+1)!}x^{2k+1}
	+\circ(x^{2n+1})\right]\notag\\
	&=\sum\limits_{k=0}^n\df{(-1)^k}{(2k+1)!}x^{2k+2}
	+\circ(x^{2n+2})\notag
\end{align}

(4) $f(x)=\df{x^2}{1+x}$
\begin{align}
	\df{x^2}{1+x}&=x^2\left[\sum\limits_{k=0}^n(-x)^k
	+\circ((-x)^n)\right]\notag\\
	&=\sum\limits_{k=0}^n(-1)^kx^{k+2}+\circ(x^{n+2})\notag
\end{align}

(5) $f(x)=\df1{\sqrt{1-x^2}}$
\begin{align*}
	\df1{\sqrt{1-x^2}}&=[1+(-x^2)]^{-\frac12}
	=\sum\limits_{k=0}^n\left(\begin{array}{c}
	-\frac12 \\ k\end{array}\right)
	(-x^2)^k+\circ((-x^2)^n) \\
	&=\sum\limits_{k=0}^n\left(\begin{array}{c}
	-\frac12 \\ k\end{array}\right)
	(-1)^kx^{2k}+\circ(x^{2n})
	=\sum\limits_{k=0}^n\frac{(2k-1)!!}{2^kk!}x^{2k}
	+\circ(x^{2n})
\end{align*}

(6) $f(x)=\cos^2x$
\begin{align}
	\cos^2x&=\df{1+\cos2x}2\notag\\
	&=\df12+\df12\left[\sum\limits_{k=0}^n\df{(-1)^k}{(2k)!}(2x)^{2k}
	+\circ(x^{2n})\right]\notag\\
	&=\df12+\sum\limits_{k=0}^n\df{(-1)^k2^{2k-1}}{(2k)!}(x)^{2k}
	+\circ(x^{2n})\notag
\end{align}

\bigskip

5. 求函数$f(x)=\df1x$在$x=-1$处带Lagrange余项的$n$阶Taylor公式

解:$f^{(n)}(x)=\df{(-1)^nn!}{x^{n+1}}$,故存在$\xi$介于$x$和$-1$之间,使得
\begin{align}
	f(x)&=\df1{-1+(x+1)}=-\df1{1-(x+1)}\notag\\
	&=-\sum\limits_{k=0}^n(x+1)^k
	+\df{(-1)^{n+1}(n+1)!}{\xi^{n+2}}(x+1)^{n+1},\notag
\end{align}
即为所求。

\bigskip

6. 将函数$f(x)=\ln x$按$x-2$的幂展开成带Peano余项的$n$阶Taylor公式。

解:
\begin{align}
	f(x)&=\ln(2+(x-2))=\ln2+\ln\left(1+\df{x-2}2\right)\notag\\
	&=\ln2+\sum\limits_{k=1}^n\df{(-1)^{k-1}}{k}
	\left(\df{x-2}2\right)^k+\circ\left(\left(\df{x-2}2\right)^n\right)\notag\\
	&=\ln2+\sum\limits_{k=1}^n\df{(-1)^{k-1}}{2^kk}
	(x-2)^k+\circ((x-2)^n)\notag
\end{align}

\bigskip

10. 利用Taylor公式求下列极限

(1)
\begin{align}
	\limx0\df{xe^x-\ln(1+x)}{x^2}
	&=\limx0\df{x+x^2+\circ(x^2)-[x-\frac{x^2}2+\circ(x^2)]}{x^2}\notag\\
	&=\df32+\limx0\circ(1)=\df32\notag
\end{align}

(2)
\begin{align}
	\limx{+\infty}&\left[\left(x^3-x^2+\df x2\right)e^{\frac1x}
	-\sqrt{x^6+1}\right]
	\xlongequal{u=1/x}\lim\limits_{u\to0^+}
	\df{(1-u+2u^2)e^u-\sqrt{1+u^6}}{u^3}\notag\\
	&=\lim\limits_{u\to0^+}\df{\left(1-u+\frac{u^2}2\right)
	\left(1+u+\frac{u^2}2+\frac{u^3}6+\circ(x^3)\right)
	-[1+\frac12u^6+\circ(u^6)]}{u^3}\notag\\
	&=\lim\limits_{u\to0^+}\df{\frac{u^3}6+\circ(u^3)}{u^3}
	=\df16+\lim\limits_{u\to0^+}\circ(1)=\df16\notag
\end{align}

(3)
\begin{align}
	\limx{+\infty}&\left[x-x^2\ln\left(1+\df1x\right)\right]
	\xlongequal{u=1/x}\lim\limits_{u\to0^+}
	\df{u-\ln(1+u)}{u^2}\notag\\
	&=\lim\limits_{u\to0^+}\df{u-[u-\frac{u^2}2+\circ(u^2)]}{u^2}
	=\df12+\lim\limits_{u\to0^+}\circ(1)=\df12\notag
\end{align}

(4)
\begin{align}
	\limx0&\df{\frac{x^2}2+1-\sqrt{1+x^2}}{x^2\sin x^2}
	=\limx0\df{\frac{x^2}2+1-[1+\frac12x^2-\frac18x^4+\circ(x^4)]}{x^4}\notag\\
	&=\df18+\limx0\circ(1)=\df18\notag
\end{align}

\bigskip

12. 常数$a,b$为何值时,成立$\limx0\left[\df{\ln(1+2x)}{x}+a
+\df bx\right]=0$?

解:
$$\df{\ln(1+2x)+ax+b}{x}=\df{2x+\circ(x)+ax+b}x=(a+2)+\circ(1)+\df bx,$$
显然以上函数若当$x\to0$时极限存在,则必有$b=0$;又由极限值为$0$,可知$a+2=0$,
故$a=-2$。

\bigskip

13. 设$f(x)$在$x=0$附近二次可导,且
$$\limx0\left(\df{\sin x}{x^3}+\df{f(x)}{x^2}\right)=0,$$
(1) 求$f(0).f'(0),f''(0)$;(2)计算$\limx0\df{1+f(x)}{x^2}$.

解:由已知
$$0=\limx0\left(\df{\sin x}{x^3}+\df{f(x)}{x^2}\right)
=\limx0\df{x-\frac{x^3}6+\circ(x^3)+xf(x)}{x^3}
=\limx0\df{1+f(x)}{x^2}-\df16,$$
故$\limx0\df{1+f(x)}{x^2}=\df16$。
$$\limx0[1+f(x)]=\limx0\df{1+f(x)}{x^2}\limx0x^2=0,$$
从而$f(0)=\limx0f(x)=-1$。又
$$f'(0)=\limx0\df{f(x)-f(0)}{x-0}=\limx0\df{f(x)+1}{x}
=\limx0\df{f(x)+1}{x^2}\limx0x=0,$$
于是由Taylor公式,可设$f(x)=-1+\df{f''(0)}2x^2+\circ(x^2)$,进而
$$\df16=\limx0\df{1+f(x)}{x^2}=\df{f''(0)}2+\limx0\circ(1)
=\df{f''(0)}2\quad\Rightarrow\quad f''(0)=\df13.$$
以上即为所求。

\bigskip

14. 设函数$f(x)$在$[0,1]$上二次可导,且$f(0)=f(1)$,$|f''(x)|\leq2$,
证明:$x\in[0,1]$时,有$|f'(x)|\leq 1$.

证:对任意$x\in[0,1]$,由Taylor公式,存在$\xi_1\in(0,x),\xi_2\in(x,1)$,
使得
$$f(0)=f(x)+f'(x)(-x)+\df{f''(\xi_1)}2x^2,$$
$$f(2)=f(x)+f'(x)(1-x)+\df{f''(\xi_2)}2(1-x)^2,$$
以上两式相减,整理后可得
$$f'(x)=-\df12\left[f''(\xi_2)(1-x)^2-f''(\xi_1)x^2\right],$$
从而
\begin{align}
	|f'(x)|&=\df12\left|f''(\xi_2)(1-x)^2-f''(\xi_1)x^2\right|\notag\\
	&\leq\df12\left[|f''(\xi_2)|(1-x)^2+|f''(\xi_1)|x^2\right]\notag\\
	&\leq(1-x)^2+x^2\notag
\end{align}
注意到,当$x\in[0,1]$时,$(1-x)^2+x^2$最大值为$1$,故由上式可得$|f'(x)|\leq1$,
即证。

\section*{补充例题}
\addcontentsline{toc}{section}{补充例题}

\section*{补充例题}
% \addcontentsline{toc}{section}{补充例题}



























% \begin{tabbing}
% 	\hspace{3cm}\=\hspace{3cm}\=\hspace{3cm}\=\kill
% 	\quad\quad\quad
% 	(A)\;$\lambda<0,k<0$\>  
% 	\quad\quad\quad
% 	(B)\;$\lambda<0,k>0$\>
% 	\quad\quad\quad  
% 	(C)\;$\lambda\geq0,k<0$\>
% 	\quad\quad\quad 
% 	(D)\;$\lambda\leq0,k>0$
% \end{tabbing} 







