\setcounter{chapter}{9}



\newpage

\section*{课后作业}
\addcontentsline{toc}{section}{课后作业}

{\bf 【必作题】}

\begin{itemize}
  \setlength{\itemindent}{1cm}
  \item 习题10.1:9(3,4),12,13
  \item 习题10.2:5(3,4),8,9,13,14,20
  \item 习题10.3:8,16,18,23,24,26,27
  \item 习题10.4:6,7,8,9,13
  \item 习题10.5:3,4,5,7,10,14,15
\end{itemize}

\bigskip

\hrule

\bigskip

{\bf 【思考题】}

\begin{itemize}
  \setlength{\itemindent}{1cm}
  \item 习题10.1:15
  \item 习题10.2:11,16,17,18
  \item 习题10.3:所有未布置的题目
  \item 习题10.4:12,14,15,16
  \item 习题10.5:16,17,18,19
\end{itemize}

{\it\b 如果能够在本章开课一周内搞定以下的所有题,可以免交
本章必作题!欢迎选作!}

\begin{enumerate}
  \setlength{\itemindent}{1cm}
  \item 证明以下函数可微,但偏导数不连续\ps{本章必须掌握的一类题}
  	$$
	f(x,y)=\left\{\begin{array}{ll}
	0 & ,(x,y)=(0,0)\\
	(x^2+y^2)\sin\df1{x^2+y^2}& ,else
	\end{array}\right.
	$$
  \item 设$f(x,y)$一阶偏导连续,$f(1,1)=1,f\,'_x(1,1)=2$,
	$f\,'_y(1,1)=3$,又$\varphi(x)=f(x,f(x,x))$,求
	\ps{考试中经常出但又容易作错的小题}
	$$\left.\df{\d\varphi^3(x)}{\d x}\right|_{x=1}$$
  \item 设$u=u(x,y,z)$具有连续偏导数,且\ps{基本的偏导数计算问题}
	$$x=r\sin\theta\cos\varphi,y=r\sin\theta\sin\varphi,z=r\cos\theta$$
	证明:若$x\df{\p u}{\p x}+y\df{\p u}{\p y}+z\df{\p u}{\p z}=0$,
	则$u$与$r$无关。
  \item 设$u=u(x,y,z)$,证明:$u$为$x,y,z$的线性函数,当且仅当
	$\bigtriangledown u$为常矢量。	\ps{正确叙述解题过程是关键!}
  \item 证明:当$x,y$很小时,$\arctan\df{x+y}{1+xy}\approx x+y$
  \item 求以$(0,0),(0,1),(1,0)$为顶点的三角形区域内
	到该三点距离平方和最大的点。\ps{基础题,必考!}
  \item 求函数$z=x^2-xy+y^2$在区域$D:|x|+|y|\leq 1$
	上的最大和最小值。\ps{基础题,必考!}
  \item 在半径为$R$的圆的内接三角形中,求面积最大者。\ps{基础题,必考!}
  \item 设$F(x,y,z)$在条件$\varphi(x,y,z)=0$和$\psi(x,y,z)=0$
	之下在点$P_0(x_0,y_0,z_0)$处取得极值$m$,证明:曲面
	$F(x,y,z)=m,\varphi(x,y,z)=0$和$\psi(x,y,z)=0$
	在$P_0$的法线共面,其中函数$F,\varphi,\psi$均有连续且不同时
	为零的一阶偏导数。
\end{enumerate}

\newpage

\section*{习题参考解答}
\addcontentsline{toc}{section}{习题参考解答}

{\bf 习题10.1}

9.(3)\;解:
$$
	\lim\limits_{(x,y)\to(0,0)}\df{2-\sqrt{xy+4}}{xy}
	=\lim\limits_{(x,y)\to(0,0)}\df{-xy}{xy\left(2+\sqrt{xy+4}\right)}
	=-\df14
$$

(4)\;解:
$$
	\lim\limits_{(x,y)\to(2,0)}\df{\sin(xy)}{y}
	=\lim\limits_{(x,y)\to(2,0)}\df{xy}{y}
	=\lim\limits_{(x,y)\to(2,0)}x=2
$$

12.(1)\;证:因为
$$
	\lim\limits_{x\to 0\atop{y=kx}}\df{x+y}{x-y}
	=\lim\limits_{x\to 0\atop{y=kx}}\df{x+kx}{x-kx}=\df{1+k}{1-k},
$$
结果与$k$有关,这说明当自变量沿不同方向趋于$(0,0)$时,函数有不同极限,所以
原二重极限不存在。

(2)\;证:因为
$$
	\lim\limits_{x\to 0\atop{y=kx}}\df{\ln(1+xy)}{x(x+y)}
	=\lim\limits_{x\to 0\atop{y=kx}}\df{xy}{x(x+y)}
	=\lim\limits_{x\to 0\atop{y=kx}}\df{kx^2}{(1+k)x}=\df{k}{1+k},
$$
结果与$k$有关,这说明当自变量沿不同方向趋于$(0,0)$时,函数有不同极限,所以
原二重极限不存在。

13.(1)\;解:设$x=\rho\cos\theta,y=\rho\sin\theta$,则
$$
	\lim\limits_{(x,y)\to(0,0)}\df{3x^2-x^2y^2+y^2}{x^2+y^2}
	=3\cos^2\theta+\sin^2\theta,
$$
结果与$\theta$有关,故$f(x,y)$当$(x,y)\to(0,0)$时极限不存在,从而无论
如何定义$f(0,0)$,函数$f(x,y)$都不可能在$(0,0)$处连续。

(2)\;解:设$x=\rho\cos\theta,y=\rho\sin\theta$,则
$$
	\lim\limits_{(x,y)\to(0,0)}=\lim\limits_{\rho\to0}
	2\rho\cos\theta\sin^2\theta=0,
$$
由此可知,令$f(0,0)$即可。

{\bf 习题10.2}

5.(3)\;解:
$$z''_{xx}=\df{-2x}{(1+x^2)^2},\quad
z''_{yy}=\df{-2y}{(1+y^2)^2}\quad
z''_{xy}=0.$$

(4)\;解:
\begin{align*}
	z''_{xx}&=y^x\ln^2y\ln(xy)+\df{2y^x\ln y}x-\df{y^x}{x^2}\\
	z''_{yy}&=x(x-1)y^{x-2}\ln(xy)+2xy^{x-2}-y^{x-2}\\
	z''_{xy}&=xy^{x-1}\ln y\ln(xy)+y^{x-1}\ln(xy^2)+y^{x-1}
\end{align*}

8.\;(1)$\d f=e^{\frac{\sin y}x}\df1{x^2}\cos\df yx(x\d y-y\d x)$

(2)$\d f=e^x(\sin y+\cos y+x\sin y)\d x+e^x(-\sin y+x\cos y)\d y$

(3)$\d f=x^{y^2z}\left(\df{y^2z}x\d x+2yz\ln x\d y+y^2\ln x\d z\right)$

(4)$\d f=\df{xyz}{1+(xyz)^2}\left(\df{\d x}x+\df{\d y}y+\df{\d z}z\right)$

9.\;$\df13\d x+\df23\d y$

13.\;$\df12x^2+y^2\ln x+\sin y-\df12$

14.\;$z'_x=-e^{-x^2},\;z'_y=2e^{-4y^2}$

15.\;解:设圆锥的地面半径和高分别为$r,h$,则其体积$V=\df13\pi r^2h$,进而
$$\Delta V\approx\d V=\df23\pi rh\d r+\df13\pi r^2\d h\leq 20\pi,$$
也即计算圆锥体积的最大误差为$20\pi(\mathrm{cm}^3)$

{\bf 习题10.3}

8.$-1$

16.\;切线:$\df{x+2}{25}=\df{y-1}{28}=\df{z-6}{12}$,
法平面:$25x+28y+12z-50=0$.

18.$f'_y(x,2x)=x-\df12x^3$.

23.\;(1)$-2$;(2)$-1$.

24.\;$\varphi(1)=1,\;\varphi'(1)=a+ab+ab^2+b^3$,提示:
$$
	\varphi'(x)=f'_1(x,f(x,f(x,x)))+f'_2(x,f(x,f(x,x)))\left\{
	f'_1(x,f(x,x))+f'_2(x,f(x,x))[f'_1(x,x)+f'_2(x,x)]\right\}
$$

26.\;(2)[提示]:$xz'_x-yz'_y=xf(u)+\df{x^2}yf(u)$,代入方程可得
$$f(u)+2uf'(u)=0\quad\Rightarrow\quad f(u)=-Cu^{-1/2}\;(C\in\mathbb{R})$$

27.\;$f(u)=C_1e^u+C_2e^{-u},\;(C_1,C_2\in\mathbb{R})$

{\bf 习题10.4}

6.\;$5+2(x-1)^2-(x-1)(y+2)-(y+2)^2$

7.\;$y+xy-\df12y^2+\circ(x^2+y^2)$

8.\;$\df{\sqrt2}3$

9.\;$\sqrt2\df{\sqrt{a^2+b^2}}{ab}$

{\bf 习题10.5}

3.\;在$\left(\df{18}{13},\df{12}{13}\right)$处取极小值$\df{36}{13}$

4.\;$\df{100}{3}\left(1,1,1\right)$

5.\;最大值$z(-3,4)=125$,最小值$z(3,-4)=75$

7.\;最小值点$\left(\df12,\df14\right)$,最小值$\df{7\sqrt2}8$

10.\;最大值$z(-1,1)=5$,最小值$z(1,-1)=-5$

15.\;从距离两边各$8\mathrm{cm}$处折起,腰与底边夹角为$\df{\pi}3$时断面面积最大

\ifvisible

\section*{关于评教}

{\it (以下都是我特别想了解的问题,欢迎大家结合自己的感受发表宝贵的意见,提出改进的建议)}

\begin{enumerate}
  \item 关于课堂
  \begin{itemize}
    \item 我们目前授课的节奏,你觉得合适吗?进度是否太快,或者,太慢?
    \item 关于我上课的方式和课堂讲解,你觉得自己的接受程度如何?我哪方面特别需要改进?
    \item 多媒体演示还需要多一些吗?你觉得怎样能使演示的效率更高、效果更好? 
    \item 有必要继续安排课前写题吗?每次讲评总是会占用一部分正课时间,有没有其他好的解决办法?
  \end{itemize}
  \item 关于作业与练习
  \begin{itemize}
    \item 作业量是否过多?难度合适吗?
    \item 对目前采取的作业批改方式,你觉得效果如何,有没有什么改进的方法?
    \item 对于我批改的作业以及作业评分,你感觉满意吗?有什么具体的意见和建议?
    \item 你个人课后看书、作作业以及看辅导资料的时间充足吗?
    \item 能不能估算出一周除了上课,你自己大概还有多少时间花在了和高数有关的事情上?
    能否把估算的结果告诉我?
  \end{itemize}
  \item 关于交流与讨论
  \begin{itemize}
    \item 微信群和云盘这样的交流平台,你觉得真的有效果吗?你有没有下载过云盘上的资料?
    \item 你觉得老师和学生之间应该用什么样的方式来讨论问题?
    \item 我们有必要安排定期的答疑吗?最长多少时间合适呢?
  \end{itemize}
  \item 其他任何与我们这门课和我个人教学有关的意见和建议
\end{enumerate}

\fi

% \newpage
% 
% {\bf 第10章习题课作业}
% 
% {\it 本次作业请抄题!}



