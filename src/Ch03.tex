\setcounter{chapter}{2}

\chapter{函数极限与连续}

极限是微积分理论体系中最基本也最重要的概念之一,后面我们将要讨论的导数(微分)和
积分的概念都是建立在极限概念之上的,更具体地说,导数和定积分都是特定类型的极限。

在上一章学习了数列极限的基础上,本章我们将进一步讨论更具一般性的函数的极限,二者
的定义和性质,从很多方面来看都是相似的,只是由于函数极限的类型更为丰富,相对而言
有关结论也更为多样,有些性质是数列极限所不具备的。

\section{函数极限的概念与性质}

函数极限是我们后续定义导数、积分的概念的基础,因此可以说函数极限的理论也是整个
微积分学的基础。

\subsection{函数极限的概念}

不同于数列极限只有单一的$n\to\infty$一种(自变量的变化)趋势,
{\bf 函数极限共有六种可能的(自变量变化)趋势},分为两大类:

\begin{itemize}
  \setlength{\itemindent}{1cm}
  \item 趋于无穷
  $$x\to\infty,\quad x\to+\infty,\quad x\to-\infty$$
  \item 趋于有限值
  $$x\to x_0,\quad x\to x_0^+,\quad x\to x_0^-$$
\end{itemize}

{\bf 思考:} $\limn f(n)=A\Leftrightarrow\limx{+\infty}f(x)=A$?

{\bf 答}:显然,{\b$\limx{+\infty}f(x)=A$显然可以推出$\limn f(n)=A$,但反之不然},
\ps{$f(n)$的值域是$f(x)$值域的子集}

例如:$y=\sin\pi x$当$x\to+\infty$时不收敛,但$\limn\sin n\pi=0$。

\subsubsection{【趋于无穷时的函数极限】}

{\bf 定义3.1.1}
\ps{\b 和$\limn a_n=a$的定义做类比,理解(1),然后推广到(2),(3)}
\begin{enumerate}[(1)]
  \setlength{\itemindent}{1cm}
  \item $\limx{+\infty}f(x)=A\Leftrightarrow\forall \e>0,\exists X,\forall
  x>X,|f(x)-A|<\e$
  \item $\limx{-\infty}f(x)=A\Leftrightarrow\forall \e>0,\exists X,\forall
  x<X,|f(x)-A|<\e$
  \item $\limx{\infty}f(x)=A\Leftrightarrow\forall \e>0,\exists X,\forall
  |x|>X,|f(x)-A|<\e$
\end{enumerate}

{\bf 例:}证明$\limx{+\infty}\df1{\sqrt x}=0$.

{\bf 注:}不等式右端的$\e$可以改写为$C\e$,其中$C>0$为常数

{\bf 注:}{\b 趋于无穷的函数极限存在,意味着函数存在{\it 水平渐近线}},
且水平渐近线最多可以有两条,例如:$y=\arctan x$

% \begin{center}
% 	\resizebox{!}{4cm}{\includegraphics{./images/ch3/arctan.pdf}}
% \end{center}

\begin{center}
	\begin{overpic}[scale=0.3]{./images/ch3/arctan.pdf}
		\put(0,10){\b $\limx{-\infty}\arctan x=-\df{\pi}2$}
		\put(65,53){\b $\limx{+\infty}\arctan x=\df{\pi}2$}
	\end{overpic}
\end{center}

\subsubsection{【趋于有限值时的函数极限】}

{\bf 定义3.1.2}
\begin{enumerate}[(1)]
  \setlength{\itemindent}{1cm}
  \item $\limx{x_0}f(x)=A\Leftrightarrow\forall \e>0,\exists\delta>0,\forall
  x\in U_0(x_0,\delta),|f(x)-A|<\e$
  \item $\limx{x_0^+}f(x)=A\Leftrightarrow\forall \e>0,\exists\delta>0,\forall
  x\in(x_0,x_0+\delta),|f(x)-A|<\e$
  \item $\limx{x_0^-}f(x)=A\Leftrightarrow\forall \e>0,\exists\delta>0,\forall
  x\in(x_0-\delta,x_0),|f(x)-A|<\e$
\end{enumerate}

{\bf 例}(习题3.1-1)根据图形判断极限的存在性
\begin{center}
	\resizebox{!}{4cm}{\includegraphics{./images/ch3/limxf.pdf}}
\end{center}
\begin{enumerate}[(1)]
  \setlength{\itemindent}{1cm}
  \item $\limx{1}f(x)$\underline{不存在}
  \item $\limx{2}f(x)$\underline{$=1$}
  \item $\limx{1}f(x)$\underline{$=0$}
\end{enumerate}

{\bf 思考:}
\begin{itemize}
  \setlength{\itemindent}{1cm}
  \item 为什么在定义中要求$0<|x-x_0|<\delta$,而不是$|x-x_0|<\delta$?
  {\it (因为极限表示一种趋势,与函数在该点处的取值无关)}
  \item $f(x_0+0),f(x_0-0)$与$f(x_0)$是何关系?
  {\it(前两者分别表示左右极限,最后一个是函数值,三者相互无关)}
\end{itemize}

{\bf 例:}证明
\begin{enumerate}[(1)]
  \setlength{\itemindent}{1cm}
  \item $\limx{x_0}\sin x=\sin x_0$
  \item $\limx{1}x^2=1$
%   \item $\limx{1}\df1x=1$
\end{enumerate}

{\bf 证:}(1)对任意$\e>0$,令{\b$\delta=\e$},则当$0<|x-x_0|<\delta$时,有
$$|\sin x-\sin x_0|=2\left|\cos\df{x+x_0}2\right|
\left|\sin\df{x-x_0}2\right|\leq|x-x_0|<\delta=\e,$$
由函数极限定义,即证。

(2) 对任意$\e>0$,令{\b$\delta=\min\left\{\df12,\e\right\}$},
\ps{令$\delta\leq\df12$,作用在于使得$|x+1|$有界}
则当$0<|x-1|<\delta$时,有
$$|x^2-1|=|(x+1)(x-1)|<\df32\cdot\delta<\df32\e,$$
由函数极限定义,即证。

\begin{shaded}
	{\bf 【函数极限的反面说法】}
	
	\begin{itemize}
	  \item 当$x\to x_0$时$f(x)$不以$A$为极限 
	    $$\exists\e_0>0,\forall\delta>0, \exists x^*\in
	    U_0(x_0,\delta),|f(x^*)-A|\geq\e_0$$ 
	  \item 当$x\to x_0$时$f(x)$无极限
		$$\forall A\in\mathbb{R},\exists\e_0>0,\forall\delta>0, \exists x^*\in
	    U_0(x_0,\delta),|f(x^*)-A|\geq\e_0$$ 
	\end{itemize}
	
	{\bf 例:}证明:Dirichlet函数在任意点处无极限。
\end{shaded}

\subsection{函数极限的基本性质}

\subsubsection{【唯一性】}

{\bf 定理3.1.2:}函数极限若存在,必唯一。

\subsubsection{【有界性】}

{\bf 定理3.1.3:}
\begin{enumerate}[(1)]
  \setlength{\itemindent}{1cm}
  \item 若$\limx{+\infty}f(x)=A$,则$f(x)$当$x$充分大时有界
  \item 若$\limx{x_0}f(x)=A$,则$f(x)$在$x_0$的某去心领域内有界
\end{enumerate}

{\bf 注意:}函数极限有界性和数列极限有界性的叙述存在的差异:
{\it\b 数列收敛则数列整体有界,函数收敛只能说明在趋近的过程中有界!}
\ps{数列极限中的数列整体有界,
是由数列自身的“稀疏”特性所决定的}

\subsubsection{【保号性】}

{\bf 定理3.1.4:}
\begin{enumerate}[(1)]
  \setlength{\itemindent}{1cm}
  \item 若$\limx{+\infty}f(x)=A>0$,则当$x$充分大时,$f(x)>0$
  \item 若$\limx{x_0}f(x)=A>0$,则在$x_0$的某去心领域内,$f(x)>0$
\end{enumerate}

{\bf 课堂思考:}
\begin{enumerate}[(1)]
  \setlength{\itemindent}{1cm}
  \item 若$x\to x_0$时,$f(x)$有极限,$g(x)$无极限,则当$x\to x_0$时,以下哪些函数必无极限:
  $$f(x)g(x),\quad [g(x)]^2,\df{g(x)}{f(x)}, f(x)+g(x)$$ 
  \item 若$\limx{x_0}g(x)=A,\lim\limits_{u\to A}f(u)=B$,是否必有
  $$\limx{x_0}f[g(x)]=B\;?$$
  \item 若$\limx{x_0}f(x)g(x)=0$,则当$x\to
  x_0$时, $f(x),$ $g(x)$之一必趋于$0$ ({$\times$})
\end{enumerate}

{\bf 例:}设$\limx{x_0}f(x)=A$,用定义证明:$\limx{x_0}[f(x)]^3=A^3$

{\bf 证:}对任意$\e>0$,由$\limx{x_0}f(x)=A$,存在$\delta>0$,对任意
$0<|x-x_0|<\delta$,有
$$|f(x)-A|<\e.$$
不妨设$\e<1$,由上式可知$0<|x-x_0|<\delta$时,有$|f(x)|<|A|+1$。

综上,记$C=(|A|+1)^2+(|A|+1)|A|+A^2$,则当$0<|x-x_0|<\delta$时,总有
\begin{align}
	|f^3(x)-A^3|&=|f(x)-A||f^2(x)+f(x)A+A^2|\notag\\
	&\leq |f(x)-A|[|f^2(x)|+|f(x)||A|+A^2]\notag\\
	&<\e[(|A|+1)^2+(|A|+1)|A|+A^2]=C\e\notag,
\end{align}
有极限的定义,即证。

\section{函数极限的判敛}

\subsection{四则运算}

{\bf 定理3.2.1:}若函数极限存在,则极限运算可以和四则运算交换次序。

{\bf 注:}把握两点
\begin{itemize}
  \setlength{\itemindent}{1cm}
  \item 有限次四则运算
  \item 可以推广到初等函数
\end{itemize}

\subsection{复合函数的极限}

{\bf 定理3.2.2:}设有复合函数$y=f[g(x)]$,其中\ps{与子数列的收敛性质作对比}
$$\limx{x_0}g(x)=u_0,\quad\lim\limits_{u\to u_0}f(u)=A,$$
在$x_0$附近$g(x)\ne u_0$,则
$$\limx{x_0}f[g(x)]=A.$$

{\bf 注:}
\begin{itemize}
  \setlength{\itemindent}{1cm}
  \item 直观理解:若相关极限都存在,则极限运算可以和函数运算交换次序
  \item {\b 为什么要求“在$x_0$附近$g(x)\ne u_0$”?}
  
  {\it 答:因为$f(x)$可能在$x_0$处的定义与极限值无关,例如:
  $$f(x)=\left\{\begin{array}{ll}
  1,&x\ne0\\0,&x=0
  \end{array}\right.$$
  $g(x)\equiv 0$,则$f(g(x))\equiv0$,从而$\limx{0}f(g(x))=0$,
  而不是如定理所述$\limx{0}f(g(x))=1$}
  \item 可以类似地推广到$x\to\infty$的情形
\end{itemize}

\subsection{函数极限与数列极限的关系}

{\bf 定理3.2.3}(Hiene定理)$\limx{\Delta}f(x)=A\Leftrightarrow
$若数列$\{x_n\}$满足:$x_n\to \Delta(n\to$ $\infty)$,则
$$\limn f(x_n)=A$$

\begin{itemize}
  \setlength{\itemindent}{1cm}
  \item 以上$\Delta$对应于函数极限的六种不同趋势 
  \item {\bf 用途一:}证明极限不存在性
  \item {\bf 用途二:}利用函数极限计算对应的数列极限
\end{itemize}

{\bf 教材3.2.2节-例8:}{\b 证明:$f(x)=\sin\df 1x$当$x\to 0$时无极限。}

{\bf 证}:令
$$x^{(1)}_n=\df1{n\pi},\quad
x^{(2)}_n=\df1{2n\pi+\df{\pi}2},\quad n\in\mathbb{Z}_+$$ 
显然,$\limn x^{(1)}_n=\limn x^{(2)}_n=0$,且
$$f(x^{(1)}_n)\equiv0,\quad f(x^{(2)}_n)\equiv1,$$
进而
$$\limn f(x^{(1)}_n)=0\ne1=\limn f(x^{(2)}_n),$$
由Henie定理,即知极限不存在。

\begin{center}
	\resizebox{!}{5cm}{\includegraphics{./images/ch3/sin1x.pdf}}
	
	{\it 以上所取的$x^{(1)}_n$和$x^{(2)}_n$分别取自图上绿色和黄色水平线与
	函数$\sin\df1x$的交点,事实上,对于任意的$a\in[-1,1]$(红色水平线),
	都可以取得类似的点列,用于构造证明所需的收敛到不同值得函数值数列,请思考以下如何写
	出其表达式?
	$$x^{(a)}_n=\df1{2n\pi+\arcsin a},\quad n\in\mathbb{Z}_+$$}
\end{center}

{\bf 例}(习题3.2-7)证明Dirichlet函数在任意点处无极限。

\begin{shaded}
	{\bf 命题:}对任意$x_0\in\mathbb{R}$,都存在这样的两个数列
	$\{x_n^{(1)}\}$和$\{x_n^{(2)}\}$,满足
	$$x_n^{(1)}\in\mathbb{Q},\;x_n^{(2)}\notin\mathbb{Q}\;(n\in\mathbb{Z}_+),$$
	且
	$$\limn x_n^{(1)}=\limn x_n^{(2)}=x_0.$$
	
	{\bf 思考:}试给出$\{x_n^{(1)}\}$和$\{x_n^{(2)}\}$的通项表达式。
	
	{\bf 参考答案:}
% 	1)若$x_0\in\mathbb{Q}$,可令
% 	$$x_n^{(1)}=x_0+\df1n,\quad
% 	x_n^{(2)}=x_0+\df{\sqrt2}n,\;(n\in\mathbb{Z}_+)$$
% 	2)若$x_0\notin\mathbb{Q}$,可令
% 	$$x_n^{(1)}=[x_0]+x_0\mbox{\it 小数部分的前}n\mbox{\it 位},
% 	\;(n\in\mathbb{Z}_+)$$
% 	例如$x_0=\pi=3.1415926\ldots$,则
% 	$$x_1^{(1)}=3.1,\;x_2^{(1)}=3.14,\;x_3^{(1)}=3.141,\;
% 	x_4^{(1)}=3.1415,\;\ldots$$
% 	另,
% 	$$x_n^{(2)}=x_0+\df1n,\;(n\in\mathbb{Z}_+)$$
% 	
% 	{\bf 又解:}
	对任意$x_0$,定义
	$$x_n=[x_0]+x_0\mbox{\it 小数部分的前}n\mbox{\it 位},$$
	$$x_n^{(1)}=x_n+\df1n,$$
	$$x_n^{(2)}=x_n+\df{\sqrt2}n,$$
	显然$\{x_n^{(1)}\}$和$\{x_n^{(2)}\}$分别为趋于$x_0$的有理和无理数列,而
	$$\limn D(x_n^{(1)})=1\ne0=\limn D(x_n^{(2)}),$$
	从而由Hiene定理,即证。
\end{shaded}

{\bf 例:}设在$(0,+\infty)$上,恒有$f(x^2)=f(x)$,且
$$\limx{0^+}f(x)=\limx{+\infty}f(x)=f(1)$$
证明:$f(x)=f(1)\,(x\in(0,+\infty))$

\subsection{夹逼定理}

{\bf 定理3.2.5:}设在$x_0$的某邻域内,恒有
$$\varphi(x)\leq f(x)\leq\psi(x), $$
且$\limx{x_0}\varphi(x)=\limx{x_0}\psi(x)=A$,则
$$\limx{x_0}f(x)=A.$$

{\bf 教材3.2.3节-例9}(重要极限一)证明:
$$\limx{\infty}\left(1+\df 1x\right)^x=e$$

[思路]:利用数列极限$\limn\left(1+\df1n\right)^n=e$,构造夹逼所需的函数
$$\left(1+\df1{[x]+1}\right)^{[x]}<\left(1+\df1x\right)^x
<\left(1+\df1{[x]}\right)^{[x]+1},$$
然后利用函数极限的运算性质,证明不等式两端函数当$x\to+\infty$时,极限均为$e$。

$x\to-\infty$的结果利用极限运算的性质易证。

\begin{shaded}
{\bf 例}(另一种证明思路)令$f_u=x^ue^{-x}(x\geq 0)$,对于确定的$u$,证明:
\begin{enumerate}[(1)]
  \setlength{\itemindent}{1cm}
  \item $f_u(x)$在$x=u$处取最大值;
  \item 由$f_u(x)>f_u(u+1)$和$f_{u+1}(u+1)>f_u(u)$推出
  $$\left(\df{u+1}u\right)^u<e<\left(\df{u+1}u\right)^{u+1}$$
  \item $\df u{u+1}e<\left(\df{u+1}u\right)^u<e$,由此推出
  $$\limx{+\infty}\left(1+\df 1u\right)^u=e$$
\end{enumerate}
\end{shaded}

{\b{\bf 例:}{\it 这些都是重要极限的应用,须牢记有关结果!!!}
\begin{enumerate}[(1)]
  \setlength{\itemindent}{1cm}
  \item $\limx{0}(1+x)^{1/x}=1$ 
  \item $\limx 0(1+\sin x)^{1/\sin x}=1$ 
  \item $\limx 0\df{\ln(1+ax)}{x}=a$
  \item $\limx 0\df{e^{ax}-1}{x}=a$ 
  \item $\limx 0\df{a^x-1}{x}=\ln a$ 
  \item $\limx 0\df{(1+x)^a-1}x=a$
  \item $\limn n(\sqrt[n]{a}-1)=\ln a$ 
\end{enumerate}}

{\bf 教材3.2.3节-例11}(重要极限二)证明:
$$\limx{0}\df {\sin x}x=1$$

{[提示]}:
\begin{center}
	\resizebox{!}{5cm}{\includegraphics{./images/ch3/xsintan.pdf}}
\end{center}

如图,显然弧$AB$的长度大于直线$BD$,即$\sin\theta<\theta$;又扇形$ABO$的面积$\df12\theta$
小于三角形$ACO$的面积$\df12\tan\theta$,从而$\theta<\tan\theta$

{\b{\bf 例:}{\it 牢记!!!}
\begin{enumerate}[(1)]
  \setlength{\itemindent}{1cm}
  \item $\limx{0}\df{\sin\sin x}{\sin x}=1$ 
  \item $\limx 0\df{1-\cos x}{x^2}=\df12$ 
  \item $\limx 0\df{\sin mx}{\sin nx}=\df mn$
  \item $\limx 0\df{\tan x}{x}=1$
  \item $\limx 0\df{\arcsin x}x=1$
  \item $\limx 0\df{\arctan x}x=1$
  \item $\limx a\df{\sin x-\sin a}{x-a}=\cos a$
\end{enumerate}}

{\bf 例:}设$f(x)=\sum\limits_{i=1}^na_i\sin
ix$,其中$a_i(i=1,2,\ldots,n)$为常数,且对任意$x\in\mathbb{R}$, $|f(x)|\leq |\sin x|$,证明:
$$\left|a_1+2a_2+\ldots+na_n\right|\leq 1$$

[提示]:{\it 极限运算可以和绝对值运算交换次序!}

\section{无穷大、无穷小和函数的渐近线}

在求解(函数)极限的问题中,分子分母同时趋于零
\ps{所谓不定是相对诸如有界量乘以无穷小(趋于零)、有界量除以无穷大一类容易确定的形式而言的}
(例如:$\limx{0}\df{\cos
x-\cos2x}{x^2}$),或者一个趋于零的函数和一个趋于无穷的函数的乘积(例如:$\limx{+\infty}x^2e^-x$)
的极限通常是较难求解的,这类问题我们称之为“$\df00$”和“$0\cdot\infty$” 型的{\it 不定式(极限)}!


{\bf 思考:}除了“$\df00$”和“$0\cdot\infty$”型的不定式,还可能有其他形式的不定式吗?

{\bf 答:}有,型如:“$\df{\infty}{\infty}$”、“$1^{\infty}$”、“$0^0$”,例如:
$\limx{+\infty}\df{\ln x}{x}$,$\limx{\infty}\left(1+\df1{x}\right)^{\sin
x}$,$\limx{0}x^{2x}$

\subsection{无穷大和无穷小}

{\bf 定义3.3.1:}$f(x)$是$x\to\Delta$时的无穷小$\Leftrightarrow\limx{\Delta}f(x)=0$

{\bf 注:}$\Delta$可任意代表$\infty,\;+\infty,\;-\infty,\;x_0,\;x_0^+,\;x_0^-$之一

{\bf 定理3.3.1-3.3.2}(无穷小的性质)
\begin{enumerate}[(1)]
  \setlength{\itemindent}{1cm}
  \item $\limx{\Delta}f(x)=A\in\mathbb{R}\Leftrightarrow
  f(x)-A$是$x\to\Delta$时的无穷小
  \item $x\to\Delta$时的有界函数与无穷小之积仍为$x\to\Delta$时的无穷小
  \item $x\to\Delta$时的有限个无穷小之和(积)仍为$x\to\Delta$时的无穷小
\end{enumerate}

{\bf 定义3.3.2:}$f(x)$是$x\to\Delta$时的无穷大$\Leftrightarrow\limx{\Delta}\df 1{f(x)}=0$,
可记为:
$$\limx{\Delta}f(x)=\pm\infty$$

{\bf 注:}{\b 无穷大有正负之分!!}

{\bf 【渐近线】}\ps{具体内容请阅读教材自学!}

$x\to x_0$时的无穷大意味着存在{\it 铅直渐近线};$x\to\pm\infty$时的无穷大
{\it 可能}意味着存在{\it 斜渐近线},例如:若$y=f(x)$当$x\to+\infty$时的
存在斜渐近线$y=kx+b$,则
$$k=\limx{+\infty}\df{f(x)}x,\quad b=\limx{+\infty}[f(x)-kx].$$
注意,{\b $x\to\pm\infty$时的斜渐近线可能是不同的!}例如:$y=x\arctan x$趋于
$x\to\pm\infty$时的斜渐近线分别为$y=\pm\df{\pi}2x$.

% {\bf P129-例4:}证明:$x+\sin x$是$x\to\infty$时的无穷大

{\bf 定理3.3.3:}在$x\to\Delta$的同一过程中:
\begin{enumerate}[(1)]
  \setlength{\itemindent}{1cm}
  \item 有界函数与无穷大之和仍为无穷大
  \item 有限个无穷大之乘积仍为无穷大({\it 但可能反号})
\end{enumerate}

{\bf 例:}证明:$f(x)=a_0x^n+a_1x^{n-1}+\ldots+a_n(n\in\mathbb{N})$
是$x\to\infty$时的无穷大,其中:$a_0,a_1,\ldots,a_n\in\mathbb{R},a_0\ne 0$

\begin{shaded}
	{\bf 无穷大之间的比较!}
	
	当$x\to+\infty$时,存在如下的大小关系:设$a>0$,
	$$\ln x<<x^a<<e^x<<\Gamma(x)<<x^x$$
	其中$\Gamma-$函数定义为$\Gamma(x)=\dint_0^{+\infty}t^{x-1}e^{-t}\d t$,
	满足$\Gamma(n)=(n-1)!,\;(n\in\mathbb{Z}_+)$
\end{shaded}

\subsection{无穷小的比较}

{\bf 定义3.3.3:}设$y_1,y_2$均为$x\to\Delta$时的无穷小,
$\limx{\Delta}\df{y_1}{y_2}=A$为常数
\begin{enumerate}[(1)]
  \setlength{\itemindent}{1cm}
  \item $A=0$,称$y_1$为$y_2$当$x\to\Delta$时的{\it 高阶(高级)无穷小},记为:
  \ps{\b 等式中出现无穷小符号$\circ$或$\mathrm{O}$时,等号的含义,准确地
  解释应该是指左侧函数为右侧符号所标识的函数之一!}
  $$y_1=\circ( y_2)\;\;(x\to\Delta)$$
  \begin{itemize}
    \item {\it 无穷小}可记为:
    $$y_1=\circ(1)\;\;(x\to\Delta)$$
  \end{itemize}
  \item $A\ne 0$,称$y_1$为$y_2$当$x\to\Delta$时的{\it 同阶(同级)无穷小},记为:
  $$y_1=\mathrm{O}( y_2)\;\;(x\to\Delta)$$
  \item $A=1$,称$y_1$为$y_2$当$x\to\Delta$时的{\it 等价无穷小},记为:
  $$y_1\sim y_2\;\;(x\to\Delta)$$
\end{enumerate}

{\bf 注意:}{\it\b 所有的无穷小都必须是和某个$x\to\Delta$相关的,因此讨论
无穷小性质或者对其进行计算、推导时,必须说明是在怎样的$x$变化趋势之中!!!}

\begin{shaded}
	{\bf 【高阶无穷小的推导关系】}
	对包含有无穷小符号的表达式进行推导,不可避免地会涉及一些化简、合并运算,以下是一些常用
	的化简形式,以$x\to 0$时的无穷小为例:
	\begin{enumerate}[(1)]
  	  \setlength{\itemindent}{1cm}
  	  \item $C\cdot\circ(x^n)=\leadsto\circ(x^n)\;(C\in\mathbb{R}\mbox{为常数})$
	  \item $x^n=\circ(x^m)\quad (m<n)$ 
	  \item $\circ(x^n)=x^n\cdot\circ(1)$
	  \item $\circ(x^n)\pm\circ(x^n)=\circ(x^n)$
	  \item $x^n\cdot\circ(x^m)=\circ(x^{m+n})$ 
	  \item $\circ(x^n)+\circ(x^m)=\circ(x^n)\;(m\geq n)$  
	  \item $\circ(x^n)\cdot\circ(x^m)=\circ(x^{m+n})$
	\end{enumerate}
	
	{\it\b 要正确理解以上的表达式,必须注意到其中的等号的意义与我们惯常使用的有所不同,其含义
	不是指表达式两边恒等,而应该理解成类似集合运算中的属于关系,例如:$x^n=\circ(x^m)\quad (m<n)$ 
	应该理解为$x_n$是所有$x^m$的高阶无穷小中的一个(属于$\circ(x^m)$所标识的一族函数);
	$x^n\cdot\circ(x^m)=\circ(x^{m+n})$应该理解为,一个$x^m$的高阶无穷小,乘以
	$x^n$,所得函数是$x^{m+n}$的一个高阶无穷小中(属于$\circ(x^{m+n})$所标识的
	一族函数)。事实上,所有包含了高阶无穷小的等式,其中的等号都应该做类似的理解!}
	
	在今后的极限计算中,我们可能会用到Taylor展开式,例如:
	\begin{align}
		\limx{0}\df{e^{x^2}-\cos x}{x^2}&=\limx{0}
		\df{1+x^2+\circ(x^2)-1+\df{x^2}2+\circ(x^2)}{x^2}\notag\\
		&=\limx{0}\df{\df32x^2+\circ(x^2)}{x^2}=\df32\notag
	\end{align}
\end{shaded}

\subsection{(等价)无穷小代换}

{\bf 等价无穷小的基本性质:}
\begin{itemize}
  \setlength{\itemindent}{1cm}
  \item {\bf 自反性:} $y\sim y$ 
  \item {\bf 对称性:} $y_1\sim y_2\Rightarrow y_2\sim y_1$ 
  \item {\bf 传递性:} $y_1\sim y_2,y_2\sim y_3\Rightarrow y_1\sim y_3$ 
\end{itemize}

{\bf 定理3.3.4:}设$y_1\sim y_2\;(x\to\Delta)$,则
$$\limx{\Delta}y_1y_3=A\quad\Leftrightarrow\quad\limx{\Delta}y_2y_3=A$$

{\bf 注:}极限“乘法因子”中的等价无穷小可相互替代

{\b{\bf 【常用无穷小代换】}:$x\to 0$时
\begin{enumerate}[(1)]
  \setlength{\itemindent}{1cm}
  \item $x\sim \sin x\sim \tan x$ 
  \item $x \sim\arcsin x\sim\arctan x$ 
  \item $1-\cos x\sim \df 12 x^2$ 
  \item $(1+x)^a-1\sim ax$ 
  \item $\ln(1+x)\sim x$ 
  \item $a^x-1\sim x\ln a\;(a>0)$
\end{enumerate}}

{\bf 例:}计算极限
\begin{enumerate}[(1)]
  \setlength{\itemindent}{1cm}
  \item $\limx{0}\df{\arctan x}{\sin 4x}=\df14$ 
  \item $\limx{0}\df{\ln\cos ax}{\ln\cos bx}=\limx{0}\df{1-\cos ax}{1-\cos
  bx}=\df{a^2}{b^2}$
  \item $\limx{0}\df{\cos x(e^{\sin x}-1)^4}{\sin^2 x(1-\cos x)}=2$ 
  \item $\limx{0}\df{\sin x-\tan x}{x^3}=\limx{0}\df{\cos x-1}{x^2\cos
  x}=-\df12$
\end{enumerate}

{\bf\b 极限中的“加法因子”不能进行无穷小代换!}例如:{\b 典型的错误
$$\limx{0}\df{\sin x-\tan x}{x^3}=\limx{0}\df{x-x}{x^3}=0$$
}

\subsection{斜渐近线}

若$y=f(x)$当$x\to+\infty$时以$y=kx+b$为斜渐进线,则有
$$k=\limx{+\infty}\df{f(x)}{x}=k,\quad b=\limx{+\infty}[f(x)-kx]$$

{\bf 例:}求曲线$y^2-x^2=2x$的渐近线。

{\bf 教材3.3.4节-例10:}求函数$f(x)=\df{2x^2-3}{x+1}$的渐近线。

\begin{center}
	\resizebox{!}{5cm}{\includegraphics{./images/ch3/asyx.pdf}}
\end{center}

\section{函数的连续性}

\subsection{定义}

{\bf 定义3.4.1:}函数$f(x)$在$x_0$连续:
$$\limx{x_0}f(x)=f(x_0)$$

\begin{itemize}
  \item $f(x)$在$x_0$有定义 
  \item $\limx{x_0}f(x)$存在 
  \item $f(x_0)=f(x_0+0)=f(x_0-0)$
\end{itemize}

{\bf 注:}必要时,可以只考虑函数在某点一侧的连续性,即所谓左(右)连续,
参见教材定义3.4.2-3

{\bf 例:}证明:$f(x)=xD(x)$只在$x=0$处连续。

[提示]:先证明$xD(x)$仅在$x=0$处极限存在。

{\bf 教材3.4.1节-例3:}设$f(x)$在$(-\infty,+\infty)$上有定义,
且对任意$x,y\in (-\infty,+\infty)$,有
$$f(x+y)=f(x)+f(y),$$
则$f(x)$在$(-\infty,+\infty)$上连续,当且仅当$f(x)$在$x=0$连续。

% \begin{center}
% 	\resizebox{!}{4cm}{\includegraphics{./images/ch3/notCont.pdf}}
% \end{center}

{\bf 定义3.4.4:}设$x_0$是$f(x)$的间断点(不连续点),对其分类定义如下:
\begin{enumerate}[(1)]
  \setlength{\itemindent}{1cm}
  \item {\bf 第一类间断点:}$f(x_0+0),f(x_0-0)$均存在
  \begin{itemize}
    \item {\it 跳跃间断点:}$f(x_0+0)\ne f(x_0-0)$,例:$y=[x]$当$x$为整数时
    \begin{center}
		\resizebox{!}{4cm}{\includegraphics{./images/ch3/roundx.pdf}}
 	\end{center}
    \item {\it 可去间断点:}$f(x_0)$无定义,或
    $$f(x_0)\ne f(x_0+0)=f(x_0-0)$$
    例:$y=\df{\sin x}x$在$x=0$处
    \begin{center}
		\resizebox{!}{4cm}{\includegraphics{./images/ch3/sinxox.pdf}}
 	\end{center}
  \end{itemize}
  \item {\bf 第二类间断点:}$f(x_0+0),f(x_0-0)$不同时存在
  \begin{itemize}
    \item {\it 无穷间断点:}某个单侧极限趋于无穷,例:$y=\tan x$和$y=\sec x$在$x=k\pi+\df{\pi}2$处
     \begin{center}
		\resizebox{!}{4cm}{\includegraphics{./images/ch3/tanx.pdf}}
		\resizebox{!}{4cm}{\includegraphics{./images/ch3/secx.pdf}}
 	\end{center}
    \item {\it 振荡间断点:}某个单侧极限不存在,且非无穷,例:$y=\sin\df1x$在$x=0$处
    \begin{center}
		\resizebox{!}{5cm}{\includegraphics{./images/ch3/sin1ox.pdf}}
 	\end{center}
  \end{itemize}
\end{enumerate}

{\bf 例:}指出以下函数的间断点,判断其类型
\begin{enumerate}[(1)]
  \setlength{\itemindent}{1cm}
  \item $y=\df{x^2-1}{x-1}$\quad{\it 可去}
  \item $y=\df{x^2+1}{x-1}$\quad{\it 无穷}
  \item $y=D(x)$\quad{\it 振荡}
\end{enumerate}

\subsection{基本性质}

{\bf 定理3.4.1-3.4.4:}
\begin{enumerate}[(1)]
  \setlength{\itemindent}{1cm}
  \item {\it 四则运算:} 四则运算仍保持函数的连续性 
  \item {\it 复合函数:} 连续函数的函数运算可以和极限运算交换次序 
  \item {\it 反函数:} 连续函数的反函数也连续 
  \item {\it 初等函数:} 初等函数在其定义域内连续
\end{enumerate}

\subsection{连续函数在有界闭区间上的性质}

\subsubsection{【最值定理】}

{\bf 定理3.4.5:}设$f(x)\in C[a,b]$,则$f(x)$在$[a,b]$上可取到最大和最小值。

{\it 也即,存在$\xi,\eta\in[a,b]$,是对任意$x\in[a,b]$,总有$f(\xi)
\leq f(x)\leq f(\eta)$}

{\bf 推论:}设$f(x)\in C[a,b]$,则$f(x)$在$[a,b]$上有界。

{\bf 例:}设$f(x)\in C[a,+\infty)$,且$\limx{+\infty}f(x)$存在,
证明:$f(x)$在$[a,+\infty)$上有界。

[提示]:首先利用函数极限的有界性证明,存在$M_1$和$X$,对任意$x>X$,有$|f(x)|<M_1$。
然后,利用连续函数在闭区间上的有界性,说明存在$M_2$,对任意$x\in[a,X]$,有$|f(x)|<M_2$。
综合即证。

\subsubsection{【介值定理】}

{\bf 定理3.4.6:}设$f(x)\in C[a,b]$,$M,m$分别为$f(x)$在$[a,b]$上的最大和最小值,
则对任意$\gamma\in[m,M]$,存在\ps{\b “存在”与“至少存在”意义是完全一致的!}
$\xi\in[a,b]$,使得$f(\xi)=\gamma$。

{\bf 推论}(零值定理/零点存在性)
设$f(x)\in C[a,b]$,且$f(a)f(b)<0$,则$f(x)$在$[a,b]$上必有零点。

{\bf 推论'}(零点存在性定理的推广)
\begin{enumerate}[(1)]
  \setlength{\itemindent}{1cm}
  \item {\b 设$f(x)\in C(a,b)$,且$f(a+0)\cdot f(b-0)<0$,则$f(x)$在$(a,b)$内有零点。} 
  \item {\b 设$f(x)\in C(-\infty,+\infty)$,且$f(-\infty)\cdot f(+\infty)<0$,
  则$f(x)$在$(-\infty,+\infty)$内有零点。}
\end{enumerate}

[提示]:利用极限的保号性证明。

{\bf 例:}设$a_0\ne 0$,证明:以下方程至少有一个实根\ps{\b 奇数次多项式方程至少有一个实根!}
$$a_0x^{2n+1}+a_1x^{2n}+\ldots+a_{2n}x+a_{2n+1}=0.$$

[提示]:$\limx{\infty}\df{f(x)}{x^{2n+1}}=a_0$,利用极限保号性说明
$f(x)$当$x$充分大(小)时,分别有取正和取负的值。

{\bf 例:}已知$f(x)\in C[0,3]$,且$f(0)=f(3)$,证明:至少存在
一个$\xi\in[0,2]$,使得$f(\xi+1)=f(\xi)$

{\bf 证:}设$F(x)=f(x+1)-f(x)$,显然$F(x)\in C[0,2]$。

注意到若$f(0)=f(1)$或$f(1)=f(2)$或$f(2)=f(3)$,结论显然成立。故以下设$f(0)\ne f(1),
f(1)\ne f(2),f(2)\ne f(3)$。

不妨设$f(1)>f(0)$,则$F(0)>0$。此时,若$f(2)<f(1)$,则$F(1)<0$,
由连续函数的介值定理可知必有$\xi\in[0,1]$,使得$F(\xi)=0$。
又若$f(2)>f(1)$,则$F(1)>0$,
$$F(2)=f(3)-f(2)=f(0)-f(2)<f(1)-f(2)=-F(1)<0,$$
同样由连续函数的介值定理可知必有$\xi\in[1,2]$,
使得$F(\xi)=0$。

综上,必有$\xi\in[0,2]$,使得$F(\xi)=0$,也即$f(\xi+1)=f(\xi)$。

{\bf 例:}设$G$是第一象限内的一个有界区域(其边界为连续的简单闭曲线),$ABCD$表示
它的一个外接矩形。矩形的一条边与$x$轴的夹角记为$\theta$,如果对于任意$\theta\in
[0,\pi/2]$,$G$总可以被某个外接矩形所围,证明:至少存在某个$\theta_0\in[0,\pi/2]$,
使得与之对应的外接矩形恰好为正方形。({\it 推论:任一平面有界区域都可以内接于某个正方形})

\begin{center}
	\resizebox{!}{6cm}{\includegraphics{./images/ch3/sqOut.pdf}}
\end{center}

[提示]:记$l_1(\theta),l_2(\theta)$分别为$AB$和$BC$的长度,显然
$$l_1(0)=l_2(\pi/2),\quad l_2(0)=l_1(\pi/2).$$
令$f(\theta)=l_1(\theta)-l_2(\theta)$,由介值定理可以证明$f(\theta)$存在零点。

{\bf 例:}证明:给定任意平面有界区域,以及一个向量,一定存在一条与该向量平行的直线
平分该区域。

{\bf 例:}证明:给定任意平面有界区域,一定存在两条相互垂直的直线,将其四等分。

[提示]:设其中一条直线的极角为$\theta$,则另一条的为$\theta+\pi/2$。
分别记两条直线为$l_{\theta}$和$l_{\theta+\pi/2}$,显然可以使得二者都平分给定区域。
设此时四个区域的面积按顺时针方向依次为$S_1,S_2,S_3,S_4$,且由平分可满足
$$A_1+A_2=A_3+A_4,\quad A_1+A_4=A2+A_3,$$

\begin{center}
	\resizebox{!}{6cm}{\includegraphics{./images/ch3/4cut.pdf}}
\end{center}

由此可得
$$A_1=A_3,\quad A_2=A_4.$$
进而,只需适当取$\theta$,使得$A_1=A_2$即可。

记$f(\theta)=A_1(\theta)-A_2(\theta)$,不妨设$f(0)>0$,从而
$$f(\pi/2)=A_1(\pi/2)-A_2(\pi/2)=A_4(0)-A_1(0)=A_2(0)-A_1(0)=-f(0)<0,$$
由介值定理,即证。

{\bf 例:}任意金属圆环上,总存在相对的两点温度相同。
({\it 圆环可以看成赤道,温度即为两点的气温})

[提示]:默认假设温度总是连续分布的。

{\bf 思考:}构造更多类似的例题,用介值定理证明某种状态的存在性。

% \section*{补充例题}
% \addcontentsline{toc}{section}{补充例题}
% 
% {\bf 例}:证明函数$f(x)=\left\{\begin{array}{ll}
% x,\;&x\in\mathbb{Q}\\ 0,\;&\mathrm{else}
% \end{array}\right.$
% 仅在$x=0$连续
% 
% {\bf 证:}显然$f(0)=0$。注意到
% $$0<|f(x)|\leq |x|,$$
% 而$\lim\limits_{x\to 0}|x|=0$,故由夹逼定理
% $$\lim\limits_{x\to 0}f(x)=0=f(0),$$
% 也即$f(x)$在$x=0$连续。
% 
% \bigskip
% 
% 设$x_0\ne 0$,下证$\lim\limits_{x\to x_0}f(x)$不存在,进而可知
% $f(x)$在$x_0$处不连续。事实上,若设$\lim\limits_{x\to x_0}f(x)$存在,
% 则
% $$\lim\limits_{x\to x_0}D(x)=\frac{\lim\limits_{x\to x_0}f(x)}
% {\lim\limits_{x\to x_0}x}$$
% 也存在,从而与$D(x)$在任意点处极限不存在矛盾。
% 
% 综上,$f(x)$仅在$x=0$连续。
% 
% \newpage
% 
% \newpage

\section{小结}

函数极限与数列极限最明显的不同,就是所考虑的自变量变化过程,从单一的趋于正无穷扩展到了
趋于无穷和趋于有限值两大类,共六种不同的过程。

在学习函数极限的概念,或者说“$\e-X$”(“$\e-\delta$”)定义时,一定要注意和数列极限
的“$\e-N$”定义进行类比,严格来说,在所有这些不同极限的定义中,有所变化的只是所趋近的目标
(无穷或有限值)的邻域概念,定义的结构和内涵都是完全一样的,即:{\it 只要自变量充分靠近
趋近的目标,则函数值也会充分靠近某个稳定的值。}

正因文定义上的相似,所以在基本的性质上,函数极限和数列极限也没有本质的不同。需要注意
的仅仅是有些性质在表述上的变化,例如{\it 函数极限的有界性}和数列极限的有界性在表述上就存在
明显的差别,能够正确表述函数极限的有界性,是本章学习的第一个有挑战的知识点。

和数列极限一样,本章学习的重点是函数极限的判敛和计算问题,特别是后者。我们所学习的{\it 两个
重要极限和众多由它们衍生的常用极限},是我们今后计算很多极限的基础。在此需要提醒同学们的是,
拿到一个极限问题,首先要判断其基本的形态(例如:“$0/0$”,“$1^{\infty}$”),然后
根据其形态上的特点选择运算的方向(或者说变换的目标),最终将其化为一些我们常见或者说
基本的极限。这一点,常常被很多同学所忽视。

无穷小的概念是微积分真正成为一门新学科,在数学理论中确立其独特地位的源头。Newton最早
有意识地使用了无穷小的概念,并利用其种种性质达到简化计算和推理的目的,但无穷小的有关理论是在
其诞生后近百年后才真正稳固的。也正因为如此,无穷小也是本章学习的难点和重点。所谓的难点,
主要体现在对无穷小的“运算”方法(例如:$\circ(x^2)+\circ(x)=\circ(x)$)的理解,以及
何时可以在有关的计算(不仅仅是极限问题)中准确地将其忽略不计。至于其重要性可以说是不言而喻
的,{\it 无穷小代换}的思想对于我们计算各种不定式的极限给出了崭新的思路和表述方法,极大提高了
极限计算的效率。在本章之外,无穷小的故事还会继续。

函数的连续性概念源于生活的直觉,但并不等同于直觉,事实上,在现实中就没有“在某一点连续”
的直觉对应。{\it 函数连续的本质是,函数运算和极限运算可以交换次序},换句话说,对于在某点连续的
函数而言,计算其在该点的极限和计算其函数值没有什么不同。

{\it 有界闭区间上连续函数的性质}是本章最富有趣味也是颇具挑战性的问题,特别是和我们后续
学习中的中值定理相结合,将展示给大家极为丰富和多样化的挑战性问题,并成为本门课程的又一个
标志性的难点。

\newpage

\section*{课后作业}
\addcontentsline{toc}{section}{课后作业}

{\bf 【必作题】}

\begin{itemize}
  \item 习题3.1:4(1,3),7,15
  \item 习题3.2:2,3,4,5
  \item 习题3.3:2,4,5,7(2,6),8,9,10(3,4)
  \item 习题3.4:5(1-3),9,15,17
\end{itemize}

\bigskip

\hrule

\bigskip
\bigskip

{\bf 【思考题】}
\begin{itemize}
  \item 习题3.1:10,11,12
  \item 习题3.2:6,7
  \item 证明:$$\limn\left\{\lim\limits_{m\to\infty}\left[\cos^{2m}(n!\pi
	x)\right]\right\}=D(x)$$
	其中$D(x)$为Dirichlet函数
%   \item 自学3.3.3节:渐近线 
  \item 习题3.3:1,6,12,13
  \item  计算极限
	\begin{enumerate}[(1)]
% 	  \setlength{\itemindent}{1cm}
	  \item $\limx{+\infty}(\sqrt{x^2+x+1}-\sqrt{x^2+x-1})$ 
	  \item $\limx{0}\df{\sqrt[3]{1+x}-1}{x}$ 
	  \item $\limx{0}\df{\sqrt{1-\cos x^2}}{1-\cos x}$ 
	  \item $\limx{0}\df{1-\cos x\cos 2x}{1-\cos x}$
	  \item $\limx{\pi /4}(\tan x)^{\tan 2x}$ 
	  \item $\limx{0}\left(2e^{\frac{x}{x+1}}-1\right)^{\frac{x^2+1}{x}}$ 
	  \item $\limx{+\infty}\left(\sqrt{x^2+x}-\sqrt[3]{x^3+x^2}\right)
% 	  =
% 	  \limx{+\infty}x\left(1+\df1x\right)^{\df13}\left[\left(
% 	  1+\df1x\right)^{\df16}-1\right]=\df16
		$ 
	  \item $\limx{+\infty}\left(\df{x^2-1}{x^2+1}\right)^{x^2}$
	  \item $\limx{0}\left(\df{a^x+b^x+c^x}{3}\right)^{1/x}\,(a,b,c>0)$ 
	  \item $\limx{0}\left(\df{a^{x+1}+b^{x+1}+c^{x+1}}{a+b+c}\right)^{1/x}
	  \,(a,b,c>0)$ 
	  \item $\limx{0}\df{\tan(\tan x)-\sin(\sin x)}{\tan x-\sin x}$
	\end{enumerate}
  \item 习题3.4:1,3,13,14,16,19
  \item 设$a_1<a_2<\ldots<a_n$,证明以下方程有$n-1$个实根
	$$\df 1{x-a_1}+\df 1{x-a_2}+\ldots+\df 1{x-a_n}=0.$$
\end{itemize}

\newpage

\section*{补充例题}
\addcontentsline{toc}{section}{补充例题}

{\bf 例:}设当$x\to
x_0$时,$f(x)$不是无穷大,则下列命题中正确的是(D)
\begin{enumerate}[(A)]
  \setlength{\itemindent}{1cm}
  \item 若$g(x)$是$x\to x_0$时的无穷小,则$f(x)g(x)$必为$x\to x_0$时的无穷小
  \item 若$g(x)$不是$x\to x_0$时的无穷小,则$f(x)g(x)$必不是$x\to x_0$时的无穷小
  \item 若$g(x)$在$x_0$的某邻域内无界,则$f(x)g(x)$必为$x\to x_0$时的无穷大
  \item 若$g(x)$在$x_0$的某邻域内有界,则$f(x)g(x)$必不是$x\to x_0$时的无穷大
\end{enumerate}

{\bf 例:}设$f(x)$在$x_0$的某邻域内有定义,且$x_0$为其间断点,则下列函数
必以$x_0$为间断点的是(B)

\quad
(A)\;$f(x)\sin x$\hspace{2cm}
(B)\;$f(x)+\sin x$\hspace{2cm}
(C)\;$f^2(x)$\hspace{2cm}
(D)\;$|f(x)|$ 

{\bf 例:}设$x\to 0$时,$e^{\tan x}-e^x$与$x^n$为同阶无穷小,则$n$=(C)
\ps{Taylor展开}

\quad
(A)\;$1$\hspace{2cm}
(B)\;$2$\hspace{2cm}
(C)\;$3$\hspace{2cm}
(D)\;$4$ 

{\bf 例:}设$x\to 0$时,$x-\sin ax$与$x^2\ln(1-bx)$为同阶无穷小,则(A)
\ps{Taylor展开}

\quad
(A)\;$a=1,b=-\df16$\hspace{1em}
(B)\;$a=1,b=\df16$\hspace{1em}
(C)\;$a=-1,b=-\df16$\hspace{1em}
(D)\;$a=-1,b=\df16$

{\bf 例:}若$f(x)=\df{\sqrt[3]{x}}{\lambda-e^{-kx}}$在$(-\infty,+\infty)$
上连续,且$\limx{+\infty}f(x)=0$,则(D)

\quad
(A)\;$\lambda<0,k<0$\hspace{1cm}
(B)\;$\lambda<0,k>0$\hspace{1cm}
(C)\;$\lambda\geq0,k<0$\hspace{1cm}
(D)\;$\lambda\leq0,k>0$

% \begin{tabbing}
% 	\hspace{3cm}\=\hspace{3cm}\=\hspace{3cm}\=\kill
% 	\quad\quad\quad
% 	(A)\;$\lambda<0,k<0$\>  
% 	\quad\quad\quad
% 	(B)\;$\lambda<0,k>0$\>
% 	\quad\quad\quad  
% 	(C)\;$\lambda\geq0,k<0$\>
% 	\quad\quad\quad 
% 	(D)\;$\lambda\leq0,k>0$
% \end{tabbing} 

{\bf 例:}设$f(x)$在$(a,b)$内均有定义且单调有界,则$f(x)$在$(a,b)$内的间断点类型只能是(C)
\begin{tabbing}
	\hspace{8cm}\=\kill
	\quad\quad\quad
	(A)\;可去间断点 \> 
	(B)\;第二类间断点 \\ 
	\quad\quad\quad
	(C)\;跳跃间断点\>
	(D)\;不能确定
\end{tabbing}

{\bf 例:}若$x\to 0$时,$\ln\left(\cos\df{2x}3\right)\sim Ax^k$,则
$A=$\underline{$-\df29$},$k=$\underline{$2$}

{\bf 例:}计算以下极限
\begin{enumerate}[(1)]
  \setlength{\itemindent}{1cm}
  \item $\limx0\df{\sqrt[n]{1+\sin x}-1}{\tan x}$
  \item $\limx0(x+e^x)^{\frac1x}$
  \item $\limx0\left(\df{e^x+xe^x}{e^x-1}-\df1x\right)$
  \item $\limn\ln\left(\df{n-2na+1}{n(1-2a)}\right)^n\quad(a\ne1/2)$
  \item $\limx0\df{[\sin-\sin(\sin x)]\sin x}{x^4}$
  \item $\limx0\df{e^{\tan x}-e^{\sin x}}{x\sin^2x}$
  \item $\limx0\df{\sqrt{1+\tan x}-\sqrt{1+\sin x}}{x(1-\cos x)}$
  \item $\limn\left(n\tan\df1n\right)^{n^2}$
  \item $\limx0\df{\sin x+x^2\sin\df1x}{(1+\cos x)\ln(1+x)}$
  \item $\limx0\left[\df
  ax-\left(\df1{x^2}-a^2\right)\ln(1+ax)\right]$\quad($a\ne 0$)
  \item $\limx{\infty}\df{(x+a)^{x+a}(x+b)^{x+b}}{(x+a+b)^{2x+a+b}}$
  \item $\limx{0^+}\df{x^x-(\sin x)^x}{x^2\ln(1+x)}$
  \item $\limx0\df{\cos x-e^{-\frac{x^2}2}}{x^2[x+\ln(1-x)]}$
  \item $\limx{+\infty}\left(x+e^x\right)^{\frac1x}$
  \item $\limx{+\infty}\left(x+\sqrt{1+x^2}\right)^{\frac1x}$
  \item $\limx0\df{\sin x-x\cos x}{\sin^3x}$
  \item $\limx{+\infty}\df{e^x}{\left(1+\df1x\right)^{x^2}}$
  \item $\limx0\df1{x^3}\left[\left(\df{2+\cos x}3\right)^x-1\right]$
  \item $\limx0\df{\ln(\sin^2x+e^x)-x}{\ln(x^2+e^{2x})-2x}$
  \item $\limx1\df{x-x^x}{1-x+\ln x}$
  \item $\limx0\df{1-\cos x\sqrt{\cos2x}}{x^2}$
  \item $\limx{0^+}\df{\ln(1+e^{\frac2x})}{\ln(1+e^{\frac1x})}$
  \item $\limx{+\infty}\df{e^x-e^{-x}}{e^x+e^{-x}}$
  \item $\limx0\left(\df1n\sum\limits_{k=1}^na_k^x\right)^{\frac1x}
  \quad (a_k>0,k=1,2,\ldots,n)$
  \item $\limn n^2\left(\arctan\df an-\arctan\df a{n+1}\right)\quad (a>0)$
  \item $\limx0\df{(1+x)^{\frac1x}-(1+2x)^{\frac1{2x}}}{\sin x}$
  \item $\limn\left[\left(n^3-n^2+\df n2\right)e^{\frac1n}-\sqrt{1+n^6}\right]$
\end{enumerate}

{\bf 例:}设$\limx0\df{\ln\left(1+\df{f(x)}{\sin x}\right)}{a^x-1}=A\;(a\ne 1)$,求
$\limx0\df{f(x)}{x^2}$

{\bf 例:}设$\limn\df{n^a}{n^b-(n-1)^b}=10$,则$a=\underline{-9/10},
b=\underline{1/10}$

[提示]:
$\limn\df{n^a}{n^b-(n-1)^b}=10=\limn\df{n^{a-b}}{1-\left(1-\df1n\right)^b}
=\limn\df{n^{a-b+1}}b,$
由此可知,显然$b\ne 0$,且只有当$a-b+1=0$时,极限为为零有限值。

{\bf 例:}设$x\in(0,1]$时,$f(x)=x^{\sin x}$,且对任意$x$
$$f(x)+k=2f(x+1),$$
求常数$k$的值,使得极限$\limx0f(x)$存在。

[提示]:易知$\limx{0^+}f(x)=1$,有$x\in(-1,0]$时,
$$f(x)=2(x+1)^{\sin(x+1)}-k,$$
故$\limx{0^-}f(x)=2-k$,从而可得$k=1$.

{\bf 例:}设$f(x)$在$[a,b]$上连续,$\{x_n\}$为$[a,b]$上任一数列,求
$\limn\sqrt[n]{\df1n\sum\limits_{k=1}^ne^{f(x_k)}}$

[提示]:$e^{f(x)}$在$[a,b]$上连续且非负,故可设$m,M$分别为其在$[a,b]$上的最大和最小值,
从而由夹逼定理
$$\sqrt[n]m\leq\sqrt[n]{\df1n\sum\limits_{k=1}^ne^{f(x_k)}}\leq\sqrt[n]M,$$
由此易知原式$=1$。
