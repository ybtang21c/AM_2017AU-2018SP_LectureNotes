\setcounter{chapter}{2}

\chapter{微分中值定理与导数的应用}

微分学最初是独立于积分学的,因为其自身已经可以解决许多的应用问题。
本章讨论的极值、最值、单调性和曲率等,都是微分学的独立应用,其中
中值定理和Taylor公式又可以说是整个微分学的“制高点”,集成和融汇
了导数与微分的绝大部分思想与概念,正确地理解和熟练掌握这部分的知识
是对微积分有一个全面深刻认识的基础。

\section{微分中值定理}

中值定理在微分学中是一个非常富于技巧性和魅力的领域,我接下来我们所介绍的
从Rolle定理、Lagrange中值定理、Cauchy中值定理到Taylor公式的演进
路径,真正地将微分学推向了发展和应用的顶峰。

\subsection{Rolle定理}

\begin{thx}
	{\bf Fermat引理}:若函数$f(x)$在$x_0$处可导,且在$x_0$的某邻域内,有
	$$f(x)\geq f(x_0)\quad (\mbox{或}\quad f(x)\leq f(x_0))$$
	则$f'(x_0)=0$。
\end{thx}

[证]:设当$x\in U_0(x_0,\delta)$时,总有$f(x)\geq f(x_0)$,则
当$x>x_0$时,总有
$$\df{f(x)-f(x_0)}{x-x_0}\geq 0,$$
而当$x<x_0$时,总有
$$\df{f(x)-f(x_0)}{x-x_0}\leq 0,$$
由极限的保号性,分别令$x\to x_0^+$和$x\to x_0^-$,则有
$$f'(x_0)\geq 0,\quad\mbox{且}\quad f'(x_0)\leq 0,$$
由此即知$f'(x_0)=0$。\hfill$\Box$

\begin{shaded}
	{\bf 关于Fermat和Fermat's Last Theorem}
	
	Pierre de Fermat(1601-1665),法国律师和业余数学家。
	他在数学上的成就不比职业数学家差,他似乎对数论最有兴趣,亦对现代微积分的建立有所贡献。
	\begin{itemize}
	  \setlength{\itemindent}{1cm}
	  \item 微积分:Fermat引理给出了求极值的必要条件;
	  \item 将Apollonius of Perga(公元前262-190)的几何分析中用代数
	  方法来重新建立,开辟了{\it 解析几何}之路。
	  (Fermat只使用一轴,只接受正数的答案。后世多以Descartes为解析几何的创立者,
	  主因是Fermat没有发表其作品。)
	  \item 1654年,和Pascal在书信中的讨论,被认为是{\it 概率论}的开端,
	  1656年和概率论的正式创立者Christiaan Huygens(1629-1695)的交流,
	  使后者增加了对概率论的兴趣
	\end{itemize}
	
	\begin{tcolorbox}
		Fermat{\it 小定理}:假设$a\in\mathbb{Z}$,$p$为素数,则$a^p-a$可以整除$p$,
		或者写为
		$$a^p\equiv a(\mathrm{mod}p),$$
		特别低,若$a$不是$p$的倍数,则上式可写为$a^{p-1}\equiv 1(\mathrm{mod} p)$
	\end{tcolorbox}
	
	Fermat小定理是RSA公钥加密体制的理论基础,如果没有公钥加密体制,信息技术将
	不可能呈现今天的面貌。下面的是Fermat最重要的一个猜想,也是最后被证明的一个猜想。
	
	\begin{tcolorbox}
		Fermat{\it 大定理}(也称Fermat最后定理):对于大于$2$的正整数$n$,以下方程
		无正整数解
		$$x^n+y^n=z^n.$$
	\end{tcolorbox}
	
	这个定理造成了数学史上最大的悬案:1637年,费马在阅读丢番图《算术》拉丁文译本时,
	曾在第11卷第8命题旁写道:
	{\it 将一个立方数分成两个立方数之和,或一个四次幂分成两个四次幂之和,
	或者一般地将一个高于二次的幂分成两个同次幂之和,这是不可能的。
	关于此,我确信已发现了一种美妙的证法,可惜这里空白的地方太小,写不下。}
	
	一直被称为“Fermat猜想”,直到英国数学家Andrew John Wiles及其学生Richard
	Taylor于1995年将他们的证明出版后,才称为“Fermat大定理”。
	经过数学家们三个多世纪的努力,猜想才变成了定理。在冲击这个数论世纪难题的过程中,
	无论是不完全的还是最后完整的证明,都给数学界带来很大的影响;很多的数学结果、
	甚至数学分支在这个过程中诞生了,包括代数几何中的{\it 椭圆曲线}和{\it 模形式},
	以及{\it	Galois理论}和{\it Hecke代数}等。这也令人怀疑
	当初费马是否真的找到了正确证明。
	
	\begin{itemize}
	  \setlength{\itemindent}{1cm}
	  \item 1770年,Euler证明 $n=3$时定理成立
	  \item 1823年,Legendre证明$n=5$时定理成立
      \item 1832年,Dirichlet试图证明$n=7$失败,但证明$n=14$时定理成立
	  \item 1839年,Gabriel Lamé证明$n=7$时定理成立
	  \item 1850年,Ernst Eduard Kummer证明$2<n<100$时除$37$、$59$、$67$三数外定理成立
	  \item 1955年,Harry Vandiver以电脑计算证明了对所有不超过$2521$的素数定理成立
	  \item 1976年,Samuel Wagstaff以电脑计算证明对所有不超过$125000$的素数定理成立
	  \item 1985年,电脑计算证明了对所有小于$4$百万的素数定理成立
	  \item 1987年,格朗维尔以电脑计算证明了$2<n<10^{{1800000}}$时定理成立
	  \item 1995年,Wiles证明$n>2$时定理成立。
	\end{itemize}
	
	Wiles证明Fermat大定理的过程亦甚具戏剧性。他用了七年时间,在不为人知的情况下,
	得出了证明的大部分;然后于1993年6月在一个学术会议上宣布了他的证明,并瞬即
	成为世界头条。但在审批证明的过程中,专家发现了一个极严重的错误。Wiles和他的学生Taylor
	然后用了近一年时间尝试补救,终在1994年9月以一个之前Wiles抛弃过的方法得到成功。
	
	为什么Fermat大定理在数学史上的地位如此重要?有三个主要的原因:一是问题基本,长时间
	(350年)悬而未决,吸引了众多数学大师对其加以研究;二是研究问题的过程中产生了新的思想和方法,
	带动了数学很多领域的发展;三是涉及Taniyama-Shimura theorem({\it 谷山—志村猜想}),
	是现代数学大热门Langlands program({\it 朗兰兹纲领})的重要组成部分。1967年提出的
	Langlands program指出三个相对独立发展起来的数学分支:数论、代数几何和群表示论,
	实际上是密切相关的,而连接这些数学分支的纽带是一些特别的函数,被称为L-函数。
	L-函数可以说是Langlands program的中心研究对象。数学界著名的七个
	{\it “千禧年大奖问题”}(Millennium Prize Problems)中有两个就是关于L-函数的,
	它们分别是{\it Riemann假设}和{\it BSD猜想}。
	
	延伸阅读:
	\begin{enumerate}[1.]
	  \setlength{\itemindent}{1cm}
	  \item 知乎:为什么费马大定理在数学史上的地位如此重要?
	  \item 知乎:为什么费马大定理表述起来这么简单,证明却这么复杂?
	  \item https://en.wikipedia.org/wiki/Fermat's\_Last\_Theorem
	  \item https://zh.wikipedia.org/wiki/费马大定理
	  \item https://en.wikipedia.org/wiki/Andrew\_Wiles
	  \item https://en.wikipedia.org/wiki/Pierre\_de\_Fermat
	  \item 推翻费马大定理!https://st.im/wGX
	  \item 又一千禧年大奖难题被攻破,下一个向P/NP进发吧!https://st.im/wGL
	\end{enumerate}
\end{shaded}

\begin{thx}
	{\bf Rolle定理:}若函数$f(x)$满足:
	\begin{enumerate}[(1)]
	  \setlength{\itemindent}{1cm}
	  \item 在区间$[a,b]$上连续;
	  \item 在区间$(a,b)$内可导;
	  \item $f(a)=f(b)$。
	\end{enumerate}
	则:存在$\xi\in(a,b)$,使得$f\,'(\xi)=0$。
\end{thx}

[证]:$f(x)$在$[a,b]$上连续,故必存在最大和最小值,由于$f(a)=f(b)$,
故除非$f(x)\equiv f(a)$(此时定理显然成立),最大和最小值点不可能
同时为$a$和$b$,故必存在某个$\xi\in(a,b)$为$f(x)$的一个最值点。
因为$f(x)$在$(a,b)$内可导,故由Fermat引理,必有$f'(\xi)=0$,即证。\hfill$\Box$

{\bf 注:}{\b 三个条件缺一不可!!}\ps{为什么?每种情况给出相应的反例!}

\begin{center}
	\includegraphics[width=0.8\textwidth]{./images/ch3/antiRolle.jpg}
\end{center}

{\bf 例:}证明:对函数$f(x)=(x-1)(x-2)(x-3)$,至少存在一点
$\xi\in(1,3)$,使得$f\,''(\xi)=0$。

[证]:$f(x)$是任意阶可导的,且在$[1,2],[2,3]$上均满足Rolle定理条件,
故必存在$\xi_1\in(0,1),\xi_2\in(2,3)$,使得
$$f'(\xi_1)=f'(\xi_2)=0.$$
进一步地,容易验证$f'(x)$在$[\xi_1,\xi_2]$上满足Rolle定理条件,
故必存在$\xi\in(\xi_1,\xi_2)$,使得$f''(\xi)=0$。
\hfill$\Box$

参照这个例题的证明思路,可以很容易地得到如下的推论:

\begin{thx}
	{\bf Rolle定理的高阶推广}:
	设$f(x)$在$[x_0,x_n]$上有$n-1$阶连续导函数,在$(x_0,x_n)$内
	$n$阶可导,且
	$$f(x_0)=f(x_1)=\ldots=f(x_n),\quad(x_0<x_1<\ldots<x_n),$$
	则存在$\xi\in(x_0,x_n)$,使得$f^{(n)}(\xi)=0$。
\end{thx}

利用Rolle定理证明可以证明一些有趣的结论,也可以构造出很多看似花样百出
的{\it 存在性问题}:

[提示]:利用极限的保号性,说明在$a,b$附近各有一点,两点函数值反号,然后利用
介值定理和Rolle定理。

{\bf 例:}设$\df{a_0}{n+1}+\df{a_1}{n}+\ldots+a_n=0$,证明:方程
$$a_0x^n+a_1x^{n-1}+\ldots+a_n=0$$
在区间$(0,1)$内至少有一个根。

[提示]:令
$$F(x)=\df{a_0}{n+1}x^{n+1}+\df{a_1}{n}x^n+\ldots+a_nx$$

{\bf 例:}设$f(x)\in C[a,b]$,在$(a,b)$内可导,且$f(a)=f(b)=0$,证明:存在
$\xi\in(a,b)$,使得:
$$f'(\xi)+f(\xi)=0.$$

[提示]:$F(x)=f(x)e^x$

\begin{thx}
	辅助函数的构造可以从{\it 不定积分}(求导的逆运算)中获得一些启发
	\begin{itemize}
	  \setlength{\itemindent}{1cm}
	  \item $f(\xi)+\lambda f'(\xi)=0$:令$F(x)=e^{\lambda x}f(x)$
	  \item $f'(\xi)+f(\xi)g'(\xi)=0$: 令$F(x)=e^{\lambda g(x)}f(x)$
	  \item $nf(\xi)+xf'(\xi)=0$: 令$F(x)=x^nf(x)$ 
	  \item $f'(\xi)g(\xi)+f(\xi)g'(\xi)=0$:令$F(x)=f(x)g(x)$
	  \item $f'(x)g(x)-f(x)g'(x)=0$: 令$F(x)=\df{f(x)}{g(x)}$
	\end{itemize}
\end{thx}

{\bf 例}({\kaishu Darboux定理})设$f(x)$在$[a,b]$上可导,
且$f\,'_+(a)f\,'_-(b)<0$,则存在$\xi\in(a,b)$,使得$f\,'(\xi)=0$。

\begin{center}
	\includegraphics[width=0.8\textwidth]{./images/ch3/Darboux.jpg}
	
	\it Darboux定理的证明思路
\end{center}

[证]:不妨设$f'_+(a)>0$,且$f(a)>f(b)$(如左图)。

由$f'_+(a)>0$,也即$\limx{a^+}\df{f(x)-f(a)}{x-a}>0$,
由极限的保号性,必存在某个$a'>a$,使得$\df{f(a')-f(a)}{a'-a}>0$,
进而$f(a')>f(a)$。

显然$f(x)$在$[a',b]$上连续,$f(a)\in(f(b),f(a'))$,故由介值定理,
必存在$c\in(a',b)$,使得$f(c)=f(a)$。
注意到$f(x)$在$[c,b]$上满足Rolle定理条件,故必存在$\xi\in(c,b)\subset(a,b)$,
使得$f'(\xi)=0$。\hfill$\Box$

Darboux定理说明导函数具有{\it 介值性},即可以证明:介于区间端点处的导数值之间的所有
导数值,都可以在区间内部的某点上取到\ps{请自行尝试证明}。
介值性和连续性的关系一度是微积分理论中的难题,人们曾经猜想介值性和连续性是等价的,
直到发现不连续的导函数(例如:$f(x)=x^2\sin\frac1x$的导函数在$x=0$不连续),
并且通过Darboux定理验证了导函数的介值性,才发现这个猜想是错误的。

{\bf 思考题:}设$f(x)$在$[a,b]$上满足Rolle定理条件,
且$f\,'_+(a)f\,'_-(b)>0$,则$f\,'(x)=0$在$(a,b)$内至少有两个根。

\begin{center}
	\includegraphics[width=0.5\textwidth]{./images/ch3/Darboux2.jpg}
	
	\it 示意图
\end{center}
[提示]:先利用Rolle定理证明存在$\xi_1$,再根据$f(\xi_1)$与$f(a)$的大小关系,
采用与证明Darboux定理类似的方法证明$\xi_2$的存在。

\subsection{Lagrange中值定理}

下面这个定理据说实际上是由Cauchy证明的:

\begin{thx}
	{\bf Lagrange中值定理}:若函数$f(x)$满足:
	\begin{enumerate}[(1)]
	  \setlength{\itemindent}{1cm}
	  \item 在区间$[a,b]$上连续 
	  \item 在区间$(a,b)$内可导 
	\end{enumerate}
	则:存在$\xi\in(a,b)$,使得 
	$$f\,'(\xi)=\df{f(b)-f(a)}{b-a}.$$
\end{thx}

[提示]:结论即为\ps{\b 从结论出发构造辅助函数是证明中值问题的常用思路}
$$f'(\xi)(b-a)-[f(b)-f(a)]=0,$$
考虑
$$F'(x)=f'(x)(b-a)-[f(b)-f(a)],$$
则
$$F(x)=f(x)(b-a)-[f(b)-f(a)]x+C,$$
不妨令$C=0$,可以验证$F(b)=F(a)=bf(a)-af(b)$,再利用Rolle定理即可。

关于该证明的构造思路,有着非常生动形象的几何解释,具体请参见教材。从图示上看,
新构造的$F(x)$相当于是将原来的$f(x)$两端“拉平”(按比例平移)后得到的,
这种变换的一个重要特点是变换前后两个函数的连续性和可导性完全一致。

然而,从我们的分析过程来看,求解此类题目,完全不需要任何几何的直观,
可以直接从结论出发,通过求导的逆向过程(也就是{\it 不定积分})来构造出
所需的辅助函数。

Lagrange中值定理的结论有时也写作:
\begin{thx}
	存在$(\theta\in(0,1))$,使得
	$$f(x+\Delta x)=f(x)+f'(x+\theta\Delta x)\Delta x.$$
\end{thx}

{\bf 例:}证明:
\begin{enumerate}[(1)]
  \setlength{\itemindent}{1cm}
  \item 导数恒为零的函数取值恒为常数。
  \item 导数恒大于零的函数严格单调递增。
\end{enumerate}

{\bf 例:}证明方程$x^5+x-1=0$只有唯一实根。

{\bf 例:}证明:$\df{\ln x-\ln x_1}{x-x_1}<\df 1{x_1},\quad (0<x_1<x)$

{\bf 例:}证明不等式:$\df{x}{1+x}<\ln(1+x)<x,\;x>0$

% {\bf 注:}在证明Eular常数
% $$\gamma=\limn\left(1+\df12+\ldots+\df1n-\ln n\right)$$
% 时,用到了这个不等式。

{\bf 例:}设$f(x)\in C[a,b]$,且$f(x)$在$(a,b)$内二阶可导,
连接函数曲线两端点的直线在$(a,b)$内至少与曲线存在一个交点,
则存在$\xi\in(a,b)$,使得$f\,''(\xi)=0$。

{\bf 思考:}{\b 若函数曲线的端点连线与函数曲线存在多个交点,能够得到什么结论?}

{\bf 例:}设$f(x)$在$[a,b]$上二阶可导,$f(a)=f(b)=0$,证明:对任意
$c\in(a,b)$,存在$\xi\in(a,b)$,使得
$$f(c)=\df{f\,''(\xi)}{2}(c-a)(c-b)$$

[提示]:设
$$F''(x)=f''(x)-\df{2f(c)}{(c-a)(c-b)},$$
则可得
$$F(x)=f(x)-\df{f(c)}{(c-a)(c-b)}x^2+C_1x+C_2,$$
其中$C_1,C_2$为常数。不妨令$C_1=C_2=0$,以下可以验证:
$$\df{F(c)-F(a)}{c-a}=\df{F(b)-F(c)}{b-c}
=-\df{f(c)}{(c-a)(c-b)}(a+b),$$
由Lagrange中值定理,存在$\xi_1\in(a,c),\xi_2\in(c,b)$,使得
$$F'(\xi_1)=F'(\xi_2)=-\df{f(c)}{(c-a)(c-b)}(a+b),$$
进而再由Rolle定理,存在$\xi\in(\xi_1,\xi_2)$,使得$F''(\xi)=0$,即证。

[注]:直接令辅助函数
$$F(x)=f(x)-\df{(x-a)(x-b)}{(c-a)(c-b)}f(c)$$
亦可,这个辅助函数的右半部分事实上是过$(a,0),(b,0),(c,f(c))$
三点的{\kaishu Lagrange插值多项式}!

\begin{shaded}
	\begin{tcolorbox}
		{\bf 过$n$个点$(x_i,f(x_i))(i=1,2,\ldots,n)$的Lagrange插值多项式}
		$$L_n(x)=\sum\limits_{i=1}^n\prod_{j=1,j\ne i}^{n}
		\df{x-x_j}{x_i-x_j}f(x_i)$$
	\end{tcolorbox}
	
	插值可以理解成利用一条曲线在给定的点之间“穿插”,而插值多项式就是用于穿插
	(连接)所有给定点的一个多项式函数。在工程应用中,对离散的观测数据进行建模,
	得到反映变化(运动)规律的函数曲线,从而对未来的变化(运动)趋势进行预测,
	是很常见的问题。
	
	{\bf 例:}过点$(0,0),(1,2),(2,-1),(3,-2)$的Lagrange插值多项式为
	$$L(x)=\df{x^3}6-8x^2+\df{41}6x$$
	
	\begin{center}
		\resizebox{!}{5cm}{\includegraphics{./images/ch5/L4x.pdf}}
	\end{center}
	%% code of Mathematica 
	% f[x_] := 7/6*x^3 - 6 x^2 + 41/6*x;
	% CCurve = Plot[f[x], {x, -0.5, 3.5}];
	% DCurve = ListPlot[Table[{n, f[n]}, {n, 0, 3}], Filling -> Axis];
	% Show[CCurve, DCurve]
\end{shaded}

\subsection{Cauchy中值定理}

\begin{thx}
	{\bf Cauchy中值定理}:若函数$f(x),\varphi(x)$满足: 
	\begin{enumerate}[(1)]
	  \setlength{\itemindent}{1cm}
	  \item 在$[a,b]$上连续 
	  \item 在$(a,b)$内可导,且$\varphi'(x)\ne 0$ 
	\end{enumerate}
	则:存在$\xi\in(a,b)$, 使得
	$$\df{f(b)-f(a)}{\varphi(b)-\varphi(a)}=\df{f'(\xi)}{\varphi'(\xi)}$$
\end{thx}

[提示]:结论即为
$$[f(b)-f(a)]\varphi'(\xi)-[\varphi(b)-\varphi(a)]f'(\xi)=0,$$
因此考虑
$$F'(x)=[f(b)-f(a)]\varphi'(x)-[\varphi(b)-\varphi(a)]f'(x),$$
从而
$$F(x)=[f(b)-f(a)]\varphi(x)-[\varphi(b)-\varphi(a)]f(x)+C.$$
不妨令$C=0$,可以验证$F(b)=F(a)$,利用Rolle定理即可。

这个证明思路延续了我们此前证明很多中值问题所使用的方法,与教材上的方法存在
比较明显的差异,但从解题的角度来看,更具普遍的可操作性。

{\bf 注:}Cauchy中值定理可视为{\it 参数化}的Lagrange中值定理,即:
由$\varphi'(x)\ne 0$可知,$\varphi(x)$为单调(可逆)函数,从而设
$A=\varphi(a),B=\varphi(b)$,故由Lagrange中值定理,存在$\xi\in(a,b)$,
也即$C=\varphi(\xi)$介于$A,B$之间,使得
$$\df{f(b)-f(a)}{\varphi(b)-\varphi(a)}
=\df{f(\varphi^{-1}(B))-f(\varphi^{-1}(A))}{B-A}
=[f(\varphi^{-1}(x))]'_{x=\xi}=\df{f'(\xi)}{\varphi'(\xi)}.$$

{\bf 综合练习}:
{\bf 例:}设$0<a<b$,$f(x)$在$[a,b]$上连续,在$(a,b)$内可导,证明:
\begin{enumerate}[(1)]
  \setlength{\itemindent}{1cm}
  \item 存在$\xi\in(a,b)$,使得:
	$$f(b)-f(a)=\ln\df ba\cdot \xi f\,'(\xi)$$
  \item 存在$\eta\in(a,b)$,使得:
    $$2\eta[f(b)-f(a)]=(b^2-a^2)f\,'(\eta)$$
  \item 存在$x_1,x_2,x_3\in(a,b)$,使得
	$$f\,'(x_1)=(b+a)\df{f\,'(x_2)}{2x_2}=(a^2+ab+b^2)
	\df{f\,'(x_3)}{3x_3^2}$$ 
  \item 若$f\,'(x)\ne 0$,则存在$\xi,\eta\in(a,b)$,使得
	$$\df{f'(\xi)}{f'(\eta)}=\df{e^b-e^a}{b-a}e^{-\eta}$$
  \item 存在$c\in(a,b)$,使得
	$$\df{1}{a-b}\left|\begin{array}{cc}
	a & b\\ f(a) & f(b)
	\end{array}\right|=f(c)-cf'(c).$$
\end{enumerate}

通过对结论适当变形整理,以上结论都可以使用Rolle定理证明。

\section{L'Hospital法则}

在第一章的“无穷小比较”一节,我们曾讨论过一些不定式极限,这类极限的特点是
从其自身的构造特定,无法直接看出极限的结果,因此成为极限计算问题中一类
颇有难度的问题。和用无穷小代换的方法计算极限相比,L'Hospital法则在用法
上显得更为简单,应用的范围也非常广泛,因此可以被称为是“最受欢迎”的的
一类极限计算方法。

常见的不定式(型)极限包括如下一些类型:\ps{$0,1,\infty$在此均表示一种趋势,而不是具体的值}
$$\bm{\df{0}{0}}, \quad \bm{\df{\infty}{\infty}}, \quad
\bm{1^{\infty}}, \quad \bm{0\cdot\infty}, \quad
\bm{\infty-\infty}, \quad \bm{\infty^0}, \bm{\quad 0^0}$$
\begin{center}
	\resizebox{!}{3.5cm}{\includegraphics{./images/ch5/lim00.pdf}}
\end{center}
通过与各种函数运算相结合,它们之间可以实现互相的转化,因此和此前类似,在此
我们只讨论前两种情形的极限计算方法即可。

% {\bf 举例:}
% \begin{enumerate}[(1)]
%   \setlength{\itemindent}{1cm}
%   \item $\limx{0}\df{x-\sin x}{x}$ 
%   \item $\limx{+\infty}\df{\ln x}{x}$ 
%   \item $\limx{1}(1-x^2)\tan\df{\pi}2x$ 
%   \item $\limx{0}(x^2+2^x)^{1/x}$
% \end{enumerate}

L'Hospital是一个“著名”的数学爱好者,他为John Bernoulli提供赞助,并获得
对方一些最新的研究结果。以下的求极限法则最初由后者发现,而由前者发表并命名,
Bernoulli后来为此感到非常后悔,曾想要回这个本属于自己的定理。

\begin{thx}
	{\bf 计算$\df00$型不定式极限的L'Hospital法则(I):}设
	\begin{enumerate}[(1)]
	  \item $\limx{a}f(x)=\limx{a}F(x)=0$;
	  \item $f(x)$和$F(x)$在$a$的某去心邻域内可导,且$F'(x)\ne0$;
	  \item $\limx{a}\df{f'(x)}{F'(x)}$存在,或为无穷大,
	\end{enumerate}
	则
	$$\limx{a}\df{f(x)}{F(x)}=\limx{a}\df{f'(x)}{F'(x)}.$$
\end{thx}

[证]:由条件(1),不妨设$f(a)=F(a)=0$,任取$x$为条件(2)所述$a$的去心邻域内一点,
则$f(x)$和$F(x)$在$a$到$x$的区间内满足Cauchy中值定理条件,从而
存在$\xi$介于$a$和$x$之间,使得
$$\df{f(x)}{F(x)}=\df{f(x)-f(a)}{F(x)-F(a)}
=\df{f'(\xi)}{F'(\xi)},$$
在该式两端{\b 令$x\to a$,由$\xi$的特性可知,必有$\xi\to a$},进而由
条件(3)即证。\hfill$\Box$

以上结论可以直接推广到$\infty/\infty$不定式的情形,事实上:
若条件(1)改为$\limx{a}f(x)=\limx{a}F(x)=+\infty$,则
$\limx{a}\df1{f(x)}=\limx{a}\df1{F(x)}=0$;由条件(2)
可得$\df1{f(x)}$和$\df1{F(x)}$均可导,且$[1/F(x)]'\ne 0$;
条件(3)不变,对$\df1{f(x)}$和$\df1{F(x)}$直接使用以上定理的结论,可得
$$
	\limx{a}\df{f(x)}{F(x)}
	=\limx{a}\df{1/F(x)}{1/f(x)}
	=\limx{a}\df{[1/F(x)]'}{[1/f(x)]'}
	=\limx{a}\df{F'(x)}{f'(x)}\limx{a}\df{f^2(x)}{F^2(x)},
$$
稍加整理,即得结论。关于这个结论有一个形象的解释:两个物体都飞向无穷远处,
他们所经历的路程之比最终取决于他们的速度之比。

在使用L'Hospital法则时,特别需要注意的是
{\b 由$\limx{\Delta}\df{f'(x)}{g'(x)}=A$可以推出
$\limx{\Delta}\df{f(x)}{g(x)}=A$,但反之不然。}反例如下:
$\limx{+\infty}\df{x+\sin x}x=1$,
但$\limx{+\infty}(1+\cos x)$不存在!

最后,对于$x$趋于无穷时的极限,也有类似的结论,证明在此略过。

\begin{thx}
	{\bf 计算$\df00$型不定式极限的L'Hospital法则(II):}设
	\begin{enumerate}[(1)]
	  \item $\limx{\infty}f(x)=\limx{\infty}F(x)=0$;
	  \item $f(x)$和$F(x)$在当$x$充分大时可导,且$F'(x)\ne0$;
	  \item $\limx{\infty}\df{f'(x)}{F'(x)}$存在,或为无穷大,
	\end{enumerate}
	则
	$$\limx{\infty}\df{f(x)}{F(x)}=\limx{\infty}\df{f'(x)}{F'(x)}.$$
\end{thx}

{\bf 例:}计算如下极限
\begin{enumerate}[(1)]
  \setlength{\itemindent}{1cm}
  \item $\limx{0}\df{x-\sin x}{x^3}$ 
  \item $\limx{0}\df{e^x-e^{-x}-2x}{\tan^3x}$ 
  \item $\limx{0}\df{(1+x)^{1/x}-e}{x}$ 
  \item $\limx{+\infty}\df{x^n}{e^{\lambda
  x}}(n\in\mathbb{N},\lambda>0)$ 
  \item $\limx{+\infty}\df{\ln x}{x^\alpha}(\alpha>0)$ 
  \item $\limx{\infty}\df{x+\sin x}{x+\cos x}$
  \item $\limx{1}(1-x^2)\tan\df{\pi}{2}x$ 
  \item $\limx{0}\left(\df 1{x^2}-\df 1{x\tan x}\right)$ 
  \item $\limx{0^+}x^x$
  \item $\limx{0}(x^2+2^x)^{1/x}$ 
  \item $\limx{0}\left(\cot x-\df 1x\right)$ 
  \item $\limn\sqrt[n]n$
\end{enumerate}

{\b{\bf 讨论:}设$f(x)$在$x_0$附近存在二阶连续导函数,则
$$\lim\limits_{h\to 0}\df{f(x_0+h)+f(x_0-h)-2f(x_0)}{h^2}=f\,''(x_0).$$
若条件减弱为“$f(x)$在$x=x_0$处二阶可导”,结论是否仍然成立?\ps{难点!}

[答]:成立,如果条件是$f(x)$在$x_0$附近存在二阶连续导函数,可以如下进行推导,
\begin{align}
	\lim\limits_{h\to 0}&\df{f(x_0+h)+f(x_0-h)-2f(x_0)}{h^2}
	=\lim\limits_{h\to 0}\df{f'(x_0+h)-f'(x_0-h)}{2h}\notag\\
	&=\lim\limits_{h\to 0}\df{f''(x_0+h)+f''(x_0-h)}{2}
	=f''(0)\notag
\end{align}
但如果条件减弱为$f(x)$在$x=x_0$处二阶可导,这样的推导就是不对的。
其中的错误在于,此时不能保证$f(x)$在$x_0$以外的点(例如$x_0\pm h$)
处存在二阶导数,更谈不上二阶导数连续,因此最后的两个等号都是不成立的!

事实上,正确的推导应该是
\begin{align}
	\lim\limits_{h\to 0}&\df{f(x_0+h)+f(x_0-h)-2f(x_0)}{h^2}
	=\lim\limits_{h\to 0}\df{f'(x_0+h)-f'(x_0-h)}{2h}\notag\\
	&=\df12\lim\limits_{h\to 0}\df{f'(x_0+h)-f'(x_0)}{h}
	+\df12\lim\limits_{h\to 0}\df{f'(x_0-h)-f'(x_0)}{-h}
	=f''(0)\notag
\end{align}
其中最后一个等号只用到了二阶导数的定义。}

在L'Hospital法则的证明中,“令$x\to a$时,由于$\xi$介于$x$和$a$之间,进而
有$\xi\to a$”,这是一个非常有用的技巧,利用这一技巧,可以证明类似如下的结论:

{\bf 例:}设$\limx{+\infty}f(x)$和$\limx{+\infty}f\,'(x)$均存在,证明:
$$\limx{+\infty}f\,'(x)=0.$$

[证]:由Lagrange中值定理,对任意$x\in\mathbb{R}$,存在$\xi\in[x,x+1]$,满足
$$f(x+1)-f(x)=f'(\xi),$$
注意到{\b$x\to+\infty\Leftrightarrow\xi\to+\infty$},
$\limx{+\infty}f(x+1)=\limx{+\infty}f(x)$,故
$$0=\limx{+\infty}[f(x+1)-f(x)]=\limx{+\infty}f'(\xi)
=\lim\limits_{\xi\to+\infty}f'(\xi)=\limx{+\infty}f'(x)$$
即证。

{\bf 例:}设$f(x)$在$U(x_0)$内连续,在$U^0(x_0)$内可导,
且$\limx{x_0}f\,'(x)=l$,则$$f\,'(x_0)=l.$$

[证]:由Lagrange中值定理,对任意$\Delta x\in\mathbb{R}$,存在
$\xi\in[x_0,x_0+\Delta x]$,使得
$$\df{f(x_0+\Delta x)-f(x_0)}{\Delta x}=f'(\xi),$$
注意到{\b$\Delta x\to 0\Leftrightarrow\xi\to0$},故
$$f'(x_0)=\lim\limits_{\Delta x\to 0}\df{f(x_0+\Delta x)-f(x_0)}{x_0}
=\lim\limits_{\Delta x\to 0}f'(\xi)=\lim\limits_{\xi\to 0}f'(\xi)=l,$$
即证。

{\bf 例:}
设$f(0)=1$,且
$$\limx{0}\df{\ln(1-x)+f(x)\sin x}{e^{x^2}-1}=0,$$
证明:$f(x)$在$x=0$处可导,求$f\,'(0)$。

\section{Taylor公式}

Taylor公式直观地说是对给定函数的一种{\it 多项式逼近},也即用一个多项式函数
近似地表示某个给定的函数。相对于微分所体现的“以直代曲”,这种逼近方式可以称为
“以曲代曲”,不难想到,由于多项式函数的类型比线性函数(事实上就是$1$阶的多项式)
更加丰富,这样的逼近效果应该会更好。下面,我们首先从“以曲代曲”
的角度引入Taylor多项式的概念,然后分析其逼近的效果,最后给出Taylor公式
的有关应用。可以说,Taylor公式是整个微分学的“制高点”,其中蕴含了微分学中
的众多结论,或者说,微分学中的众多结论其实都只是它的推论而已。

“以曲代曲”的核心是所谓的{\it $n$阶相切}:

\begin{thx}
	$P_n(x)$称为{\bf $f(x)$在点$x_0$处的$n$阶Taylor多项式},若
	\begin{itemize}
% 	  \setlength{\itemindent}{1cm}
	  \item $P_n(x)$是$n$次多项式,形如:$\sum\limits_{k=0}^na_k(x-x_0)^k$;
	  \item $y=P_n(x)$在$x_0$处与$y=f(x)$处至少{\bf $n$阶相切},即
	  对任意$k=0,1,2,\ldots,n$,有
	  $$P^{(k)}_n(x_0)=f^{(k)}(x_0).$$
	%   \begin{enumerate}[(1)]
	%     \item $P_n(x_0)=f(x_0)$
	%     \item $P'_n(x_0)=f'(x_0)$
	%     \item $P''_n(x_0)=f''(x_0)$
	%     \item \ldots
	%     \item $P^{(n)}_n(x_0)=f^{(n)}(x_0)$
	%   \end{enumerate} 
	\end{itemize}
	特别地,若$x_0=0$,$P_n(x)$称为{\bf $f(x)$的$n$阶Maclaurin多项式}
\end{thx}

根据以上定义,可设
$$P_n(x)=a_0+a_1(x-x_0)+a_2(x-x_0)^2+\ldots+a_n(x-x_0)^n,$$
对其两边逐次求导(直至$n$阶导数),然后令$x=x_0$,不难得到
\begin{align*}
	P_n(x_0)&=a_0,\\
	P'_n(x_0)&=a_1,\\
	P''_n(x_0)&=2a_2,\\
	P^{(3)}_n(x_0)&=3!a_3,\\
	\ldots&\ldots\\
	P^{(n)}_n(x_0)&=n!a_n.\\
\end{align*}
利用以上定义中的第二个条件,解得
$$a_k=\df{f^{(n)}(x_0)}{k!},\;k=0,1,2,\ldots,n.$$
至此就得到了
\begin{thx}
	{\bf $f(x)$在点$x_0$处的$n$阶Taylor多项式:}
	$$P_n(x)=\sum\limits_{k=0}^n\df{f^{\,(k)}(x_0)}{k!}(x-x_0)^k$$
\end{thx}
这个公式说明,给定$f(x)$、$x_0$和$n$,$P_n(x)$是唯一确定的。更直观地说,
{\it\b 要在给定点处,按照指定的(相切)阶数,近似表示给定的函数,
只有唯一的一个多项式可以满足条件}。

{\bf 例:}求下列函数的Maclaurin多项式
\begin{enumerate}[(1)]
  \setlength{\itemindent}{1cm}
  \item $f(x)=e^x$ 
  $${P_n(x)=1+x+\df{x^2}{2!}+\ldots+\df{x^n}{n!}}$$ 
  \item $f(x)=\cos x$ 
  $${P_{2m}(x)=1-\df{x^2}{2!}+\df{x^4}{4!}-\ldots+(-1)^m\cdot\df{x^{2m}}{(2m)!}}$$
\end{enumerate}

\subsection{误差估计}

有了Taylor多项式,接下来需要确认它是否是一个好的逼近。为此,我们首先给出余项和
Taylor公式的概念。

\begin{thx}
	\begin{itemize}
% 	  \setlength{\itemindent}{1cm}
	  \item {\bf 余项:}
	  $$R_n(x)=|f(x)-P_n(x)|$$ 
	  \item {\bf 带Peano余项的$n$阶Taylor公式:}
	  $$f(x)=P_n(x)+\circ[(x-x_0)^n]$$
	\end{itemize}
\end{thx}

可以看到Taylor公式由Taylor多项式和余项两个部分构成,今后如果允许使用“{\it 无穷次}”
的多项式(也就是{\it 幂级数}),则可以进一步得到所谓的{\kaishu Taylor级数}。


下面的定理说明多项式逼近是一种{\it 有效}的逼近,其在给定的点附近与给定函数的误差,
是自变量改变量的高次幂(取决于展开的阶数)一个高阶无穷小。

\begin{thx}
	{\bf Taylor公式的误差:}设函数$f(x)$在$x_0$处$n$阶可导,$P(x)$是$f(x)$在点$x_0$处
	的$n$阶Taylor多项式,则当$x\to x_0$时
	$${|f(x)-P_n(x)|=\circ[(x-x_0)^n]}$$
\end{thx}

需要特点指出一点,Taylor公式的{\b 阶数是多少,关键看余项!}例如$\sin x$的7阶和8阶Maclaurin公式分别为
$$\sin x=x-\df{x^3}{3!}+\df{x^5}{5!}-\df{x^7}{7!}+{\b\circ(x^7)},$$
$$\sin x=x-\df{x^3}{3!}+\df{x^5}{5!}-\df{x^7}{7!}+{\b\circ(x^8)}.$$

高阶无穷小形式的余项从理论意义上确保了Taylor多项式的逼近效果,但对于需要具体估计
逼近误差的场合,还是不够的,因此就有了所谓的{\kaishu Lagrange余项},对应的
Taylor公式也称为Taylor中值定理:

\begin{thx}
	{\bf Taylor中值定理:}设$f(x)$在$x_0$的某邻域内$n+1$阶可导,
	对该邻域内任一点$x$,存在介于$x_0$和$x$之间的一点$\xi$,满足
	$$f(x)=P_n(x)+\df{f^{(n+1)}(\xi)}{\,(n+1)!}(x-x_0)^{n+1}$$
\end{thx}

上式称为{\bf 带Lagrange余项的$n$阶Taylor公式},有时也写为
$${\b f(x_0+h)=P_n(x_0+h)+\df{f^{\,(n+1)}(x_0+\theta
h)}{(n+1)!}h^{n+1},(0<\theta<1)}$$
不难看出,Taylor中值定理是对Lagrange中值定理的推广。

[证]:
先考虑$x>x_0$的情况,其余情况可以类似地证明。注意到$P^{(n+1)}(x)=0$,
$f^{(k)}(x_0)=P^{(k)}_n(x_0),k=0,1,2,\ldots,n$,
反复利用Cauchy中值定理,存在$x_0<\xi<\xi_n<\xi_{n-1}
<\ldots<\xi_2<\xi_1<x$,使得
\begin{align}
	\df{f(x)-P_n(x)}{(x-x_0)^{n+1}}
	&=\df{f'(\xi_1)-P'_n(\xi_2)}{(n+1)(\xi_1-x_0)^{n}}
	=\df{f''(\xi_2)-P''_n(\xi_2)}{(n+1)n(\xi_2-x_0)^{n-1}}\notag\\
	&=\ldots=\df{f^{(n)}(\xi_n)-P^{(n)}_n(\xi_n)}{(n+1)!(\xi_n-x_0)}
	=\df{f^{(n+1)}(\xi)}{(n+1)!}\notag
\end{align}
即证。\hfill$\Box$

利用以上定理,立即可以得到如下推论:
\begin{thx}
	{\bf Taylor公式的误差估计:}若存在常数$C>0$,使当$x\in(a,b)$时,恒有
	$$|f^{\,(n+1)}(x)|\leq C,\;n=0,1,2,\ldots$$
	则
	$$\limn [f(x)-P_n(x)]=0.$$
\end{thx}
该推论告诉我们,{\it 在一定条件下}(各阶导数有界),Taylor多项式的次数越高,逼近精度就越高。

\begin{shaded}
	{\bf Taylor公式的适用条件}
	
	以下内容主要引自《数学桥》P92。
	
	除了要求被展开的函数应该具有高阶导数之外,在运用Taylor公式的时候,有一个
	条件特别我们需要注意。如果所讨论的函数的导数值增加得太快的话,这个结论就
	不成立了,因为我们无法让给定点附近的各阶导数都被某个确定值所界定。
	例如,下面的函数
	$$f(x)=\left\{\begin{array}{ll}
		\exp\left(-\df1{x^2}\right),&x\ne 0;\\
		0,& x=0.
	\end{array}\right.$$
	\begin{center}
		\includegraphics[width=0.6\textwidth]{./images/ch3/e-1x2.pdf}
	\end{center}
	通过计算,不难发现,对任意$n\in\mathbb{Z}^+$,均有
	$$f^{(n)}(0)=0.$$
	因此,$f(x)$在$x=0$处的Taylor公式就是
	$$f(x)=\sum\limits_{k=0}^n\df{0}{k!}x^k+\circ(x^n)
	=0+\circ(x^n).$$
	(更进一步说,$f(x)$的Taylor级数应该表示为$f(x)=0$。)
	这显然和原函数存在较大“出入”,问题出在哪里呢?
	
	问题在于$f(x)$的$n$阶导函数都包含如下的一部分
	$$\df{2^n}{x^{2n}}\exp\df1{-x^2}.$$
	对于任意给定的实数$C$,只要$|x|<1$,当$n$充分大时,该值总会超过$C$。
	由于这个原因,我们无法把原点任何邻域内的所有导数值都限定在一定的范围内,
	这意味着前面关于Taylor公式误差估计的结论不再有效,换言之,此时的
	Taylor多项式已经不是给定函数的一个有效逼近了。
\end{shaded}

\begin{thx}
	{\bf 一些常用的Maclaurin公式}
	\begin{enumerate}[(1)]
% 	  \setlength{\itemindent}{1cm}
	  \item $e^x =\sum\limits_{k=0}^n\df{x^k}{k!}
	  +\df{e^{\theta x}}{(n+1)!}x^{n+1}\;
	  (0<\theta<1,x\in\mathbb{R})$
	  \item $\sin x
	  =\sum\limits_{k=1}^m(-1)^{k-1}\df{x^{2k-1}}{(2k-1)!} 
	  +(-1)^m\df{\cos\theta x}{(2m+1)!}x^{2m+1}\;
	  (0<\theta<1,x\in\mathbb{R})$
	  \item $\cos x= \sum\limits_{k=0}^m(-1)^{k}\df{x^{2k}}{(2k)!}
	  +(-1)^{m+1}\df{\sin\theta
	  x}{(2m+2)!}x^{2m+2}\; (0<\theta<1,x\in\mathbb{R})$
	  \item
	  $\ln(1+x) =\sum\limits_{k=1}^n(-1)^{k-1}\df{x^k}{k}
	  +\df{(-1)^nx^{n+1}}{(n+1)(1+\theta
	  x)^{n+1}}\; (0<\theta<1,x>-1)$
	  \item
	  $(1+x)^\alpha =\sum\limits_{k=0}^n
	  \df{\alpha(\alpha-1)\ldots(\alpha-k+1)}{k!}x^k$

		\hspace{2cm}$+\df{\alpha(\alpha-1)\ldots(\alpha-n)}{(n+1)!}\df{x^{n+1}}{(1+\theta
		x)^{n+1-\alpha}}\quad (0<\theta<1,x\ne -1)$
	\end{enumerate}
\end{thx}

\begin{shaded}
	{\bf Taylor公式:等价无穷小的“宝库”}
	
	从Taylor公式出发,可以很容易地得到一些等价无穷小,例如,由
	$$\sin x=x-\df{x^3}{3!}+\df{x^5}{5!}-\df{x^7}{7!}+
	\ldots+(-1)^n\df{x^{2n+1}}{(2n+1)!}+\circ(x^{2n+1}).$$
	两边同时除以$x$,再令$x\to 0$,可得
	\begin{align*}
		\limx{0}\df{\sin x}x&=1+\limx{0}\df1x\left[-\df{x^3}{3!}+
		\df{x^5}{5!}-\df{x^7}{7!}+ 
		\ldots+(-1)^n\df{x^{2n+1}}{(2n+1)!}+\circ(x^{2n+1})\right]\\
		&=1+\limx{0}\left[-\df{x^2}{3!}+
		\df{x^4}{5!}-\df{x^6}{7!}+ 
		\ldots+(-1)^n\df{x^{2n}}{(2n+1)!}+\circ(x^{2n})\right]=1,
	\end{align*}
	故$x\to 0$时,
	$$\sin x\sim x.$$
	
	类似地,注意到
	$$\sin x-x=-\df{x^3}{3!}+\df{x^5}{5!}-\df{x^7}{7!}+
	\ldots+(-1)^n\df{x^{2n+1}}{(2n+1)!}+\circ(x^{2n+1}).$$
	两边同时除以$x^3$,再令$x\to 0$,可得
	\begin{align*}
		\limx0\df{\sin x-x}{x^3}
		&=-\df1{3!}+\limx0\df1{x^3}\left[\df{x^5}{5!}-\df{x^7}{7!}+ 
		\ldots+(-1)^n\df{x^{2n+1}}{(2n+1)!}+\circ(x^{2n+1})\right]\\
		&=-\df16+\limx0\left[\df{x^2}{5!}-\df{x^4}{7!}+ 
		\ldots+(-1)^n\df{x^{2n-2}}{(2n+1)!}+\circ(x^{2n-2})\right]
		=-\df16.
	\end{align*}
	由此可知,当$x\to 0$时,
	$$\sin x-x\sim -\df16x^3.$$
	
	依此类推,从$\sin x$的Taylor公式中可以得到许许多多的等价无穷小关系(当$x\to 0$时):
% 	\begin{tcolorbox}[breakable]
		\begin{align*}
			\sin x&\sim  x\\
			\sin x-x&\sim  -\df16x^3\\
			\sin x-x+\df{x^3}{3!} & \sim  \df{x^5}{5!}\\
			\sin x-x+\df{x^3}{3!}+\df{x^5}{5!} & \sim  -\df{x^7}{7!}\\
			\ldots & \sim  \ldots\\
			\sin x-\sum\limits_{k=0}^n(-1)^n\df{x^{2n+1}}{(2n+1)!}
			&\sim  (-1)^{n+1}\df{x^{2n+3}}{(2n+3)!}
		\end{align*}
% 	\end{tcolorbox}
\end{shaded}

\subsection{函数的Taylor展开}

求给定函数的Taylor公式(或Taylor级数)的过程称为对该函数的{\kaishu Taylor展开}。
求一个函数的Taylor展开,除了直接求各阶导数然后套用公式的“{\it 直接法}”\ps{有时计算可能很复杂},
更多的很多时候我们可以尝试利用幂级数的一些解析性质,通过间接的方法进行。

% {\bf 问题:}{ 给定函数$f(x)$,求其在$x_0$的$n$阶Taylor公式} 
% \begin{enumerate}[(1)]
%   \setlength{\itemindent}{1cm}
%   \item {\bf 直接法(公式法)} 
%   \begin{itemize}
%     \item 逐个计算Taylor系数,给出相应的公式 
%   \end{itemize}
%   \item {\bf 间接法} \ps{能够使用间接法求展开的根本原因是Taylor展开具有唯一性!}
%   \begin{itemize}
%     \item 利用已知函数的Maclaurin公式 
%     \item 利用级数和多项式的性质
%   \end{itemize}
% \end{enumerate}

我们首先考虑最简单的一类初等函数,多项式函数的Taylor展开:

\begin{thx}
	{\bf 多项式函数的Taylor展开:}
	多项式函数$P_n(x)=\sum\limits_{k=0}^na_kx^k$的$m$阶Maclaurin多项式
	为其$m$次 {\it 截断多项式:} 
	$$P_m(x)=\sum\limits_{k=0}^ma_kx^k$$
\end{thx}

{\bf 例:}求$f(x)=x^3+3x^2-2x+4$的各阶Maclaurin多项式和在
$x=1$处的Taylor多项式。

[提示]:{\b$f(x)$自身已经是一个Maclaurin多项式,但若所求展开式是在某个$x_0$处,
则应该使用形如$t=x-x_0$的变化先求对应的Maclaurin公式,再代回即可。}

\begin{thx}
	{\bf 幂级数的解析性质}
	
	若在区间$I$内,$f(x)=\sum\limits_{n=0}^{\infty}a_nx^n$,
	$g(x)=\sum\limits_{n=0}^{\infty}b_nx^n$,则对任意$x\in I$,
	\begin{enumerate}[(1)]
% 	  \setlength{\itemindent}{1cm}
	  \item {\it 变量替换:}若$x\to 0$与$u\to 0$等价,则
	  $$f(u)=\sum\limits_{n=0}^{\infty}a_nu^n$$
	  \item {\it 线性性:}对任意$\lambda,\mu\in\mathbb{R}$,
	  $$\lambda f(x)+\mu g(x)=\sum\limits_{n=0}^{\infty}(\lambda a_n+\mu
	  b_n)x^n$$ 
	  \item {\it 逐项求导:}
	  $$f\,'(x)=\sum\limits_{n=0}^{\infty}na_{n}x^{n-1}$$ 
	  \item {\it 逐项积分:} 
	  $$\displaystyle\int f(x)dx=\sum\limits_{n=0}^{\infty}
	  \df{a_{n}}{n+1}x^{n+1}$$
	\end{enumerate}
\end{thx}

{\bf 例:}求$f(x)=\df 12\ln\df{1+x}{1-x}$的各阶带有Peano余项的Maclaurin公式。

[解]:当$|x|<1$时,
$$\ln(1+x)=\sum\limits_{k=1}^n(-1)^k\df{x^k}k+\circ(x^n)
=x-\df{x^2}2+\df{x^3}3+\ldots+(-1)^n\df{x^n}n+\circ(x^n).$$
注意到,$x\to0\Leftrightarrow-x\to 0$,故
\begin{align*}
	\ln(1-x)&=\sum\limits_{k=1}^n(-1)^k\df{(-x)^k}k+\circ((-x)^n)\\
	&=-x-\df{(-x)^2}2+\df{(-x)^3}3+\ldots+(-1)^n\df{(-x)^n}n+\circ((-x)^n)\\
	&=-x-\df{x^2}2-\df{x^3}3-\ldots-\df{-x^n}n+\circ(x^n).
\end{align*}
综上,
\begin{align*}
	f(x)&=\df12\ln(1+x)-\df12\ln(1-x)\\
	&=x+\df{x^3}3+\df{x^5}5+\ldots+\df{x^{2n+1}}{2n+1}+\circ(x^{2n+1})\\
	&=\sum\limits_{k=0}^n\df{x^{2k+1}}{2n+1}+\circ(x^{2n+1}).
\end{align*}
即为所求。\hfill$\Box$

{\bf 注:}利用变量代换进行Taylor展开时,经常会遇到高阶无穷小的变换与合并,例如本例中
$\circ((-x)^n)$可以直接改写成$\circ(x^n)$,而两个$\circ(x^n)$合并后仍写作
$\circ(x^n)$,其背后的原因其实很简单。前者是因为有界量乘以无穷小仍为无穷小,并且
不改变其阶数,后者是因为两个同阶无穷小合并(即使是相减),至少会得到一个仍然是该
阶数的无穷小(也可能是更高阶的)。

在使用非直接的方法进行Taylor展开时,一个常见的顾虑是不知道展开后的式子是否就是我们要求
的,其实检验的方法非常简单。不要忘记,对于给定的函数,在给定点处,按照指定阶数,其Taylor
公式具有唯一性。也就是说,{\b 如果有一个关于$(x-x_0)$的$n$阶多项式函数$P(x)$,与指定函数$f(x)$只相差
$(x-x_0)^n$的一个高阶无穷小,那么毫无疑问,这个$P(x)$就是$f(x)$在$x_0$处的$n$阶
Taylor多项式,相应地$f(x)$在$x_0$处的$n$阶Taylor公式就是$f(x)=P(x)+\circ((x-x_0)^n)$。
总结一下,我们可以说,只要展开后的公式具备Taylor公式的正确形式,那么结果就必然是正确的。}

{\bf 例:}求$f(x)=\df 1{2+x}$在$x=1$处的$7$阶Taylor多项式,并求$f^{(7)}(1)$。

[解]:
$$f(x)=\df13\cdot\df1{1+(x-1)/3}.$$
记$u=\df{x-1}3$,则$x-1\to0\Leftrightarrow u\to0$,又
$$\df1{1+x}=\sum\limits_{k=0}^n(-1)^kx^k+\circ(x^n).$$
故
\begin{align*}
	f(x)&=\df13\df1{1+u}
	=\df13\sum\limits_{k=0}^n(-1)^ku^k+\circ(u^n)\\
	&=\df13\sum\limits_{k=0}^n(-1)^k\left(\df{x-1}3\right)^k
	+\circ\left(\left(\df{x-1}3\right)^n\right)\\
	&=\sum\limits_{k=0}^n\df{(-1)^k}{3^{k+1}}x^k
	+\circ((x-1)^3).
\end{align*}
注意到$a_7=\df{(-1)^7}{3^8}=-\df1{3^8}$,故
$$f^{(7)}(1)=7!\cdot a_7=-\df{7!}{3^8}.$$
以上即为所求。\hfill$\Box$

{\bf 思考:}以上对$f(x)$的展开方式是唯一的吗?结果呢?
为什么不能直接将$f(x)$改写成$\df1{1+(x+1)}$展开?

[答]:Taylor展开在给定了$f(x)$、展开的位置和阶数后就是唯一确定的,至于展开的
过程可能存在不同的方式。如果按照$\df1{1+(x+1)}$展开,所得到的不是$x=1$
处而是$x=-1$处的Taylor公式,不符合题意的要求。

{\bf 例:}\ps{KD教材习题5.3-4}
求带有Peano余项的Maclaurin公式\ps{请特别注意高阶无穷小的简化}
\begin{enumerate}[(1)]
  \setlength{\itemindent}{1cm}
  \item $f(x)=\ln(2+x)=\ln2+\ln\left(1+\df x2\right)
  =\ln2+\sum\limits_{k=1}^n\df{(-1)^{k-1}}{k2^k}x^k+\circ(x^n)$ 
  \item $f(x)=e^{-x^2}=\sum\limits_{k=0}^n\df{(-1)^k}{k!}x^{2k}
  +\circ(x^{2n})$
  \item $f(x)=x\sin x=\sum\limits_{k=0}^{n}\df{(-1)k}{k!}x^{2k+2}
  +\circ(x^{2n+2})$ 
  \item $f(x)=\df{x^2}{1+x}=\sum\limits_{k=0}^n(-1)^kx^{k+2}
  +\circ(x^{n+2})$ 
  \item $f(x)=\df{1}{\sqrt{1-x^2}}=\sum\limits_{k=0}^n\left(
  \begin{array}{c}
  -\frac12\\ k
  \end{array}\right)(-1)^kx^{2k}+\circ(x^{2n})$ 
  \item $f(x)=\cos^2x=\df{1+\cos2x}2=\df12+
  \sum\limits_{k=0}^{n}\df{(-1)^k2^k}{(2k)!}x^{2k}+\circ(x^{2n})$
\end{enumerate}

{\bf 例:}设$y=\arctan x$,求$y^{(n)}(0)$。
\ps{介绍求高阶导数的Leibniz公式时曾做过此题}

[解]:注意到
$$y'=\df1{1+x^2}$$
而当$|x|<1$时
$$\df1{1+x}=\sum\limits_{k=0}^{\infty}(-1)^kx^k.$$
又$x\to0\Leftrightarrow x^2\to0$,故
$$\df1{1+x^2}=\sum\limits_{k=0}^{\infty}{(-1)^k}x^{2k},\quad (|x|<1),$$
上式两边积分\ps{这里涉及到收敛幂级数的有关性质,先使用结论即可},可得
\begin{align*}
	\arctan x&=\dint_0^x\df1{1+t^2}\d t
	=\dint_0^x\sum\limits_{k=0}^{\infty}{(-1)^k}t^{2k}\d t\\
	&=\sum\limits_{k=0}^{\infty}{(-1)^k}\dint_0^xt^{2k}\d t
	=\sum\limits_{k=0}^{\infty}\df{(-1)^k}{2k+1}x^{2k+1},
	\quad (|x|<1),
\end{align*}
从而可知
$$y^{(n)}(0)=\left\{\begin{array}{ll}
0,& n=2k,\\ {(-1)^k}(2k)!,& x=2k+1
\end{array}\right.$$
\hfill$\Box$

\subsection{Taylor公式的应用}

\subsubsection{近似计算}

{\bf 例:}计算$\sin 1$的值,误差不超过$10^{-5}$。

[解]:对任意$x\in\mathbb{R}$,存在$\theta\in(0,1)$,使得
$$\sin x=x-\df{x^3}{3!}+\df{x^5}{5!}-+\ldots
+(-1)^{n-1}\df{x^{2n-1}}{(2n-1)!}
+(-1)^{n}\df{\cos\theta x}{(2n+1)!}x^{2n+1}.$$
注意到
$$\left|(-1)^{n}\df{\cos\theta x}{(2n+1)!}x^{2n+1}\right|
\leq\df{|x|^{2n+1}}{(2n+1)!},$$
故要使$x=1$时的误差小于$10^{-5}$,只需
$$\df1{(2n+1)!}\leq 10^{-5},$$
计算不难发现
$$\df1{9!}=\df1{362880}<10^{-5}<\df1{5040}=\df1{7!},$$
故当$n=4$时,即可满足精度要求,此时
$$\sin 1\approx 1-\df1{3!}+\df1{5!}-\df1{7!}=\df{4241}{5040}\approx 0.841468.$$
\hfill$\Box$

{\bf 例:}计算$e$的值,误差不超过$10^{-5}$。

\begin{center}
	\resizebox{!}{5.5cm}{\includegraphics{./images/ch5/exse1.pdf}}\quad
	\resizebox{!}{5.5cm}{\includegraphics{./images/ch5/exse2.pdf}}
\end{center}

\subsubsection{计算不定式极限}

要证明Taylor公式的强大,用它来计算各种不定式极限无疑是最好的途径。某种意义上说,
没有用Taylor公式无法解决的极限问题,关键的问题是,值不值得动用这个“{\it 牛刀杀鸡}”。

{\bf 例:}计算以下极限
\begin{enumerate}[(1)]
  \setlength{\itemindent}{1cm}
  \item $\limx{0}\df{x-\sin x}{x^3}
  =\limx0\df{x-[x-\frac{x^3}3!+\circ(x^3)]}{x^3}=\df16$
  \item $\limx{0}\df{\cos x-e^{-x^2/2}}{x^4}
  =\limx0\df{1-\frac{x^2}2+\frac{x^4}{4!}-[1-\frac{x^2}2+
  \frac{x^4}8]+\circ(x^4)}{x^4}=-\df1{12}$ 
  \item $\limx{0}\df{\df{x^2}{2}+1-\sqrt{1+x^2}}{x^2\sin x^2}
  =\limx0\df{\frac{x^2}2+1-[1+\frac12{x^2}-\frac18x^4+\circ(x^4)]}{x^4}
  =\df18$ 
  \item $\limx{0}\df{e^x-1-x}{\df{1}{\sqrt{1+x}}-\cos\sqrt x}
  =\limx0\df{\frac{x^2}2+\circ(x^2)}{1-\frac12x+\frac38x^2-[1-\frac{x}2+\frac{x^2}{4!}]+\circ(x^2)}
  =\df32$ 
  \item $\limx{0}\df{\cos(\sin x)-\cos x}{x^4}
  =-2\limx0\df{\sin\frac{\sin x+x}2\sin\frac{\sin x-x}2}{x^4}=\df16$
  \item $\limx{+\infty}\left[x-x^2\ln\left(1+\df 1x\right)\right]
  \xlongequal{u=1/x}\lim\limits_{u\to0}\df{u-\ln(1+u)}{u^2}$
\end{enumerate}

从解题的过程来看,对于各种“$\df{\bm0}{\bm0}$”型的不定式极限,都可以先将分子分母
都展开成Taylor公式,然后通过简单的有理函数极限运算求得结果。因此,在此类极限问题上,
我们说Taylor是“{\it 万能}”的也并不为过。

然而,Taylor公式“{\it 天生}”比我们此前接触的计算工具更复杂
(这完全是因为它推广或者说蕴含了过去的大部分结论),因此在使用它时也需要加倍地谨慎。
在对分子分母进行Taylor展开时,一般都需要作一些“试探性”地尝试,
确定上下各展开到多少阶合适(太低解决不了问题,太高计算过于复杂),当然,
最重要的是,必须确保每一个展开式都精确无误。

最后,提醒一点,在Taylor展开中,灵活地进行高阶无穷小的合并,也是非常重要的。

{\bf 例:}设$f(x)$在$x=0$附近二次可导,且
$$\limx{0}\left(1+x+\df{f(x)}{x}\right)^{1/x}=e^3,$$
求:
\begin{enumerate}[(1)]
  \setlength{\itemindent}{1cm}
  \item $f(0),f\,'(0),f\,''(0)$;
  \item $\limx{0}\left(1+\df{f(x)}{x}\right)^{1/x}$.
\end{enumerate}

[解]:注意到$1/x\to\infty\;(x\to0)$,故已知极限存在,当且仅当
$$\limx0\left(1+x+\df{f(x)}x\right)=1
\quad\Rightarrow\quad f(0)=\limx0f(x)=0.$$
又原式即为
$$\left\{\limx0\left(1+x+\df{f(x)}{x}\right)
^{\frac{x}{x^2+f(x)}}\right\}^{\limx0\frac{x^2+f(x)}{x^2}}=e^3,$$
故必有
$$\limx0\frac{x^2+f(x)}{x^2}=3\quad\Rightarrow\quad
\limx0\df{f(x)}{x^2}=2,$$
进而
$$f'(0)=\limx0\df{f(x)-f(0)}x=\limx0\df{f(x)}{x^2}\limx0x=0.$$
至此,由Taylor公式,可设$f(x)=\df{f''(0)}2x^2+\circ(x^2)$,于是
$$2=\limx0\df{f(x)}{x^2}=\df{f''(0)}2\quad\Rightarrow\quad
f''(0)=4.$$
综上,$f(x)=2x^2+\circ(x^2)$,
\begin{align}
	\limx{0}\left(1+\df{f(x)}{x}\right)^{1/x}
	&=\limx0(1+2x+\circ(x))^{\frac1x}\notag\\
	&=\left\{\limx0(1+2x+\circ(x))^{\frac1{2x+\circ(x)}}\right\}
	^{\limx0{[2+\circ(1)]}}=e^2.\notag	
\end{align}
\hfill$\Box$

\subsubsection{证明等式或不等式}

{\bf 例:}证明:$x>0$时,$e^x>1+x+\df{x^2}{2}$

[证]:$x>0$时,由Taylor公式,存在$\xi\in(0,x)$,使得
$$e^x=1+x+\df{x^2}2+\df{e^{\xi}}{3!}x^3>1+x+\df{x^2}2.$$
即证。\hfill$\Box$

{\bf 例:}设$f(x)$在$(a,+\infty)$内具有二阶导数,
且$f(x),f\,''(x)$在$(a,+\infty)$
有界,证明:$f\,'(x)$在$(a,+\infty)$有界。

[证]:由$f(x)$和$f''(x)$有界,可知存在$M>0$,对任意$x>a$,均有
$$|f(x)|\leq M,\quad |f''(x)|\leq M.$$
对任意$x>a$,有如下的Taylor公式:存在$\xi\in(x,x+1)$,使得
$$f(x+1)=f(x)+f'(x)+\df12f''(\xi).$$
从而可知
$$f'(x)=f(x+1)-f(x)-\df12f''(\xi),$$
进而有
\begin{align*}
	|f'(x)|&=|f(x+1)-f(x)-\df12f''(\xi)|
	\leq |f(x+1)|+|f(x)|+\df12|f''(\xi)|\\
	&\leq M+M+\df12M=\df52M.
\end{align*}
这意味着$f'(x)$当$x>a$时有界,即证。\hfill$\Box$

从上面的例题不难看出,{\b 用Taylor公式证明等式和不等式,因为涉及到具体的估计和放缩,
都必须使用Lagrange余项}。

{\bf 例:}证明:若在$(a,b)$内,$f''(x)>0$,则对任意$a<x_1<x_2<b$,恒有
$$f\left(\df{x_1+x_2}{2}\right)<\df{f(x_1)+f(x_2)}{2}.$$

[证]:记$c=\df{x_1+x_2}2$,则由Taylor公式,存在$\xi_1\in(x_1,c),
\xi_2\in(c,x_2)$,使得
$$f(x_1)=f(c)+f'(c)\df{x_1-x_2}2+f''(\xi_1)
\left(\df{x_1-x_2}2\right)^2$$
$$f(x_2)=f(c)+f'(c)\df{x_2-x_1}2+f''(\xi_2)
\left(\df{x_2-x_1}2\right)^2$$
两式相加,且由$f''(x)>0$,可得
$$f(x_1)+f(x_2)=2f(c)+[f''(\xi_1)+f''(\xi_2)]
\left(\df{x_2-x_1}2\right)^2>2f(c),$$
即证。\hfill$\Box$

{\bf 思考:}本例的结论可以进一步推广为:对任意$\lambda\in(0,1)$,
$$f(\lambda x_1+(1-\lambda)x_2)<\lambda
f(x_1)+(1-\lambda)f(x_2).$$
请考虑该如何证明。

{\bf 例:}设在$f(x)$在$[0,a]$上二阶连续可导,
$|f\,''(x)|\leq M$,且$|f(x)|$在$(0,a)$内可取到最大值,证明:
$$|f\,'(0)+f\,'(a)|\leq Ma.$$

[提示]:将$f'(0)$和$f'(a)$在最大值点处展开,然后将两个展开式合并即可。

{\bf 例:}设$f(x)$在$[0,1]$上二阶连续可导,且$f(0)=f(1)$,
$|f\,''(x)|\leq A$,证明:对任意$x\in(0,1)$,恒有
$$|f\,'(x)|\leq\df A4.$$

[证]:由Taylor公式,对任意$x\in(0,1)$,存在$\xi_1\in(0,x),
\xi_2\in(x,1)$,使得
$$f(0)=f(x)+f'(x)(-x)+\df{f''(\xi_1)}2x^2,$$
$$f(1)=f(x)+f'(x)(1-x)+\df{f''(\xi_2)}2(1-x)^2,$$
两式相减,且由$f(0)=f(1)$,可得
$$0=f'(x)+\df12\left[f''(\xi_2)(1-x)^2-f''(\xi_1)x^2\right],$$
由$|f''(x)|\leq A$,故
$$|f'(x)|\leq\df A2[(1-x)^2+x^2],$$
注意到当$x\in(0,1)$时,$(1-x)^2+x^2\leq1$,故$|f'(x)|\leq\df A2$,即证。
\hfill$\Box$

通过上面的例题,可以总结一下使用Taylor公式证明等式和不等式的一些规律:
{\b
\begin{itemize}
  \item 如果同时涉及到多个不同阶的导数,可以考虑使用Taylor公式;
  \item 必须使用带Lagrange余项的Taylor公式;
  \item 展开和被展开点的常见选择: 区间端点、 极(最)值点、
  区间中点、 距离为常数的点、 已知条件中提到的特殊点。
\end{itemize}
}

{\bf 例:}设$f(x)\in C^3[-1,1]$,且$f(-1)=0,f(1)=1,f'(0)=0$,
证明:存在$\xi\in(-1,1)$,使得$f'''(\xi)=3$.

[提示]:将$f(-1),f(1)$在$x=0$处展开到三阶带Lagrange余项的Taylor公式,
然后利用介值定理。

{\bf 例:}\ps{辅导书P153-例37}
设$f(x)$在$[a,b]$上二阶连续可导,且$f_+'(a)=f_-'(b)=0$,则至少存在一点
$c\in(a,b)$,使得
$$|f\,''(c)|\geq \df 4{(b-a)^2}|f(b)-f(a)|.$$

[提示]:将$f\left(\df{a+b}2\right)$在$x=a,b$展开,取绝对值放缩后,
利用介值定理。

\section{函数的单调性与凹凸性}

\subsection{单调性的判定}

利用Lagrange中值定理,可以很容易地证明如下结论:若$f(x)$在$[a,b]$上连续,
$(a,b)$内可导,且$f\,'(x)$恒大(小)于零,则$f(x)$在$[a,b]$上严格单调递增(减)。

需要注意的是,这仅仅是判定可导函数严格单调的充分条件,而非充要条件
(例如:$f(x)=x+\sin x$严格单调递增,但导数可能为零)。
当然,如果定理中的“大(小)于”改成“大(小)于等于”,
且结论改为(非严格的)单调递增(减)情形,则为充要条件。

仔细观察可以发现,如果定理中的“大(小)于”改成“大(小)于等于”,且导数不可能在任何一段
连续区间内恒等于零,则可推出函数为{\it 严格}单调的。

\begin{thx}
	{\bf 可导函数严格单调的充要条件:}设$f(x)$在$[a,b]$上连续,$(a,b)$内可导,
	则$f(x)$在$[a,b]$上严格单调递增,当且仅当:
	\begin{enumerate}[(1)]
% 	  \setlength{\itemindent}{1cm}
	  \item $f\,'(x)\geq 0,\;x\in(a,b)$
	  \item 在$(a,b)$的任意子区间上$f\,'(x)$不恒为零
	\end{enumerate}
\end{thx}

利用该结论,可以很方便地证明类似如下的例子:

{\bf 例:}讨论$y=x-\sin x$的单调性。

\begin{center}
	\resizebox{!}{5cm}{\includegraphics{./images/ch5/xpmSinx.pdf}}
	
	{\it $y=x\pm\sin x$的函数图形}
\end{center}

\subsubsection{单调性的应用}

证明不等式是单调性最常见的一种应用:

\begin{thx}
	{\bf 推论1:}设$\varphi(x),\psi(x) $均在$[a,b]$上可导,且:
	\begin{enumerate}[(1)]
% 	  \setlength{\itemindent}{1cm}
	  \item $\varphi'(x)>\psi'(x),\;x\in(a,b)$;
	  \item $\varphi(a)\geq\psi(a)$,
	\end{enumerate}
	则在$(a,b)$上,恒有$\varphi(x)>\psi(x)$。
\end{thx}

{\bf 例:}证明:当$x>0$时,恒有
$$x-\df 16x^3<\sin x<x.$$

\begin{center}
	\resizebox{!}{5cm}{\includegraphics{./images/ch5/xSinxx3.pdf}}
\end{center}

很多时候,为了证明不等式,需要多次应用以上的结论,从而有如下的推论:

\begin{thx}
	{\bf 推论2:}设$\varphi(x),\psi(x) $均在$[a,b]$上$n$阶可导,且:
	\begin{enumerate}[(1)]
% 	  \setlength{\itemindent}{1cm}
	  \item $\varphi^{(n)}(x)>\psi^{(n)}(x),\;x\in(a,b)$;
	  \item $\varphi^{(k)}(a)\geq\psi^{(k)}(a),k=0,1,2,\ldots,n-1$,
	\end{enumerate}
	则在$(a,b)$上,恒有$\varphi(x)>\psi(x)$。
\end{thx}


{\bf 例:}证明下列不等式
\begin{enumerate}[(1)]
  \setlength{\itemindent}{1cm}
  \item $\ln(1+x)>\df{\arctan x}{1+x},\;(x>0)$
  \item $e^x>1+x+\df
  {x^2}{2!}+\df{x^3}{3!}+\ldots+\df{x^n}{n!},\;(x>0,n\in\mathbb{N})$
\end{enumerate}

\begin{center}
	\resizebox{!}{4cm}{\includegraphics{./images/ch5/lnxArctanx.pdf}}\quad
	\resizebox{!}{4cm}{\includegraphics{./images/ch5/exTaylor.pdf}}
\end{center}

此外,单调性也常常用来判定解的唯一性,例如:

{\bf 例:}已知$x>0$时,
$$kx+\df1{x^2}=1$$
有且仅有一个根,求$k$的取值范围。\ps{此类题目的叙述一定要注意逻辑表达,
既要避免过于繁琐,也要防止遗漏重要的论证}

解:令$f(x)=kx^3-x^2+1\;(x>0)$。显然原方程在$x>0$内有且仅有一个根,
当且仅当$f(x)$在$x>0$内有唯一零点。

(1)$k\leq 0$时,$f(+\infty)=-\infty$,注意到$f(0+0)=1$,
故由介值定理,$f(x)$在$x>0$上至少有一个零点。又当$x>0$时,
$f'(x)=(3kx-2)x<0$,故$f(x)$严格单调递减,从而以上零点唯一;

(2)$k>0$时,$f(+\infty)=+\infty$,令$f'(x)=0$,可得$x=\df2{3k}$
为$f(x)$的在$x>0$内的唯一驻点,又$f''\left(\df2{3k}\right)=2>0$,
故$x=\df2{3k}$为$f(x)$的最小值点。$f_{\min}=f\left(\df2{3k}\right)=1
-\df4{27k^2}$,由此可知:

若$k<\df29\sqrt3$,则$f_{\min}<0$,由介值定理,$f(x)$至少有两个零点;

若$k>\df29\sqrt3$,则$f_{\min}>0$,则当$x>0$时$f(x)>0$,无零点;

若$k=\df29\sqrt3$,则$x=\df2{3k}$为$f(x)$的唯一零点。

综上,当$k\leq 0$或$k=\df29\sqrt3$即为所求。\hfill$\Box$

\subsection{凹凸性与拐点}

任给一条平面曲线,从几何的角度我们可能会这样看待它,它是平直的还是弯曲的?
它向什么方向弯曲和倾斜?它的平滑程度如何?它有多长?

通过前面的学习,我们已经了解了,导数可以用来刻画曲线的倾斜程度(斜率),
同时告诉我们曲线本身是否平滑。但是,这还只是部分地回答了上面的问题。
本小结我们将关注的焦点集中在曲线弯曲的问题上,从凹凸的角度来刻画曲线
的几何特征。对于剩余的问题,我们将会再本章稍后的内容中继续加以讨论。

% {\bf 约定:}以下的凹凸均指“上凹”和“上凸”\ps{\b 目前的教材上对于函数的凹凸和
% 曲线的凹凸定义有所不同,并且刚好相反:K}

对于凹凸这样的几何特征,通过观察我们可以这样加以描述,以凹为例:一段曲线称为
是凹的,可以理解为{\it 连接其上任意两点的曲线段总是位于连接两点的线段的下方}(可以重叠)。

\begin{center}
 	\includegraphics[width=0.9\textwidth]{./images/ch3/convexCurve.jpg}

	\kaishu 凹弧的几何特征
\end{center}

对于这一性质在数学上可以表达为:

\begin{thx}
	设函数$f(x)$在区间$I$上有定义,若对任意$x_1,x_2\in I$,
	以及任意$\lambda\in(0,1)$,有
% 	\ps{\b 直观地说,就是割线上的取值大于函数值,则称为凹,反之为凸}
	$$f[\lambda x_1+(1-\lambda)x_2]\leq\lambda f(x_1)+(1-\lambda)f(x_2)$$
	则称$f(x)$的图形在区间$I$上的是{\bf 凹弧}。若上式中的等号严格成立,则称其为
	{\bf 严格的凹弧}。类似地,可以定义{\bf 凸弧}和{\bf 严格的凸弧}。
\end{thx}
{\bf 注:}同济教材中的定义相当于取定$\lambda=\df12$,考虑到$x_1,x_2$的任意性,
其定义和以上的定义是完全等价的。请思考一下如何证明这一点?

显然,从定义上看,一段曲线是否为凹(凸)弧,与其是否是光滑的无关。事实上,由多段
折线连接而成的一段曲线也可能是凹(凸)的,但不可能是严格凹(凸)的(想想为什么?)。
而如果一段曲线具有较好的光滑性(可导性),能够对我们判定其凹(凸)提供更大的便利。

\begin{thx}
	{\bf 凹弧判定的充要条件:}
	设$f(x)$在$(a,b)$内可导,则$f(x)$的曲线在$(a,b)$是凹弧,当且仅当:
	对任意$x_1,x_2\in(a,b)$,恒有
	$$f(x_2)\geq f(x_1)+f\,'(x_1)(x_2-x_1). $$
	不等号严格成立时,对应于严格凹函数的情形。
\end{thx}
这个定理的含义是:{\it 凹弧上任意点处的切线总是位于曲线的下方}。

[证]:若$f(x)$的曲线为凹的,则对任意$x_1,x_2\in(a,b)$和任意$t\in[x_1,x_2]$,有
$$f(t)\leq\df{x_2-t}{x_2-x_1}f(x_1)+\df{t-x_1}{x_2-x_1}f(x_2)
=f(x_1)+\df{f(x_2)-f(x_1)}{x_2-x_1}(t-x_1),$$
从而
$$\df{f(t)-f(x_1)}{t-x_1}\leq\df{f(x_2)-f(x_1)}{x_2-x_1},$$
由极限的保号性,令$t\to x_1^+$,可得
$$f'(x_1)\leq\df{f(x_2)-f(x_1)}{x_2-x_1},$$
也即
$$f(x_2)\geq f(x_1)+f\,'(x_1)(x_2-x_1).$$

另一方面,由Lagrange中值定理,对任意$x_1,x_2\in(a,b)$,存在
$\xi\in(a,b)$,使得
$$\df{f(x_2)-f(x_1)}{x_2-x_1}=f'(\xi),$$
也即
$$f(x_2)=f(x_1)+f'(\xi)(x_2-x_1).$$

于是由
$$f(x_2)\geq f(x_1)+f\,'(x_1)(x_2-x_1)$$
可得
$$f'(x_1)\leq\df{f(x_2)-f(x_1)}{x_2-x_1}=f'(\xi).$$
从而
$$f(x_2)\geq f(x_1)+f'(x_1)(x_2-x_1).$$
\hfill$\Box$

这个定理虽然给出了凹弧判定的充要条件,但从结论的形式上看,是非常不易验证的。
如果进一步知道$f(x)$是二阶可导的,可以使用如下更为简单的判定方法:

\begin{thx}
	{\bf 凹凸性判定的充分条件:}
	设$f(x)$在$(a,b)$二阶可导,则
	\begin{enumerate}[(1)]
% 	  \setlength{\itemindent}{1cm}
	  \item 若$f\,''(x)$恒不小于零,$f(x)$为凹函数;
	  \item 若$f\,''(x)$恒不大于零,$f(x)$为凸函数。
	\end{enumerate}
\end{thx}

这个定理的证明需要用到Taylor公式,在第3.3节我们已经以例题的形式证明过,在此
不再重复。

一段曲线常常是由多段凹弧和凸弧相接构成的,例如:$y=x^3$的曲线在$x<0$的部分
是凸的,在$x>0$的部分是凹的。位于凹、凸弧交界处的点,我们一般称之为曲线的{\bf 拐点},
如果曲线在改点是二阶可导的,容易推出在该点处二阶导数为零。例如$x=0$就是$y=x^3$
对应曲线的拐点。但特别需要注意的是,二阶导数为零的点,不一定都是曲线的拐点。例如
$y=x^4$在$x=0$处二阶导数为零,但该点并不是对应曲线的拐点,而是一个{\it 极小值点}。

事实上,{\it\b 要判定一个点是否为曲线的拐点,只需判断一下在该点左右的二阶导数是否符号是相异的}。

\begin{shaded}
	{\bf 讨论:}由$f''(x_0)=0$是否可以推出$x=x_0$为拐点?

	[答]:不能!例如$y=x^3$和$y=x^4$,在$x=0$处,前者为拐点,后者是极值点!
	具体可参照如下的分析方法:
	\begin{center}
	\begin{tabular}{c||c|c|c|c}
		\hline 
		$f(x)=x^3$ & $x<0$ & $x=0$ & $x>0$ & \\ 
		\hline 
		$f(x)$ & - & 0 & + & \\ 
		\hline 
		$f'(x)$ & + & 0 & + & 拐点\\ 
		\hline 
		$f''(x)$ & - & 0 & + & \\ 
		\hline 
		$f'''(x)$ & + & + & + & \\ 
		\hline 
		 &  &  &  &  \\ 
		\hline 
	\end{tabular} 
	\begin{tabular}{c||c|c|c|c}
		\hline 
		$f(x)=x^4$ & $x<0$ & $x=0$ & $x>0$ & \\ 
		\hline 
		$f(x)$ & + & 0 & + & 极小值\\ 
		\hline 
		$f'(x)$ & - & 0 & + & \\ 
		\hline 
		$f''(x)$ & + & 0 & + & \\ 
		\hline 
		$f'''(x)$ & - & 0 & + & \\ 
		\hline 
		$f^{(4)}(x)$ & + & + & + & \\ 
		\hline 
	\end{tabular} 
	\end{center}
	
	综合来看,$f''(x)=0$的点只能作为可能的拐点,并且,拐点也可能出现在不可导点处。
\end{shaded}

\section{函数的极值与最值}

\subsection{极值}

\subsection{最值问题}

\section{分析绘图}

\section{曲率}

\subsection{弧微分}

\subsection{曲率的定义}

\subsection{曲率的应用}

极限是微积分理论体系中最基本也最重要的概念之一,后面我们将要讨论的导数(微分)和
积分的概念都是建立在极限概念之上的,更具体地说,导数和定积分都是特定类型的极限。

在上一章学习了数列极限的基础上,本章我们将进一步讨论更具一般性的函数的极限,二者
的定义和性质,从很多方面来看都是相似的,只是由于函数极限的类型更为丰富,相对而言
有关结论也更为多样,有些性质是数列极限所不具备的。

\section{函数极限的概念与性质}

函数极限是我们后续定义导数、积分的概念的基础,因此可以说函数极限的理论也是整个
微积分学的基础。



\section{函数极限的判敛}

\subsection{四则运算}

{\bf 定理3.2.1:}若函数极限存在,则极限运算可以和四则运算交换次序。

{\bf 注:}把握两点
\begin{itemize}
  \setlength{\itemindent}{1cm}
  \item 有限次四则运算
  \item 可以推广到初等函数
\end{itemize}







\subsection{夹逼定理}


\section{无穷大、无穷小和函数的渐近线}

在求解(函数)极限的问题中,分子分母同时趋于零
\ps{所谓不定是相对诸如有界量乘以无穷小(趋于零)、有界量除以无穷大一类容易确定的形式而言的}
(例如:$\limx{0}\df{\cos
x-\cos2x}{x^2}$),或者一个趋于零的函数和一个趋于无穷的函数的乘积(例如:$\limx{+\infty}x^2e^-x$)
的极限通常是较难求解的,这类问题我们称之为“$\df00$”和“$0\cdot\infty$” 型的{\it 不定式(极限)}!


{\bf 思考:}除了“$\df00$”和“$0\cdot\infty$”型的不定式,还可能有其他形式的不定式吗?

{\bf 答:}有,型如:“$\df{\infty}{\infty}$”、“$1^{\infty}$”、“$0^0$”,例如:
$\limx{+\infty}\df{\ln x}{x}$,$\limx{\infty}\left(1+\df1{x}\right)^{\sin
x}$,$\limx{0}x^{2x}$

\subsection{无穷大和无穷小}

{\bf 定义3.3.1:}$f(x)$是$x\to\Delta$时的无穷小$\Leftrightarrow\limx{\Delta}f(x)=0$

{\bf 注:}$\Delta$可任意代表$\infty,\;+\infty,\;-\infty,\;x_0,\;x_0^+,\;x_0^-$之一

{\bf 定理3.3.1-3.3.2}(无穷小的性质)
\begin{enumerate}[(1)]
  \setlength{\itemindent}{1cm}
  \item $\limx{\Delta}f(x)=A\in\mathbb{R}\Leftrightarrow
  f(x)-A$是$x\to\Delta$时的无穷小
  \item $x\to\Delta$时的有界函数与无穷小之积仍为$x\to\Delta$时的无穷小
  \item $x\to\Delta$时的有限个无穷小之和(积)仍为$x\to\Delta$时的无穷小
\end{enumerate}

{\bf 定义3.3.2:}$f(x)$是$x\to\Delta$时的无穷大$\Leftrightarrow\limx{\Delta}\df 1{f(x)}=0$,
可记为:
$$\limx{\Delta}f(x)=\pm\infty$$

{\bf 注:}{\b 无穷大有正负之分!!}

{\bf 【渐近线】}\ps{具体内容请阅读教材自学!}

$x\to x_0$时的无穷大意味着存在{\it 铅直渐近线};$x\to\pm\infty$时的无穷大
{\it 可能}意味着存在{\it 斜渐近线},例如:若$y=f(x)$当$x\to+\infty$时的
存在斜渐近线$y=kx+b$,则
$$k=\limx{+\infty}\df{f(x)}x,\quad b=\limx{+\infty}[f(x)-kx].$$
注意,{\b $x\to\pm\infty$时的斜渐近线可能是不同的!}例如:$y=x\arctan x$趋于
$x\to\pm\infty$时的斜渐近线分别为$y=\pm\df{\pi}2x$.

% {\bf P129-例4:}证明:$x+\sin x$是$x\to\infty$时的无穷大

{\bf 定理3.3.3:}在$x\to\Delta$的同一过程中:
\begin{enumerate}[(1)]
  \setlength{\itemindent}{1cm}
  \item 有界函数与无穷大之和仍为无穷大
  \item 有限个无穷大之乘积仍为无穷大({\it 但可能反号})
\end{enumerate}

{\bf 例:}证明:$f(x)=a_0x^n+a_1x^{n-1}+\ldots+a_n(n\in\mathbb{N})$
是$x\to\infty$时的无穷大,其中:$a_0,a_1,\ldots,a_n\in\mathbb{R},a_0\ne 0$

\begin{shaded}
	{\bf 无穷大之间的比较!}
	
	当$x\to+\infty$时,存在如下的大小关系:设$a>0$,
	$$\ln x<<x^a<<e^x<<\Gamma(x)<<x^x$$
	其中$\Gamma-$函数定义为$\Gamma(x)=\dint_0^{+\infty}t^{x-1}e^{-t}\d t$,
	满足$\Gamma(n)=(n-1)!,\;(n\in\mathbb{Z}_+)$
\end{shaded}

\subsection{无穷小的比较}



\subsection{斜渐近线}

若$y=f(x)$当$x\to+\infty$时以$y=kx+b$为斜渐进线,则有
$$k=\limx{+\infty}\df{f(x)}{x}=k,\quad b=\limx{+\infty}[f(x)-kx]$$

{\bf 例:}求曲线$y^2-x^2=2x$的渐近线。

{\bf 教材3.3.4节-例10:}求函数$f(x)=\df{2x^2-3}{x+1}$的渐近线。

\begin{center}
	\resizebox{!}{5cm}{\includegraphics{./images/ch3/asyx.pdf}}
\end{center}

\section{函数的连续性}



\subsection{基本性质}

{\bf 定理3.4.1-3.4.4:}


\subsection{连续函数在有界闭区间上的性质}

\subsubsection{【最值定理】}


\subsubsection{【介值定理】}



% \section*{补充例题}
% \addcontentsline{toc}{section}{补充例题}
% 
% {\bf 例}:证明函数$f(x)=\left\{\begin{array}{ll}
% x,\;&x\in\mathbb{Q}\\ 0,\;&\mathrm{else}
% \end{array}\right.$
% 仅在$x=0$连续
% 
% {\bf 证:}显然$f(0)=0$。注意到
% $$0<|f(x)|\leq |x|,$$
% 而$\lim\limits_{x\to 0}|x|=0$,故由夹逼定理
% $$\lim\limits_{x\to 0}f(x)=0=f(0),$$
% 也即$f(x)$在$x=0$连续。
% 
% \bigskip
% 
% 设$x_0\ne 0$,下证$\lim\limits_{x\to x_0}f(x)$不存在,进而可知
% $f(x)$在$x_0$处不连续。事实上,若设$\lim\limits_{x\to x_0}f(x)$存在,
% 则
% $$\lim\limits_{x\to x_0}D(x)=\frac{\lim\limits_{x\to x_0}f(x)}
% {\lim\limits_{x\to x_0}x}$$
% 也存在,从而与$D(x)$在任意点处极限不存在矛盾。
% 
% 综上,$f(x)$仅在$x=0$连续。
% 
% \newpage
% 
% \newpage



\newpage

\section*{课后作业}
\addcontentsline{toc}{section}{课后作业}

{\bf 【必作题】}

\begin{itemize}
  \item 习题3.1:4(1,3),7,15
  \item 习题3.2:2,3,4,5
  \item 习题3.3:2,4,5,7(2,6),8,9,10(3,4)
  \item 习题3.4:5(1-3),9,15,17
\end{itemize}

\bigskip

\hrule

\bigskip
\bigskip

{\bf 【思考题】}
\begin{itemize}
  \item 习题3.1:10,11,12
  \item 习题3.2:6,7
  \item 证明:$$\limn\left\{\lim\limits_{m\to\infty}\left[\cos^{2m}(n!\pi
	x)\right]\right\}=D(x)$$
	其中$D(x)$为Dirichlet函数
%   \item 自学3.3.3节:渐近线 
  \item 习题3.3:1,6,12,13
  \item  计算极限
  \item 习题3.4:1,3,13,14,16,19
  \item 设$a_1<a_2<\ldots<a_n$,证明以下方程有$n-1$个实根
	$$\df 1{x-a_1}+\df 1{x-a_2}+\ldots+\df 1{x-a_n}=0.$$
\end{itemize}

\newpage

\section*{补充例题}
\addcontentsline{toc}{section}{补充例题}



{\bf 例:}设$f(x)$在$x_0$的某邻域内有定义,且$x_0$为其间断点,则下列函数
必以$x_0$为间断点的是(B)

\quad
(A)\;$f(x)\sin x$\hspace{2cm}
(B)\;$f(x)+\sin x$\hspace{2cm}
(C)\;$f^2(x)$\hspace{2cm}
(D)\;$|f(x)|$ 

{\bf 例:}设$x\to 0$时,$e^{\tan x}-e^x$与$x^n$为同阶无穷小,则$n$=(C)
\ps{Taylor展开}

\quad
(A)\;$1$\hspace{2cm}
(B)\;$2$\hspace{2cm}
(C)\;$3$\hspace{2cm}
(D)\;$4$ 

{\bf 例:}设$x\to 0$时,$x-\sin ax$与$x^2\ln(1-bx)$为同阶无穷小,则(A)
\ps{Taylor展开}

\quad
(A)\;$a=1,b=-\df16$\hspace{1em}
(B)\;$a=1,b=\df16$\hspace{1em}
(C)\;$a=-1,b=-\df16$\hspace{1em}
(D)\;$a=-1,b=\df16$

{\bf 例:}若$f(x)=\df{\sqrt[3]{x}}{\lambda-e^{-kx}}$在$(-\infty,+\infty)$
上连续,且$\limx{+\infty}f(x)=0$,则(D)

\quad
(A)\;$\lambda<0,k<0$\hspace{1cm}
(B)\;$\lambda<0,k>0$\hspace{1cm}
(C)\;$\lambda\geq0,k<0$\hspace{1cm}
(D)\;$\lambda\leq0,k>0$

% \begin{tabbing}
% 	\hspace{3cm}\=\hspace{3cm}\=\hspace{3cm}\=\kill
% 	\quad\quad\quad
% 	(A)\;$\lambda<0,k<0$\>  
% 	\quad\quad\quad
% 	(B)\;$\lambda<0,k>0$\>
% 	\quad\quad\quad  
% 	(C)\;$\lambda\geq0,k<0$\>
% 	\quad\quad\quad 
% 	(D)\;$\lambda\leq0,k>0$
% \end{tabbing} 

{\bf 例:}设$f(x)$在$(a,b)$内均有定义且单调有界,则$f(x)$在$(a,b)$内的间断点类型只能是(C)
\begin{tabbing}
	\hspace{8cm}\=\kill
	\quad\quad\quad
	(A)\;可去间断点 \> 
	(B)\;第二类间断点 \\ 
	\quad\quad\quad
	(C)\;跳跃间断点\>
	(D)\;不能确定
\end{tabbing}



{\bf 例:}



{\bf 例:}设$x\in(0,1]$时,$f(x)=x^{\sin x}$,且对任意$x$
$$f(x)+k=2f(x+1),$$
求常数$k$的值,使得极限$\limx0f(x)$存在。

[提示]:易知$\limx{0^+}f(x)=1$,有$x\in(-1,0]$时,
$$f(x)=2(x+1)^{\sin(x+1)}-k,$$
故$\limx{0^-}f(x)=2-k$,从而可得$k=1$.

{\bf 例:}设$f(x)$在$[a,b]$上连续,$\{x_n\}$为$[a,b]$上任一数列,求
$\limn\sqrt[n]{\df1n\sum\limits_{k=1}^ne^{f(x_k)}}$

[提示]:$e^{f(x)}$在$[a,b]$上连续且非负,故可设$m,M$分别为其在$[a,b]$上的最大和最小值,
从而由夹逼定理
$$\sqrt[n]m\leq\sqrt[n]{\df1n\sum\limits_{k=1}^ne^{f(x_k)}}\leq\sqrt[n]M,$$
由此易知原式$=1$。
