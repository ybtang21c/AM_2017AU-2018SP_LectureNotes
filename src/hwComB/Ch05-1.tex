% !Mode:: "TeX:UTF-8"

\titlepage

% \begin{frame}{说在前面}
% 	\linespread{1.5}
% 	  \begin{itemize}[<+-|alert@+>]
% 	    \item 过往的作业不订正、不补齐的不予批改,打分不超过\,\ba{C}
% 	    \item 不交作业的默认记为\,\ba{D}
% 	    \item 需要换作业本的,请“移植”照片,写清楚个人信息
% 	    \item 请自行完成SPOC课程中的测试
% 	    \item 第二次单元测试即将发布,成绩记入期末总成绩
% 	  \end{itemize}
% \end{frame}

% \begin{frame}{需要注意的问题}
% 	\linespread{1.5}
% 	  \begin{itemize}%[<+-|alert@+>]
% 	    \item L'Hospital法则
% 	    \begin{itemize}
% 	      \item \it 只能应用于“$\df{\bm{0}}{\bm{0}}$”
% 	      和“$\df{\bm{\infty}}{\bm{\infty}}$”型
% 	      \item \it 及时使用无穷小代换进行简化
% 	      \item \it 不正规的符号:\b 
% 	      $\xlongequal{\footnotesize\mbox{“L”}}$、
% 	      $\xlongrightarrow{\footnotesize\mbox{“L'Hospital法则”}}$、
% 	      $\df{\bm{0}}{\bm{0}}$、$\df{\bm{\infty}}{\bm{\infty}}$
% 	    \end{itemize}
% 	    \item Taylor公式
% 	    \begin{itemize}
% 	      \item \it Taylor多项式不包含余项
% 	      \item \it 合并同次幂的系数
% 	      \item \it 尽量按照幂次由低到高排列,最后写余项
% 	    \end{itemize}
% 	  \end{itemize}
% \end{frame}

\begin{frame}{出现的问题}
	\linespread{1.5}
	{\bf 错在哪里?}
	
	\it\b
	由定积分中值定理,存在$\xi\in[a,b]$,使得
	\begin{align*}
		&\dint_a^bf(x)g(x)\d x=f(\xi)g(\xi)(b-a),\\
		&\dint_a^bg(x)\d x=g(\xi)(b-a),
	\end{align*}
	进而可得
	$$\dint_a^bf(x)g(x)\d x=f(\xi)\dint_a^bg(x)\d x.$$
\end{frame}

\begin{frame}{出现的问题}
	\linespread{1.5}
	{\bf 错在哪里?}
	
	\it\b
	由Lagrange中值定理,存在$\xi\in[a,b]$,使得
	\begin{align*}
		&f(b)-f(a)=f'(\xi)(b-a),\\
		&g(b)-g(a)=g'(\xi)(b-a),
	\end{align*}
	进而可得
	$$\df{f(b)-f(a)}{g(b)-g(a)}=\df{f'(\xi)}{g'(\xi)}.$$
\end{frame}

\section{5.1 定积分的概念与性质}

\begin{frame}
	\linespread{1.5}
	\ba{1.$f(x)$在$[a,b]$上非负,$x\in(a,b)$时$f''(x)>0$,$f'(x)<0$,
	$I_1=\frac{b-a}2[f(b)+f(a)]$,$I_2=\dint_a^bf(x)\d x$,$I_3=(b-a)f(b)$,
	试比较$I_1,I_2,I_3$的大小。}
	\pause
	
% 	\bigskip
	
	\small 解:\it
	$f'(x)<0$,故对任意$x\in(a,b)$,均有$f(a)>f(x)>f(b)$,从而
	$$(b-a)f(a)>\dint_a^bf(x)\d x>(b-a)f(b),$$
	由此可知$I_2>I_3$。
	\pause
	
	令$L(x)=f(a)+\df{f(b)-f(a)}{b-a}(x-a)$,
	由$f''(x)>0$,可知对任意$x\in(a,b)$,均有$f(x)<L(x)$,进而
	由定积分的保号性,可得
	$$\dint_a^bf(x)\d x<\dint_a^bL(x)\d x=\df{b-a}2[f(b)+f(a)],$$
	由此可得$I_2<I_1$。
	
	综上,$I_1>I_2>I_3$。\hfill$\Box$
\end{frame}

\begin{frame}
	\linespread{1.5}
	\ba{3.设$f(x)\in C[a,b]$,$g(x)>0$,证明:存在$\xi\in[a,b]$,使得
	$$\dint_a^bf(x)g(x)\d x=f(\xi)\dint_a^bg(x)\d x.$$}
	\pause
	
% 	\bigskip
	
	\small 证:\it 记$c=\frac{\int_a^bf(x)g(x)\d x}{\int_a^bg(x)\d x}$。
	由$f(x)\in C[a,b]$,可设$m,M$分别为$f(x)$在$[a,b]$上的
	最小和最大值,进而可知
	$$m\dint_a^bg(x)\d x\leq \dint_a^bf(x)g(x)\d x\leq M\dint_a^bg(x)\d x,$$
	从而	$m\leq c\leq M,$
	从而由连续函数的介值定理可知,必存在$\xi\in[a,b]$,使得
	$f(\xi)=c$,
	即证。\hfill$\Box$
\end{frame}

\begin{frame}
	\linespread{1.5}
	\ba{4.已知$f(x)$在$[0,\pi]$上连续,$(0,\pi)$内可导,且$f(0)=
	\dint_0^{\pi}f(x)\sin x\d x=0$,证明:存在$\xi\in(0,\pi)$,
	使得$f'(\xi)=0$。
	}\pause
	
	\bigskip
	
	\small 证:\it 
	$\dint_0^{\pi}f(x)\sin x\d x=0$,由定积分中值定理,可知必存在$\eta\in(0,\pi)$,
	使得$f(\eta)\sin\eta=0$。注意到$x\in(0,\pi)$时,$\sin x>0$,故必有
	$f(\eta)=0$。\pause
	
	至此,可以验证$f(x)$在$[0,\eta]$上满足Rolle定理条件,
	进而可知必存在$\xi\in(0,\eta)\subset (0,\pi)$,使得$f'(\xi)=0$。
	\hfill$\Box$
\end{frame}

\begin{frame}
	\linespread{1.5}
	\ba{5.设$f(x)$在$[0,1]$上连续,在$(0,1)$内可导,且$f(0)\cdot f(1)>0$,
	$f(1)+\dint_0^1f(x)\d x=0$,证明:存在$\xi\in(0,1)$,使得
	$f'(\xi)=\xi f(\xi)$。
	}\pause
	
	\bigskip
	
	\small 证:\it 
	不妨设$f(1)>0$,则$\dint_0^1f(x)\d x<0$,从而可知,存在$\eta_1\in(0,1)$,
	使得$f(\eta_1)<0$。进而由连续函数的介值定理,可知存在$\eta\in(0,\eta_1)$,
	使得$f(\eta)=0$。
	
	令$F(x)=e^{-\frac{x^2}2}f(x)$,可以验证$F(x)$在$[0,\eta]$上满足
	Rolle定理条件,从而可知必存在$\xi\in(0,\eta)\subset(0,1)$,使得
	$$F'(\xi)=e^{-\frac{\xi^2}2}[f'(\xi)-\xi f(\xi)]=0,$$
	因为$e^{-\frac{\xi^2}2}\ne0$,故必有$f'(\xi)-\xi f(\xi)=0$,即证。
	\hfill$\Box$
\end{frame}

\section{5.2 微积分基本公式}

\begin{frame}
	\linespread{1.5}
	\ba{1.计算下列定积分:
	(1)$\dint_{-1}^2[x]\max\{1,e^{-x}\}\d x$
	}
	
	\pause
	\small 解:\it 
	$$
		\dint_{-1}^2[x]\max\{1,e^{-x}\}\d x
		=-\dint_{-1}^0e^{-x}\d x+\dint_1^2\d x
		=2-e
	$$
\end{frame}

\begin{frame}
	\linespread{1.5}
	\ba{3.计算下列函数的导函数
	(1)$f(x)=\dint_0^xtg(x^2-t^2)\d t$,其中$g(x)$为连续函数;
	}
	
	\pause
	\small 解:\it 
	令$u=x^2-t^2$,则
	\begin{align*}
		f(x)&=\df12\dint_0^xg(x^2-t^2)\d t^2\\
		&=\df12\dint_{x^2}^0g(u)\d(x^2-u)
		=\df12\dint_0^{x^2}g(u)\d u,
	\end{align*}
	于是
	$$f'(x)=\df12g(x^2)2x=xg(x^2).$$
\end{frame}

\begin{frame}
	\linespread{1.5}
	\ba{3.计算下列函数的导函数
	(2)$f(x)=\dint_0^x\sin(x-t)^2\d t$。
	}
	
	\pause
	\small 解:\it 
	令$u=x-t$,则
	$$f(x)=\dint_x^0\sin u^2\d(x-u)=\dint_0^x\sin u^2\d u,$$
	故
	$$f'(x)=\sin x^2.$$
	\hfill$\Box$
\end{frame}

\begin{frame}
	\linespread{1.5}
	\ba{4.设曲线$y=f(x)$与$y=\dint_0^{\arctan x}e^{-t^2}\d t$在原点处
	相切,求$\limn nf\left(\df2n\right)$。
	}
	
	\pause
% 	\vspace{-1em}
	\small 解:\it 
	\begin{align*}
		\limn nf\left(\df2n\right)
		&=2\limn\df{f\left(\df2n\right)}{\df2n}
		=2\limx0\df{f(x)}x\\
		&=2\limx0f'(x)=2f'(0)=2y'(0)\\
		&=2e^{-\arctan^2 x}\left.\df1{1+x^2}\right|_{x=0}=2.
	\end{align*}
	\hfill$\Box$
\end{frame}

\begin{frame}
	\linespread{1.5}
	\ba{6.计算如下极限:
	(3)$\limx{+\infty}\df{\dint_0^x(\arctan t)^2\d t}{\sqrt{x^2+1}}$
	}
	
	\pause
% 	\vspace{-1em}
	\small 解:\it 
	$$
	\mbox{原式}=\limx{+\infty}\df{\dint_0^x(\arctan t)^2\d t}{x}
	=\limx{+\infty}(\arctan x)^2=\df{\pi^2}4.
	$$
	\hfill$\Box$
\end{frame}