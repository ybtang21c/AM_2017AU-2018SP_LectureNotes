% !Mode:: "TeX:UTF-8"

\titlepage

\begin{frame}{说在前面}
	\linespread{1.5}
	  \begin{itemize}[<+-|alert@+>]
	    \item 过往的作业不订正、不补齐的不予批改,打分不超过\,\ba{C}
	    \item 不交作业的默认记为\,\ba{D}
	    \item 需要换作业本的,请“移植”照片,写清楚个人信息
	    \item 请自行完成SPOC课程中的测试
	    \item 第二次单元测试即将发布,成绩记入期末总成绩
	  \end{itemize}
\end{frame}

\begin{frame}{出现的问题}
	\linespread{1.5}
	  \begin{itemize}[<+-|alert@+>]
	    \item 用中值定理证明
	    \begin{itemize}
	      \item \it 写清楚理论依据:\b 由\ldots 定理 
	      \item \it 注意陈述的完整性:\b 存在$\xi\in(a,b)$
	      \item \it 注意语序:强烈建议不要从结果说起,写得像分析过程
	    \end{itemize}
	    \item 数学符号的书写
	    \begin{itemize}
	      \item \it 希腊字母:\b $\xi$像$\e$,$\eta$像$n$,$\mu$像$u$
	      \item \b\it $\ln 2$写成了$\ln^2$
	      \item \it 微分的表示:\b $1+\ln2\d x$和$(1+\ln2)\d x$不同!
	    \end{itemize}
	  \end{itemize}
\end{frame}

\section{2.6 微分}

\begin{frame}
	\linespread{1.5}
	\ba{1.设$y=x^3-2x$,
	\begin{enumerate}[(1)]
	  \setlength{\itemindent}{1cm}
	  \item 计算在$x=2$处当$\Delta x$分别为$1,\;0.1,\;
	  0.01$时的$\Delta y$和$\d y$;
	  \item 写出$x=2$时$y$的微分表达式。
	\end{enumerate}}
	\pause
	
% 	\bigskip
	
	\small 解:\it
	\begin{align*}
		\Delta y&=[(x+\Delta x)^3-2(x+\Delta x)]-(x^3-2x)\\
		&=(3x^2-2)\Delta x+3x\Delta x^2+\Delta x^3,\\
		\d y&=(3x^2-2)\d x=(3x^2-2)\Delta x.
	\end{align*}
\end{frame}

\begin{frame}
	\linespread{1.5}
	
	\small\it 当$x=2$时,
	\begin{align*}
		\Delta y|_{x=2}&
		=10\Delta x+6\Delta x^2+\Delta x^3,\\
		\d y|_{x=2}&=10\Delta x.
	\end{align*}
	\pause 进而
	\begin{enumerate}[(i)]
	  \item 当$x=2,\Delta x=1$时,$\Delta y=17,\;\d y=10$;
	  \item 当$x=2,\Delta x=0.1$时,$\Delta y=1.061,\;\d y=1$;
	  \item 当$x=2,\Delta x=0.01$时,$\Delta y=0.100601,\;\d y=0.1$。
	\end{enumerate}
	\hfill$\Box$
\end{frame}

\begin{frame}
	\linespread{1.5}
	\ba{2.已知$2^{xy}=x-y$,求$\d y|_{x=0}$。
	}\pause
	
	\bigskip
	
	\small 解:\it 
	已知方程两边求微分,可得
	$$2^{xy}\ln2(x\d y+y\d x)=\d x-\d y,$$
	进而
	$$\d y=\df{1-y2^{xy}\ln2}{1+x2^{xy}\ln2}\d x.$$
	\pause $x=0$时,$y=-1$,带入即得
	$$\d y|_{x=0}=(1+\ln2)\d x.$$
	\hfill$\Box$
\end{frame}

\begin{frame}
	\linespread{1.5}
	\ba{5.已知当$h\to 0$时,
	$$f(x+2h)-f(x)=h\sqrt{x^2+2x}+\circ(h),$$
	求$\d\left[f\left(\df1x\right)\right]$。}\pause

	\bigskip
	
	\small 解: \it 
	由已知
	$$\lim\limits_{h\to0}\df{f(x+2h)-f(x)}{2h}
	=\lim\limits_{h\to0}\df{h\sqrt{x^2+2x}+\circ(h)}{2h}
	=\df{\sqrt{x^2+2x}}2.
	$$
	也即$f'(x)=\df{\sqrt{x^2+2x}}2$,\pause 故
	$$\d\left[f\left(\df1x\right)\right]
	=f'\left(\df1x\right)\left(-\df1{x^2}\right)\d x
	=-\df{\sqrt{1+2x}}{2x^2|x|}\d x.$$
	\hfill$\Box$
\end{frame}

\begin{frame}
	\linespread{1.5}
	\ba{6.设$a>0$,$|x|<<a$(表示$|x|$远远小于$a$),证明近似公式
	$$\sqrt[n]{a^n+x}\approx a+\df{x}{na^{n-1}}.$$}\pause
	
	\bigskip
	
	\small 证:\it
	记$f(x)=\sqrt[n]{a^n+x}$,显然$f(0)=a$,
	又
	$$f'(x)=\df1n(a^n+x)^{\frac1n-1}\quad
	\Rightarrow f'(0)=\df1{na^{n-1}}.
	$$
	故由微分的几何意义,当$|x|<<a$时,总有
	$$\sqrt[n]{a^n+x}\approx f(0)+f'(0)x
	=a+\df{x}{na^{n-1}}.$$
	\hfill$\Box$
\end{frame}

\section{3.1 微分中值定理}

\begin{frame}
	\linespread{1.5}
	\ba{1.设$0<a<b$,$f(x)$在$[a,b]$上连续,在$(a,b)$内可导,证明:
	(1)$\exists\xi\in(a,b)$,使得:
	$f(b)-f(a)=\ln\df ba\cdot \xi f\,'(\xi)$}\pause
	
	\bigskip
	
	\small 证:\it
	注意到$f(x)$和$\ln x$在$[a,b]$上均连续,在
	在$(a,b)$内均可导,且$(\ln x)'=\df1x\ne 0$,故由Cauchy中值定理,
	存在$\xi\in(a,b)$,使得
	$$\df{f(b)-f(a)}{\ln b-\ln a}=\df{f'(\xi)}{\frac1{\xi}},$$
	整理后即证。\hfill$\Box$
\end{frame}

\begin{frame}
	\linespread{1.5}
	\ba{1.设$0<a<b$,$f(x)$在$[a,b]$上连续,在$(a,b)$内可导,证明:
	(2)$\exists\eta\in(a,b)$,使得:
    
    \centering $2\eta[f(b)-f(a)]=(b^2-a^2)f\,'(\eta)$
    
    }\pause
	
	\bigskip
	
	\small 证:\it
	注意到$f(x)$和$x^2$在$[a,b]$上均连续,在
	在$(a,b)$内均可导,且$(x^2)'=2x\ne 0$,故由Cauchy中值定理,
	存在$\eta\in(a,b)$,使得
	$$\df{f(b)-f(a)}{b^2-a^2}=\df{f'(\eta)}{2\eta},$$
	整理后即证。\hfill$\Box$
\end{frame}

\begin{frame}
	\linespread{1.5}
	\ba{1.设$0<a<b$,$f(x)$在$[a,b]$上连续,在$(a,b)$内可导,证明:
	(3)存在$x_1,x_2,x_3\in(a,b)$,使得
	
	\centering $f\,'(x_1)=(b+a)\df{f\,'(x_2)}{2x_2}=(a^2+ab+b^2)
	\df{f\,'(x_3)}{3x_3^2}$
    
    }\pause
	
	\bigskip
	
	\small 证:\it
	注意到$f(x)$和$x,x^2,x^3$在$[a,b]$上均连续,在
	在$(a,b)$内均可导,且$(x)'=x\ne 0,(x^2)'=2x\ne 0,
	(x^3)'=3x^2\ne 0$,故由Cauchy中值定理,
	存在$x_1,x_2,x_3\in(a,b)$,使得
	$$\df{f(b)-f(a)}{b-a}=f'(x_1),$$
	$$\df{f(b)-f(a)}{b^2-a^2}=\df{f'(x_2)}{2x_2}
	\quad\Rightarrow\quad
	\df{f(b)-f(a)}{b-a}=(b+a)\df{f'(x_2)}{2x_2},$$
	$$\df{f(b)-f(a)}{b^3-a^3}=\df{f'(x_3)}{3x_3^2}
	\quad\Rightarrow\quad
	\df{f(b)-f(a)}{b-a}=(a^2+ab+b^2)\df{f'(x_3)}{3x_3^2},$$
	整理后即证。\hfill$\Box$
\end{frame}

\begin{frame}
	\linespread{1.5}
	\ba{1.设$0<a<b$,$f(x)$在$[a,b]$上连续,在$(a,b)$内可导,证明:
	(4)若$f\,'(x)\ne 0$,则存在$\xi,\eta\in(a,b)$,使得
	
	\centering $\df{f'(\xi)}{f'(\eta)}=\df{e^b-e^a}{b-a}e^{-\eta}$
    
    }\pause
	
	\bigskip
	
	\small 证:\it
	由Lagrange和Cauchy中值定理,存在$\xi,\eta\in(a,b)$,使得
	$$\df{f(b)-f(a)}{b-a}=f'(\xi)
	\quad\Rightarrow\quad
	f(b)-f(a)=f'(\xi)(b-a).$$
	$$\df{f(b)-f(a)}{e^b-e^a}=\df{f'(\eta)}{e^{\eta}}
	\quad\Rightarrow\quad
	f(b)-f(a)=\df{e^b-e^a}{e^{\eta}}f'(\eta).$$
	整理后即证。\hfill$\Box$
\end{frame}

\begin{frame}
	\linespread{1.5}
	\ba{1.设$0<a<b$,$f(x)$在$[a,b]$上连续,在$(a,b)$内可导,证明:
	(5)存在$c\in(a,b)$,使得
	
	\centering $\df{1}{a-b}\left|\begin{array}{cc}
	a & b\\ f(a) & f(b)
	\end{array}\right|=f(c)-cf'(c).$
    
    }\pause
	
	\bigskip
	
	\small 证:\it
	由Cauchy中值定理,存在$c\in(a,b)$,使得
	\begin{align*}
		\df{1}{a-b}\left|\begin{array}{cc}
		a & b\\ f(a) & f(b)
		\end{array}\right|
		&=\df{af(b)-bf(a)}{a-b}=\df{\frac{f(b)}b-\df{f(a)}a}{\frac1b-\frac1a}\\
		&=\df{\left(\frac{f(x)}x\right)'_{x=c}}{\left(\frac1x\right)'_{x=c}}
		=f(c)-cf'c.
	\end{align*}
	即证。\hfill$\Box$
\end{frame}

\begin{frame}
	\linespread{1.5}
	\ba{1.设$0<a<b$,$f(x)$在$[a,b]$上连续,在$(a,b)$内可导,证明:
	(6)若$f(a)=0$,则存在$\mu\in(a,b)$,使得
	
	\centering $f'(\mu)=\df{a}{a+b-2\mu}f(\mu).$
    
    }\pause
	
	\bigskip
	
	\small 证:\it
	令$F(x)=f(x)\left(\df{a+b-2x}2\right)^{\frac a2}$,可以验证
	$F(x)$在$\left[a,\df{a+b}2\right]$上满足Rolle定理条件,故存在
	$\mu\in\left(a,\df{a+b}2\right)\subset(a,b)$,使得
	$$F'(\mu)=\left(\df{a+b-2\mu}2\right)^{\frac a2-1}
	\left[f'(\mu)\left(\df{a+b-2\mu}2\right)
	-f(\mu)a\right]=0,$$
	$\left(\df{a+b-2\mu}2\right)^{\frac a2-1}\ne0$,故必有
	$f'(\mu)\left(\df{a+b-2\mu}2\right)-f(\mu)\df a2=0,$
	即证。\hfill$\Box$
\end{frame}

\begin{frame}
	\linespread{1.5}
	\ba{2.$f(x)\in C^1[a,b]$,$f(a)=f(b)=0$,$f'_+(a)f'_-(b)>0$,
	证明:$f(x)$在$(a,b)$内至少有一个零点。}\pause
	
	\bigskip
	
	\small 证:\it
	不妨设$f'_+(a)>0,f'_-(b)<0$。由$f'_+(a)>0$可知,存在$\delta_1>0$,
	使对任意$x\in(a,a+\delta_1)$,总有
	$$\df{f(x)-f(a)}{x-a}>0\quad\Rightarrow
	\quad f(x)=f(x)-f(a)>0.$$
	因此可以取到某个$x_1\in(a,b)$,使得$f(x_1)>0$。
	
	\pause 类似地,由$f'_-(b)<0$,可知比存在某个$x_2\in(a,b)$,使得$f(x_2)<0$。
	
	\pause 至此,由介值定理,必存在某个$\xi$介于$x_1$和$x_2$之间,使得$f(\xi)=0$。
	\hfill$\Box$
\end{frame}

\begin{frame}
	\linespread{1.5}
	\ba{3.证明:对任意$x>1$,
	
	\centering $(x^2-1)\ln x\geq(x-1)^2$。
	
	}\pause
	
	\bigskip
	
	\small 证:\it
	由Lagrange中值定理,存在$\xi$介于$x$和$1$之间,使得
	$$\df{\ln x-\ln 1}{x-1}=\df1{\xi},$$
	\pause 显然$0<\xi<x+1\Rightarrow\df1{\xi}>\df1{x+1}$,带入前式整理后即证。
	\hfill$\Box$
\end{frame}

\begin{frame}
	\linespread{1.5}
	\ba{4.$e<a<b<e^2$,证明:$\ln^2b-\ln^2a>\df2{e^2}(b-a)$。}\pause
	
	\bigskip
	
	\small 证:\it
	由Lagrange中值定理,存在$\xi\in(a,b)$,使得
	$$\df{\ln^2b-\ln^2a}{b-a}=\df{2\ln\xi}{\xi},$$
	注意到$e<\xi<e^2$,故
	$$1<\ln\xi<2,\quad \xi<e^2,$$
	故
	$\df{2\ln\xi}{\xi}>\df2{e^2},$
	代入前式整理即证。
	\hfill$\Box$
\end{frame}

\begin{frame}
	\linespread{1.5}
	\ba{5.$f(x)$在$x>0$二阶可导,$f''(x)>0$,令$u_n=f(n)\;
	 (n\in\mathbb{Z}_+)$,证明:若$u_1<u_2$,则$\{u_n\}$必发散;
	问:若$u_1>u_2$,$\{u_n\}$是否必收敛?(若正确,证明之;否则给出反例。)}\pause
	
	\bigskip
	
	\small 证:\it
	由$f''(x)>0$可知,$f'(x)$单调递增。若$u_1<u_2$,也即$f(1)<f(2)$,
	由Lagrange中值定理,存在$\xi_1\in(1,2)$,使得
	$$0<\df{f(2)-f(1)}{2-1}=f'(\xi_1),$$
	进而可知对任意$x>2$,均有$f'(x)>f'(\xi_1)$。
\end{frame}

\begin{frame}
	\linespread{1.5}
% 	\ba{5.$f(x)$在$x>0$二阶可导,$f''(x)>0$,令$u_n=f(n)\;
% 	 (n\in\mathbb{Z}_+)$,证明:若$u_1<u_2$,则$\{u_n\}$必发散;
% 	问:若$u_1>u_2$,$\{u_n\}$是否必收敛?(若正确,证明之;否则给出反例。)}\pause
	
% 	\bigskip
	
	\small \it
	对任意$n\geq2$,由Lagrange中值定理,存在$\xi_n\in(n,n+1)$,使得
	$$\df{f(n+1)-f(n)}{(n+1)-n}=f'(\xi_n)>f'(\xi_1)$$
	进而可得
	$$
	f(n+1)>f(n)+f'(\xi_1)>f(n-1)+2f'(\xi_1)>\ldots>f(1)+nf'(\xi_1),
	$$
	由此显然可知$\{u_n\}$发散。
	
	若$u_1>u_2$,$\{u_n\}$未必收敛。例如:$f(x)=(x-3)^2$,其中$u_1=4>1=u_2$,
	但显然$\{(n-3)^2\}$发散。\hfill$\Box$
\end{frame}