% !Mode:: "TeX:UTF-8"

\titlepage

\begin{frame}{说在前面}
	\linespread{1.5}
	  \begin{itemize}[<+-|alert@+>]
	    \item 过往的作业只画$\color{red}\times$和划线,不画$\color{red}\surd$,
	    最后写{\bb “查”};
	    \item 就近订正,方便查阅,用不同颜色的笔;
	    \item 不要忘记贴照片,作业本封面注明学员队和班号!
	    \item 作业本不要分栏写!
	    \item 不要当“复印机”!
	    \item 作业晚交、进度过慢或缺题过多最高得$\b B$,
	    不交默认为$\b C$!
% 	    \item {\b 今后每周一交作业,按照单双周对应学号尾数为单双号的同学分别上交,
% 	    平均每人两周交一次作业。} 
	  \end{itemize}
\end{frame}

\begin{frame}{出现的问题}
	\linespread{1.5}
	  \begin{itemize}[<+-|alert@+>]
	    \item 口语化的表达
	    \begin{itemize}
	      \item {\it\b $f(x)$在\ldots附近,两头收敛}
	      \item {\it\b $x=0$附近的情况与$x=2$附近的情况相同}
	    \end{itemize}
	    \item 概念错误
	    \begin{itemize}
	      \item {\it\b $\limx0f(x)=\limx0\sin x$,故$\limx0\df{f(x)}{\sin x}=1$}
	      \item {\it\b $f'_+(0)=\limx{0^+}f'(x)$} 
	    \end{itemize}
	    \item 自己发明符号
		$$\b y'_{x\to 0^+},\quad f'(x)_{\mbox{\footnotesize\it 左}},
		\quad f\left.\left(\arctan\df{1+x}{1-x}\right)'\right|_{x=0}$$
	  \end{itemize}
\end{frame}

\section{习题1-9}

\begin{frame}
	\linespread{1.5}
	\ba{2.设函数$f(x)$与$g(x)$在点$x_0$连续,证明函数
	$$\varphi(x)=\max\{f(x),g(x)\},\quad
	\psi(x)=\min\{f(x),g(x)\}$$
	在$x_0$也连续。}\pause
	
% 	\bigskip
	
	证:\it
	$$\max\{f(x),g(x)\}=\df12[f(x)+g(x)+|f(x)-g(x)|],$$
% 	$$\min\{f(x),g(x)\}=\df12[f(x)+g(x)-|f(x)-g(x)|],$$
	\pause 由已知,$\limx{x_0}f(x)=f(x_0),
	\limx{x_0}g(x)=g(x_0),$
	\pause 进而由极限的性质可得
	$$\limx{x_0}|f(x)-g(x)|=|f(x_0)-g(x_0)|,$$
\end{frame}

\begin{frame}
	\linespread{1.5}
% 	\ba{2.设函数$f(x)$与$g(x)$在点$x_0$连续,证明函数
% 	$$\varphi(x)=\max\{f(x),g(x)\},\quad
% 	\psi(x)=\min\{f(x),g(x)\}$$
% 	在$x_0$也连续。}\pause
	
% 	\bigskip
	
	\it\small 于是
	\begin{align*}
		\limx{x_0}\varphi(x)&=\limx{x_0}\max\{f(x),g(x)\}\\
		&=\limx{x_0}\df12[f(x)+g(x)+|f(x)-g(x)|]\\
		&=\df12\left[\limx{x_0}f(x)+\limx{x_0}g(x)+\limx{x_0}|f(x)-g(x)|\right]\\
		&=\df12[f(x_0)+g(x_0)+|f(x_0)-g(x_0)|]\\
		&=\max\{f(x_0),g(x_0)\}=\varphi(x_0).
	\end{align*}
	\pause 也即 $\varphi(x)$在$x_0$连续。
	\pause
	$$\psi(x)=\min\{f(x),g(x)\}=\df12[f(x)+g(x)-|f(x)-g(x)|],$$
	同理可证$\psi(x)$在$x_0$连续。\hfill$\Box$
\end{frame}

\section{习题1-10}

\begin{frame}
	\linespread{1.5}
	\ba{7.证明:若$f(x)$在$(-\infty,+\infty)$内连续,且$\limx{\infty}f(x)$
	存在,则$f(x)$必在$(-\infty,+\infty)$内有界。}\pause
	
% 	\bigskip
	
	证:\it
	$\limx{\infty}f(x)$存在,由极限的有界性,$\exists M_1>0$和$X>0$,
	对$\forall |x|>X$,均有$|f(x)|\leq M_1$。
	\pause
	注意到$f(x)\in C[-X,X]$,故$\exists M_2>0$,对$\forall x\in[-X,X]$,
	均有$|f(x)|\leq M_2$。
	\pause
	综上,对任意$x\in(-\infty,+\infty)$,均有
	$$|f(x)|\leq M=\max\{M_1,M_2\}.$$
	即证。\hfill$\Box$
\end{frame}

\section{总习题一}

\begin{frame}
	\linespread{1.5}
	\ba{14.如果存在直线$L:y=kx+b$,使当$x\to+\Delta$时
	($\Delta$表示$\infty,+\infty,-\infty$之一),
	曲线$y=f(x)$上的动点$M(x,y)$到直线$L$的距离
	$d(M,L)\to 0$,那么称$L$为直线$y=f(x)$的渐近线。当直线$L$的斜率
	$k\ne 0$时,称$L$为斜渐近线。
	\begin{enumerate}[(1)]
% 	  \setlength{\itemindent}{1cm}
	  \item 证明:直线$L:y=kx+b$为曲线$y=f(x)$的渐近线的充分必要条件是
	  $$k=\limx{\Delta}\df{f(x)}x,\quad
	  b=\limx{\Delta}[f(x)-kx].$$
	  \item 求曲线$y=(2x-1)e^{\frac1x}$的斜渐近线。
	\end{enumerate}}
\end{frame}

\section{1.2 导数的概念}

\begin{frame}
	\linespread{1.5}
	\ba{1.已知$f(x)=\left\{\begin{array}{ll}
	2e^x+b,& x\leq0\\ ax+\sin x,& x> 0
	\end{array}\right.$
	试确定$a,b$的值,使得$f(x)$在$x=0$处可导。}\pause
	
% 	\bigskip
	
	\small 证:\it
	$f(x)$在$x=0$处可导,故必连续,从而
	$$f(0-0)=2+b=f(0+0)=0\quad\Rightarrow \quad b=-2.$$
	\pause 又
	$$f'_-(0)=(2e^x+b)'|_{x=0}=2.$$
	$$f'_+(0)=\limx{0^+}\df{ax+\sin x-(2+b)}{x}
	=a+\limx{0^+}\df{\sin x}x=a+1.$$
	故要使$f(x)$在$x=0$处可导,必有$2=a+1$,从而$a=1$。
	\hfill$\Box$
\end{frame}

\begin{frame}
	\linespread{1.5}
	\ba{3.讨论函数
	$y=\left\{\begin{array}{ll}
		x^2\sin\df1x,& x\ne0;\\ 0, & x=0.
	\end{array}\right.$
	在$x=0$处的连续性、可导性以及导函数的连续性。}\pause
	
% 	\bigskip
	
	\small 证:\it
	当$x\ne 0$时,
	$$y'(x)=2x\sin\df1x-\sin\df1x.$$
	\pause 当$x=0$时,
	$$y'(0)=\limx0\df{x^2\sin\df1x-0}{x-0}=\limx0x\sin\df1x=0.$$
	由此可知$y(x)$在$x=0$处可导,且连续。
\end{frame}

\begin{frame}
	\linespread{1.5}
	\small\it
	$$
		y'=\left\{\begin{array}{ll}
			2x\sin\df1x-\sin\df1x, & x\ne 0,\\
			0, & x=0.
		\end{array}\right.
	$$
	\pause
	注意到
	$$\limx02x\sin\df1x=0,\quad \limx0\sin\df1x\mbox{不存在},$$
	故$\limx0y'(x)$不存在。由此可知$y(x)$的导函数在$x=0$处不连续。\hfill$\Box$
	
	\bigskip
	\pause 
	\ba{注:可导函数的导函数不一定是连续函数!}
\end{frame}

\begin{frame}
	\linespread{1.5}
	\ba{4.设对任意$x\in\mathbb{R}$,均有$f(x+2)=f(x)$,已知$f'(0)=1$,
	证明$f(x)$在$x=2$可导,并求$f'(2)$。}\pause
	
	\bigskip
	
	\small 解:\it
	因为$f(x+2)=f(x)$,故
	$$\lim\limits_{\Delta x\to0}\df{f(2+\Delta x)-f(2)}{\Delta x}
	=\lim\limits_{\Delta x\to0}\df{f(\Delta x)-f(0)}{\Delta x}
	=f'(0)=1,$$
	由此即知$f(x)$在$x=2$可导,且$f'(2)=1$。
	\hfill$\Box$
	
	\bigskip
	\pause
	\ba{注:周期函数的导函数必为周期函数,反之不然。}
\end{frame}

\begin{frame}
	\linespread{1.5}
	\ba{5.已知曲线$y=f(x)$和曲线$y=\sin x$在原点相切(即二者的切线相同),
	求$\limx0\df{f(3x)}x$。}\pause
	
	\bigskip
	
	\small 解:\it
	曲线$y=f(x)$和曲线$y=\sin x$在原点相切,故
	$$f(0)=\sin 0=0,\quad f'(0)=(\sin x)'|_{x=0}=1.$$
	于是
	$$\limx0\df{f(3x)}x=3\limx0\df{f(3x)-f(0)}{3x}=3f'(0)=3.$$
	\hfill$\Box$
\end{frame}

\begin{frame}
	\linespread{1.5}
	\ba{6.函数$g(x)$在$x=a$连续,问函数$f(x)=|(x-a)|g(x)$
	在$x=a$是否可导?若可导,证明之;若不可导,讨论增加什么样的条件可以使之可导。
	利用以上讨论的结果,判断$f(x)=(x^2-4)|x^2+3x+2|$有几个不可导的点。}\pause
	
% 	\bigskip
	
	\small 解:\it
	$$\df{f(a+\Delta x)-f(a)}{\Delta x}
	=\df{|\Delta x|g(a+\Delta x)}{\Delta x}
	=\df{|\Delta x|}{\Delta x}g(a+\Delta x),$$
	\pause
	当$\Delta x\to0$时极限存在当且仅当$\lim\limits_{\Delta x\to0}g(a+\Delta x)=0$,也即
	$\limx{a}g(x)=0$。
	\pause
	$f(x)=|(x+2)(x+1)|(x+2)(x-2),$
	根据前述的结论,在$x=-2$处$f(x)$可导,在$x=-1$处$f(x)$不可导。\hfill$\Box$
\end{frame}

\begin{frame}
	\linespread{1.5}
	\ba{7.已知$f'(a)f(a)\ne 0$,求
	$\limx0\left[\df{f(a+x)}{f(a)}\right]^{\frac1{\sin x}}.$}\pause
	
% 	\bigskip
	
	\small 解:\it
	\begin{align*}
		\mbox{原式}&=\limx0\left[1+\df{f(a+x)-f(a)}{f(a)}\right]^{\frac1{\sin x}}\\
		&=\limx0\left[1+\df{f(a+x)-f(a)}{f(a)}\right]^{\frac{f(a)}{f(a+x)-f(a)}
		\frac{f(a+x)-f(a)}{f(a)}\frac1{\sin x}}\\
		&=\left\{\limx0\left[1+\df{f(a+x)-f(a)}{f(a)}\right]^{\frac{f(a)}{f(a+x)-f(a)}}\right\}
		^{\limx0\frac{f(a+x)-f(a)}x\frac{x}{\sin x}\frac1{f(a)}}\\
		&=e^{\frac{f'(a)}{f(a)}}
	\end{align*}
	\hfill$\Box$
\end{frame}

\begin{frame}
	\linespread{1.5}
	\ba{8.设对任意$x,y\in\mathbb{R}$,有
	$$f(x+y)=f(x)+f(y)+x^2y+xy^2,$$
	且当$x\to0$时$f(x)$与$x$是等价无穷小,证明$f(x)$处处可导,并求其导函数。}\pause
	
% 	\bigskip
	
	\small 解:\it
	令$x=y=0$,由已知等式可得$f(0)=0$。\pause
	对任意$x_0\in\mathbb{R}$,由已知等式及$\limx{0}\df{f(x)}x=1$,
	\begin{align*}
		\lim\limits_{\Delta x\to0}\df{f(x_0+\Delta x)-f(x_0)}{\Delta x}
		&=\lim\limits_{\Delta x\to0}\df{f(\Delta x)+x_0^2\Delta x+x_0\Delta x^2}
		{\Delta x}\\
		&=\lim\limits_{\Delta x\to0}\df{f(\Delta x)}{\Delta x}+x_0^2
		=1+x_0^2.
	\end{align*}
	\pause 因为$x_0$是任意的,故$f(x)$处处可导,其导函数为
	$f'(x)=1+x^2$。\hfill$\Box$
\end{frame}

\section{2.2 函数的求导法则}

\begin{frame}
	\linespread{1.5}
	\ba{4.设$f(x)$可导,且$f'\left(\df{\pi}{4}\right)=1$,求
	$$\left.f'\left(\arctan\df{1+x}{1-x}\right)\right|_{x=0}
	\quad\mbox{和}\quad  
	\left[f\left(\arctan\df{1+x}{1-x}\right)\right]'_{x=0}.$$}\pause
	
% 	\bigskip
	
	\small 解:\it
	$$\left.f'\left(\arctan\df{1+x}{1-x}\right)\right|_{x=0}
	=f'(\arctan 1)=f'\left(\df{\pi}{4}\right)=1.$$
	\pause
	\begin{align*}
		&\left[f\left(\arctan\df{1+x}{1-x}\right)\right]'_{x=0}
		=f'\left(\arctan\df{1+x}{1-x}\right)\\
		&\quad\left.\df{1}{1+\left(\frac{1+x}{1-x}\right)^2}
		\df{(1-x)+(1+x)}{(1-x)^2}
		\right|_{x=0}
		=1\cdot\df12\cdot2=1.
	\end{align*}
	\hfill$\Box$
\end{frame}