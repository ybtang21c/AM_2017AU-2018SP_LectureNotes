% !Mode:: "TeX:UTF-8"

\titlepage

\begin{frame}{说在前面}
	\linespread{1.5}
	  \begin{itemize}[<+-|alert@+>]
	    \item 作业中打$\color{red}\times$的必须订正,否则下次不予批改,
		默认评分为$C$以下;
	    \item 写过的作业不许撕掉,除非重写;
	    \item 不要忘记贴照片,作业本封面注明学员队和班号;
	    \item 作业本不要分栏写;
	    \item 抄作业也要动脑子,不要当“复印机”!
	    \item {\b 今后每周一交作业,按照单双周对应学号尾数为单双号的同学分别上交,
	    平均每人两周交一次作业。} 
	  \end{itemize}
\end{frame}

\begin{frame}{出现的问题}
	\linespread{1.5}
	  \begin{itemize}[<+-|alert@+>]
	    \item 解题过程叙述不清,无用的话太多,未说明推理的依据;
	    \item 概念使用不熟练,例如:一道题里出现多个$\e$,不会正确地表达无界,无穷大;
	    \item 放缩不熟练,例如:$N$的取值太复杂;
	    \item 使用了不存在或无意义的极限,例如:$\limx0\sin\df1x$;
	    \item 符号书写不规范,如:$\mathbb{Z},\;\e,\;\delta$;
	    \item 书写脏乱,字迹难以辨认。
	  \end{itemize}
\end{frame}

\section{习题1-1}

\begin{frame}
	\linespread{1.5}
	\ba{5.用定义证明极限:(1)$\limn\df{\sqrt{n^2+a^2}}n=1$}\pause
	
	\bigskip
	
	证:\it 对任意$\e>0$,令$N=\left[\df{a^2}{\e}\right]+1>\df{a^2}{\e}$,
	则对任意$n>N$,均有
	$$\left|\df{\sqrt{n^2+a^2}}n-1\right|
	=\df{a^2}{n(\sqrt{n^2+a^2}+n)}<\df{a^2}n<\df{a^2}N<\e,
	$$
	由数列极限的定义,即证。\hfill$\Box$
\end{frame}

\begin{frame}
	\linespread{1.5}
	\ba{5.用定义证明极限:(4)$\limn0.\underbrace{99\ldots9}
	_{n\mbox{\footnotesize 个}}=1.$}\pause
	
	\bigskip
	
	证:\it 对任意$\e>0$,令$N=[-\lg\e]+1$,则对任意$n>N$,均有
	$$|0.\underbrace{99\ldots9}_{n\mbox{\footnotesize 个}}-1|
	=\df1{10^n}<\df1{10^N}
	<\df1{10^{-\lg\e}}=\e,$$
	由数列极限的定义,即证。\hfill$\Box$
\end{frame}

\begin{frame}
	\linespread{1.2}
	\ba{7.数列$\{x_n\}$有界,$\limn y_n=0$,证明:$\limn x_ny_n=0$。}\pause
	
	\bigskip
	
	证:\it $\{x_n\}$有界,故存在$M>0$,对任意$n\in\mathbb{Z}_+$,均有
	$$|x_n|\leq M.$$\pause
	对任意$\e>0$,令$\e_1=\df{\e}M$,由$\limn y_n=0$,存在$N$,对任意$n>N$有
	$$|y_n-0|<\e_1.$$\pause
	综上,当$n>N$时,总有
	$$|x_ny_n-0|\leq M\e_1=\e,$$
	由数列极限的定义,即证。\hfill$\Box$
\end{frame}

\begin{frame}
	\linespread{1.2}
	\ba{8.对于数列$\{x_n\}$,若$x_{2k-1}\to a\,(k\to\infty)$,
	$x_{2k}\to a\,(k\to\infty)$,证明:
	$x_n\to a\,(n\to\infty)$。}\pause
	
	\bigskip
	
	证:\it 对任意$\e>0$,由$x_{2k-1}\to a\,(k\to\infty)$,存在$K_1$,
	对任意$k>K_1$,有$|x_{2k-1}-a|<\e$;\pause
	对以上的$\e>0$,由$x_{2k}\to a\,(k\to\infty)$,存在$K_2$,
	对任意$k>K_2$,有$|x_{2k}-a|<\e$。\pause
	
	综上,令$N=\max\{2K_1-1,2K_2\}$,则对任意$n>N$,均有
	$$|a_n-a|<\e.$$
	由数列极限的定义,即证。\hfill$\Box$
\end{frame}

\section{习题1-3}

\begin{frame}
	\linespread{1.5}
	\ba{5.根据函数极限的定义证明:(3)$\limx{-2}\df{x^2-4}{x+2}=-4$.}\pause
	
	\bigskip
	
	证:\it 对任意$\e>0$,令$\delta=\e$,则对任意$x\in U_0(-2,\delta)$,总有
	$$\left|\df{x^2-4}{x+2}-(-4)\right|=|x+2|<\delta=\e.$$
	由函数极限的定义,即证。\hfill$\Box$
\end{frame}

\begin{frame}
	\linespread{1.5}
	\ba{6.根据函数极限的定义证明:
	(2)$\limx{+\infty}\df{\sin x}{\sqrt x}$.}\pause
	
	\bigskip
	
	证:\it 对任意$\e>0$,令$X=\df1{\e^2}$,则对任意$x>X$,总有
	$$\left|\df{\sin x}{\sqrt x}-0\right|\leq\df1{\sqrt x}
	<\df1{\sqrt X}=\e.$$
	由函数极限的定义,即证。\hfill$\Box$
\end{frame}

\begin{frame}
	\linespread{1.2}
	\ba{7.当$x\to2$时,$y=x^2\to 4$,问$\delta$等于多少,使当$|x-2|<\delta$时,
	$|y-4|<0.001$?}\pause
	
% 	\bigskip
	
	解:\it $$|y-4|<0.001\quad\Leftrightarrow\quad 3.999<x^2<4.001.$$
	进而当$x>0$时,必有
	$\sqrt{3.999}<x<\sqrt{4.001}$。	由此可知须
	$$|x-2|<\min\{2-\sqrt{3.999},\sqrt{4.001}-2\}$$
	时,方能满足题目要求。注意到$2-\sqrt{3.999}>\sqrt{4.001}-2$,故
	取$\delta=\sqrt{4.001}-2$,即为所求。\hfill$\Box$
\end{frame}

\begin{frame}
	\linespread{1.5}
	\ba{8.当$x\to\infty$时,$y=\df{x^2-1}{x^2+3}\to1$,问$X$等于多少,使当
	$|x|>X$时,$|y-1|<0.01$?}\pause
	
	\bigskip
	
	解:\it 由题意
	$$|y-1|=\left|\df{x^2-1}{x^2+3}-1\right|=\df4{x^2+3}<0.01,$$
	由此可解得$|x|>\sqrt{397}$,故$X=\sqrt{397}$即为所求。\hfill$\Box$
\end{frame}

\section{习题1-4}

\begin{frame}
	\linespread{1.5}
	\ba{7.证明:函数$y=\df1x\sin\df1x$在区间$(0,1]$内无界,但不是$x\to0^+$
	时的无穷大。}\pause
	
	\bigskip
	
	证:\it 先证该函数无界。对任意$M>0$,总可令$x_M=\df1{2([M]+1)\pi+\frac{\pi}2}$,
	则
	$$
	y(x_M)=2([M]+1)\pi+\frac{\pi}2>M.$$
	由函数无界的定义,即证。
\end{frame}

\begin{frame}
	\linespread{1.2}
	\ba{7.证明:函数$y=\df1x\sin\df1x$在区间$(0,1]$内无界,但不是$x\to0^+$
	时的无穷大。}\pause
	
	\bigskip
	
	\it 下证该函数不是$x\to0^+$时的无穷大。用反证法,若该函数是$x\to0^+$时的
	无穷大,则
	$$\limx{0^+}\df1y=\limx{0^+}x\df1{\sin\frac1x}=0.$$
	考虑数列$x_n=\frac1{n\pi}$,显然$\limn x_n=0$,且$\sin\df1{x_n}=0$,
	此时$\df1{y(x_n)}$无定义,故$\left\{\df1{y(x_n)}\right\}$
	必不收敛,从而由Henie定理,前述极限不成立,
	假设错误,即证。\hfill$\Box$
\end{frame}

\begin{frame}
	\linespread{1.5}
	\ba{8.求函数$f(x)=\df4{2-x^2}$的渐近线。}\pause
	
	\bigskip
	
	解:\it 由$\limx{\pm\sqrt 2}\df1{f(x)}=\df{2-x^2}4=0$,可知$f(x)$
	是$x\to\pm\sqrt2$时的无穷大,故$x=\pm\sqrt2$为$f(x)$的两条铅直渐近线;
	
	又$\limx{\infty}f(x)=0$,故$y=0$是$f(x)$的水平渐近线。\hfill$\Box$
\end{frame}

\section{习题1-5}

\begin{frame}
	\linespread{1.5}
	\ba{3.计算下列极限:(1)$\limx0x^2\sin\df1x$}\pause
	
	\bigskip
	
	解:\it 因为$\limx0x^2=0$,$|\sin\df1x|\leq1$,由无穷小的性质(无穷小
	乘以有界量仍为无穷小),可知
	$$\limx0x^2\sin\df1x=0.$$
	\hfill$\Box$
\end{frame}

\begin{frame}
	\linespread{1.5}
	\ba{3.计算下列极限:(2)$\limx{\infty}\df{\arctan x}x$}\pause
	
	\bigskip
	
	解:\it 因为$\limx{\infty}\df1x=0$,$|\arctan x|\leq\df{\pi}2$,故
	无穷小的性质,可知
	$$\limx{\infty}\df{\arctan x}x=0.$$
	\hfill$\Box$
\end{frame}