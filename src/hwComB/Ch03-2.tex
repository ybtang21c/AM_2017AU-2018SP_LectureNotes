% !Mode:: "TeX:UTF-8"

\titlepage

% \begin{frame}{说在前面}
% 	\linespread{1.5}
% 	  \begin{itemize}[<+-|alert@+>]
% 	    \item 过往的作业不订正、不补齐的不予批改,打分不超过\,\ba{C}
% 	    \item 不交作业的默认记为\,\ba{D}
% 	    \item 需要换作业本的,请“移植”照片,写清楚个人信息
% 	    \item 请自行完成SPOC课程中的测试
% 	    \item 第二次单元测试即将发布,成绩记入期末总成绩
% 	  \end{itemize}
% \end{frame}

\begin{frame}{需要注意的问题}
	\linespread{1.5}
	  \begin{itemize}%[<+-|alert@+>]
	    \item L'Hospital法则
	    \begin{itemize}
	      \item \it 只能应用于“$\df{\bm{0}}{\bm{0}}$”
	      和“$\df{\bm{\infty}}{\bm{\infty}}$”型
	      \item \it 及时使用无穷小代换进行简化
	      \item \it 不正规的符号:\b 
	      $\xlongequal{\footnotesize\mbox{“L”}}$、
	      $\xlongrightarrow{\footnotesize\mbox{“L'Hospital法则”}}$、
	      $\df{\bm{0}}{\bm{0}}$、$\df{\bm{\infty}}{\bm{\infty}}$
	    \end{itemize}
	    \item Taylor公式
	    \begin{itemize}
	      \item \it Taylor多项式不包含余项
	      \item \it 合并同次幂的系数
	      \item \it 尽量按照幂次由低到高排列,最后写余项
	    \end{itemize}
	  \end{itemize}
\end{frame}

\begin{frame}{出现的问题}
	\linespread{1.5}
	  \begin{itemize}%[<+-|alert@+>]
	    \item 奇特的无穷小代换
	    \begin{itemize}
	      \item \it\b $\ln\tan7x\sim\ln7x$
	      \item \it\b $\ln\tan7x\sim\tan7x$
	      \item \it\b $(1+x)^{\frac1x}\sim e$
	    \end{itemize}
	    \item 奇特的Taylor展开
	    \begin{itemize}
	      \item \it \b$\sum\limits_{k=0}^n
	      \df{(-1)^k(x-3)^k}{(x-1)^{k+1}}+\circ((x-3)^n)$
	      \item \b $\sqrt{\sum\limits_{k=0}^n\left(\df{x^2}4\right)^k
	      +\circ(x^{2n})}$
	    \end{itemize}
	  \end{itemize}
\end{frame}

\section{3.2 L'Hospital法则}

\begin{frame}
	\linespread{1.5}
	\ba{1.计算如下极限:}
	\pause
	
% 	\bigskip
	
	\small 解:
	(1)$\limx{0^+}\df{\ln\tan 7x}{\ln\sin 3x}=\limx{0^+}\df{\df{7\sec^27x}{\tan7x}}{\df{3\sec^23x}{\tan3x}}
	=\limx{0^+}\df{\df7{7x}}{\df3{3x}}=1.$
	
	\pause	
	(2)$\limx{+\infty}\df{\ln\left(1+\df1x\right)}{\arctan x}
	=\df{\limx{+\infty}\ln\left(1+\df1x\right)}{\limx{+\infty}\arctan
	x}=\df{0}{\frac{\pi}2}=0.$
	
	\ba{注:$0/1$型,不能使用L'Hospital法则!}
	
	\pause	
	(3)$\limx{0^+}x^{\sin x}=\exp\left(\limx{0^+}\sin x\ln x\right)
	=\exp\left(\limx{0^+}x\ln x\right)$
	
	\quad$=\exp\left(\limx{0^+}\ln x^x\right)
	=1.$
	
	\pause
	(4)$\limx{\infty}x^2e^{\frac1{x^2}}$极限不存在。

	\ba{注:$\infty\cdot 1$型,不能使用L'Hospital法则!}
\end{frame}

\begin{frame}
	\linespread{1.5}
% 	\ba{1.计算如下极限:}
% 	\pause
	
% 	\bigskip
	
 	\small 	
	(5)$\limx{\infty}\df{x^2-\cos x}{x^2+\sin x}
	=\limx{\infty}\df{1-\frac{\cos x}{x^2}}{1+\frac{\sin x}{x^2}}=1.$
	
	\ba{注:第二个等号不能使用L'Hospital法则!}
	
	\pause	
	(6)$\limx{0}\df{x^2-\sin x}{x^2+\cos x}
	=\df{\limx{0}x^2-\limx{0}\sin x}{\limx{0}x^2+\limx{0}\cos x}=0.$
	
	\ba{注:$0/1$型,不能使用L'Hospital法则!}
	
	\pause
	(7)$\limx{0}\df{\ln(1+x^2)}{\sec x-\cos x}
	=\limx{0}\df{x^2\cos x}{1-\cos^2 x}
	=\limx{0}\df{x^2}{\sin^2 x}=1.$
	
	\pause
	(8)$\limx{0}\df{(1+x)^{1/x}-e}{x}
	=e\limx{0}\df{e^{1/x\ln(1+x)-1}-1}{x}$
	
	\quad$=e\limx{0}\df{1/x\ln(1+x)-1}{x}
	=e\limx{0}\df{\ln(1+x)-x}{x^2}$
	
	\quad$=e\limx{0}\df{\frac1{1+x}-1}{2x}
	=\frac e2\limx{0}\df{-1}{1+x}=-\frac e2.$
\end{frame}

\begin{frame}
	\linespread{1.5}
% 	\ba{1.计算如下极限:}
% 	\pause
	
% 	\bigskip
	
 	\small 
	(9)$\limx{1}(1-x^2)\tan\df{\pi}{2}x
	=2\limx{1}(1-x)\tan\df{\pi}{2}x
	$
	
	$\quad=2\lim\limits_{y\to0}y\tan\left[\df{\pi}2(1-y)\right]
	=2\lim\limits_{y\to0}\df y{\tan\left(\df{\pi}2y\right)}
	=\df4{\pi}.$
	
	\pause	
	(10)$\limx{0}\left(\df 1{x^2}-\df 1{x\tan x}\right)
	=\limx{0}\df {\tan x-x}{x^2\tan x}
	=\limx{0}\df {\tan x-x}{x^3}$
	
	$\quad
	=\limx{0}\df {\sec^2 x-1}{3x^2}
	\limx{0}\df {\tan^2 x}{3x^2}=\df13.$
	
	\pause	
	(11)$\limx{0}(x^2+2^x)^{1/x}
	=\exp\left[\limx{0}\df{\ln(x^2+2^x)}x\right]
	$
	
	$\quad=\exp\left[\limx{0}\df{2x+2^x\ln 2}{x^2+2^x}\right]
	=\exp\left[\limx{0}\df{2+2^x\ln^2 2}{2x+2^x\ln2}\right]$
	
	$\quad
	=\exp\left[\limx{0}\df{2^x\ln^3 2}{2+2^x\ln^22}\right]
	=\exp\left[\limx{0}\df{\ln^3 2}{\frac2{2^x}+\ln^22}\right]
	$
	
	$\quad
	=e^{\ln 2}=2.$
\end{frame}

\begin{frame}
	\linespread{1.5}
% 	\ba{1.计算如下极限:}
% 	\pause
	
% 	\bigskip
	
 	\small 
	(12)$\limx0\df{e^{\tan x}-e^x}{\tan x-x}
	=\limx0\df{e^{\tan x-x}-1}{\tan x-x}\limx0e^x
	=\limx0\df{\tan x-x}{\tan x-x}=1.$
	
	\bigskip
	\pause
	(13)$\limx{0}\left(\cot x-\df 1x\right)
	=\limx{0}\df{x-\tan x}{x\tan x}
	=\limx{0}\df{x-\tan x}{x^2}$
	
	$\hspace{1cm}
	=\limx{0}\df{1-\sec^2 x}{2x}
	=\limx{0}\df{-\tan^2 x}{2x}=0$
\end{frame}

\begin{frame}
	\linespread{1.5}
% 	\ba{1.计算如下极限:}
% 	\pause
	
% 	\bigskip
	
 	\small
	(14)\it $\limx{\infty}\left[\df1n\left(a_1^{\frac1x}+a_2^{\frac1x}
	+\ldots+a_n^{\frac1x}\right)\right]^{nx}$,其中$a_1,a_2,\ldots,a_n>0$
	
	记$A_n=\df{\left(a_1^{\frac1x}-1\right)+\left(a_2^{\frac1x}-1\right)
	+\ldots+\left(a_n^{\frac1x}-1\right)}n$,显然$\limx{\infty}A_n=0$。
	
	\pause
	\begin{align*}
		&\mbox{原式}
		=\limx{\infty}(1+A_n)^{\frac{1}{A_n}A_nnx}
		=\left[\limx{\infty}(1+A_n)^{\frac{1}{A_n}}\right]
		^{\limx{\infty}A_nnx}\\
		&=\exp\left\{\limx{\infty}x\left[\left(a_1^{\frac1x}-1\right)
		+\left(a_2^{\frac1x}-1\right)+\ldots+\left(a_n^{\frac1x}-1\right)\right]\right\}\\
		&=\exp\left[\limx{\infty}x\left(a_1^{\frac1x}-1\right)
		+\limx{\infty}x\left(a_2^{\frac1x}-1\right)
		+\ldots+\limx{\infty}x\left(a_n^{\frac1x}-1\right)\right]\\
		&=\exp\left(\limx{\infty}\df{x\ln a_1}{x}
		+\limx{\infty}\df{x\ln a_2}{x}
		+\ldots+\limx{\infty}\df{x\ln a_n}{x}\right)\\
		&=a_1a_2\ldots a_n.
	\end{align*}
\end{frame}

\begin{frame}
	\linespread{1.5}
	\ba{2.设$x\to 0$时,$x-\sin ax$与$x^2\ln(1-bx)$为等价无穷小,求$a,b$的值。
	}\pause
	
	\bigskip
	
	\small 解:\it 
	$x\to 0$时,$x^2\ln(1-bx)\sim-bx^3$,故必有
	$$\b 1=\limx0\df{x-\sin ax}{-bx^3}
	=\limx0\df{1-a\cos ax}{-3bx^2},$$
	右侧极限存在当且仅当$a=1$,进而
	$$1=\limx0\df{x-\sin ax}{-bx^3}
	=\limx0\df{1-\cos x}{-3bx^2}
	=\limx0\df{\frac{x^2}2}{-3bx^2}=-\df1{6b},$$
	故$b=-\df16$。\hfill$\Box$
	
	\pause\ba{正确的作法应使用Taylor公式!}
\end{frame}

\begin{frame}
	\linespread{1.5}
	\ba{2.设$x\to 0$时,$x-\sin ax$与$x^2\ln(1-bx)$为等价无穷小,求$a,b$的值。
	}\pause
	
	\bigskip
	
	\small 正解:\it 
	$x\to 0$时,$x^2\ln(1-bx)\sim-bx^3$。又由Taylor公式,
	$$x-\sin ax=x-ax+\df{a^3x^3}6+\circ(x^3),$$
	故
	\begin{align*}
		1&=\limx0\df{x-\sin ax}{-bx^3}
		=\limx0\df{(1-a)x+\df{a^3x^3}6+\circ(x^3)}{-bx^3}\\
		&=\limx0\df{1-a}{-bx^2}-\df{a^3}{6b},
	\end{align*}
	上式最后的极限存在当且仅当$1-a=0$,进而可知必有
	$a=1,b=-\df16$。\hfill$\Box$
\end{frame}

\section{3.2 Taylor公式}

\begin{frame}
	\linespread{1.5}
	\ba{1.求$f(x)=x^4+x^3-2x-4$在$x=0$和$x=2$处的三阶Taylor多项式。}\pause

	\bigskip
	
	\small 解: \it 
	$f(x)$在$x=0$处的三阶Taylor多项式为
	$$P^{[0]}_3(x)=x^3-2x-4.$$
	\pause 又
	\begin{align*}
		f(x)&=[(x-2)+2]^4+[(x-2)+2]^3-2[(x-2)+2]-4\\
		&=(x-2)^4+9(x-2)^3+30(x-2)^2+42(x-2)+16,
	\end{align*}
	$f(x)$在$x=2$处的三阶Taylor多项式为
	$$P^{[2]}_3(x)=9(x-2)^3+30(x-2)^2+42(x-2)+16.$$
	\hfill$\Box$
\end{frame}

\begin{frame}
	\linespread{1.5}
	\ba{2.求函数$f(x)=\df1{x-1}$按$(x-3)$的幂展开的带有Peano余项的
	$n$阶Taylor公式。}\pause
	
	\bigskip
	
	\small 解:\it
	\begin{align*}
		f(x)&=\df1{2+(x-3)}=\df12\cdot\df1{1+\frac{x-3}2}\\
		&=\df12\left[\sum\limits_{k=0}^n\left(-\df{x-3}2\right)^k
		+\circ\left(\left(-\df{x-3}2\right)^n\right)\right]\\
		&=\sum\limits_{k=0}^n\df{(-1)^k(x-3)^k}{2^{k+1}}+\circ((x-3)^n).
	\end{align*}
	\hfill$\Box$
\end{frame}

\begin{frame}
	\linespread{1.5}
	\ba{3.求如下函数带Peano余项的Maclaurin公式}\pause
	
	\bigskip
	
	\small 解:(1)
	\begin{align*}
		\ln(2+x^2)&=\ln2+\ln\left(1+\df{x^2}2\right)\\
		&=\ln2+\sum\limits_{k=1}^n\df{(-1)^{k-1}\left(\frac{x^2}{2}\right)^k}{k}
		+\circ\left(\left(\df{x^2}{2}\right)^n\right)\\
		&=\ln2+\sum\limits_{k=1}^n\df{(-1)^{k-1}x^{2k}}{k2^k}
		+\circ(x^{2n}).
	\end{align*}
\end{frame}

\begin{frame}
	\linespread{1.5}
% 	\ba{3.求如下函数带Peano余项的Maclaurin公式}\pause
	
% 	\bigskip
	
	\small (2)
	$
		e^{\frac{x^2}2}
		=\sum\limits_{k=0}^n\df{\left(\frac{x^2}2\right)^k}{k!}
		+\circ\left(\left(\df{x^2}2\right)^n\right)
		=\sum\limits_{k=0}^n\df{x^{2k}}{k!2^k}
		+\circ\left(x^{2n}\right).
	$
	\bigskip
	
	\pause
	(3)
	$
		x\cos x^2
		=x\left[\sum\limits_{k=0}^n\df{(-1)^kx^{4k}}{(2k)!}
		+\circ(x^{2n})\right]$
		
		\quad $
		=\sum\limits_{k=0}^n\df{(-1)^kx^{4k+1}}{(2k)!}
		+\circ(x^{4n+1}).
	$
	\bigskip
	
	\pause
	(4)
	$
		\df{1+x^2}{1-x^2}
		=-1+\df2{1-x^2}
		=-1+2\left[\sum\limits_{k=0}^n(x^2)^k-\circ(x^{2n})\right]$
		
		\quad$
		=-1+\sum\limits_{k=0}^n2x^{2k}+\circ(x^{2n}).
	$
\end{frame}

\begin{frame}
	\linespread{1.5}
% 	\ba{3.求如下函数带Peano余项的Maclaurin公式}\pause
	
% 	\bigskip
	
	\small (5)
	\begin{align*}
		\df1{\sqrt{4-x^2}}
		&=\df12\left(1-\frac{x^2}4\right)^{-\frac12}\\
		&=\df12\left[\sum\limits_{k=0}^n\left(\begin{array}{c}
			-\frac12 \\ k
		\end{array}\right)\left(-\df{x^2}{4}\right)^k
		+\circ\left(\left(-\df{x^2}{4}\right)^n\right)
		\right]\\
		&=\sum\limits_{k=0}^n\df{(2k+1)!!}{k!8^k2}x^{2k}
		+\circ(x^{2n}).
	\end{align*}
	
	\pause
	(6)
	\begin{align*}
		&\sin^2x-x^3=\df12-\df12\cos2x-x^3\\
% 		&=\df12-\df12\left[1-\df{(2x)^2}{2!}+\df{(2x)^4}{4!}
% 		+\ldots+\df{(-1)^k(2x)^{2k}}{(2k)!}+\circ((2x)^{2n})\right]-x^3\\
		&=x^2-x^3-\df{2^3x^4}{4!}
		+\ldots-\df{(-1)^k2^{2k-1}x^{2k}}{(2k)!}+\circ(x^{2n}).
	\end{align*}
\end{frame}

\begin{frame}
	\linespread{1.5}
% 	\ba{3.求如下函数带Peano余项的Maclaurin公式}\pause
	
% 	\bigskip
	
	\small (7)\it 注意到$(1+x)e^x=(xe^x)'$,而
	$$xe^x=x\sumn\df{x^n}{n!}=\sumn\df{x^{n+1}}{n!},$$
	故
	$$(1+x)e^x=\left[\sumn\df{x^{n+1}}{n!}\right]'
	=\sumn\df{(x^{n+1})'}{n!}
	=\sumn\df{(n+1)x^n}{n!},$$
	于是所求带Peano余项的Maclaurin公式为
	$$(1+x)e^x=\sum\limits_{k=0}^n\df{(k+1)x^k}{k!}+\circ(x^{n+1}).$$
	\hfill$\Box$
\end{frame}