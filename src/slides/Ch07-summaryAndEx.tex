% !Mode:: "TeX:UTF-8"

\titlepage

\begin{frame}
	\frametitle{知识点回顾}
	\linespread{1.5}
	\it
	  \begin{itemize}
 	    \item 微分方程及其解(通解、特解)的概念
	    \item 一阶微分方程的解法与常用技巧
	    \item 可降阶的二(高)阶微分方程
	    \item 二阶线性微分方程解的结构
	    \item 求解二阶常系数齐次线性微分方程的特征方程法
	    \item 求解特定二阶常系数非齐次线性微分方程的待定系数法
	  \end{itemize}
\end{frame}

\section{补充例题}

\subsection{二阶线性微分方程}

% \section{问题讨论}

% \begin{frame}{问题讨论}
% 	\linespread{1.5}
% 	\alert{问:}已知$n$阶线性微分方程的$n$个解,
% 	能否写出这个微分方程及其通解?\pause\\[1ex]
% 	
% 	\alert{答:}{\it 不一定。}\pause 除非{\it\b 
% 	这$n$个解恰为$n$阶齐次线性微分方程的线性无关的特解。} \pause 
% 	
% 	\bigskip
% 	\alert{问:}适当确定微分方程通解中的参数值,可以得到其任意的特解?\pause \\[1ex]
% 	
% 	\alert{答:}{\it 错!有些特解无法用统一的通解形式来表达。}\pause {\it 例如:}{\b $y'=\sin x\cos^2y$}.
% \end{frame}

\begin{frame}{问答题}
	\linespread{1.2}
	\alert{问:}$y_1=(x-1)^2$和$y_2=(x+1)^2$都是方程
	$$(x-1)^2y''-2xy'+2y=0,$$
	和
	$$2yy''-(y')^2=0$$
	的解。但二者的线性组合
	$$y=C_1(x-1)^2+C_2(x+1)^2,\;(C_1,C_2\in\mathbb{R})$$
	却仅能满足前一个方程,为什么?\pause 
	
	\alert{答:}{\it\b 第二个方程不是线性方程!}
\end{frame}

\begin{frame}{填空}
	\linespread{1.5}
	\alert{例:}$y''+4y'+4y=1$的通解为
	\underline{\uncover<2->{\;\b{$(C_1+C_2x)e^{-2x}+1/4$}}\;}.\\[1em]
	
	\alert{例:}设$e^x(C_1\cos x+C_2\sin x)$为首项系数为$1$的某二阶常系数
	齐次线性微分方程的通解,则该微分方程为
	\underline{\uncover<3->{\;\b{$y''-2y'+2y=0$}}\;}.\\[1em]
	
	\alert{例:}设$\cos x$与$xe^x$分别为某$n$阶常系数齐次线性微分方程的两个解,
	则最小的$n=$\underline{\uncover<4->{\;\b{$4$}}\;},相应的首项
	系数为$1$的方程为\underline{\uncover<5->{\;\b{$
	y^{(4)}-2y^{(3)}+2y''-2y'+y=0$}\;}}
	
% 	方程$xy''-2xy'+2y=x\ln x$的通解为
% 	为\underline{\uncover<6->{\;\b{$
% 	C_1x+C_2x^2-\left(\df12\ln^2x+\ln x\right)x$}\;}}
\end{frame}

\begin{frame}
	\linespread{1.2}
	\pause\alert{提示:}\it\b 
	特征方程的复根总是成对出现,故$\cos x$若为方程的解,$\sin x$也必为其解,
	对应的特征根为$r=\pm i$。
	
	又若$xe^x$为齐次线性微分方程的特解,则$r=1$必为对应特征方程的一个重根,
	进而可知$e^x$也是方程的解。
	
	综上,方程的特征根为$r=\pm i$和$r=1$(二重),于是特征方程为
	$$(r+i)(r-i)(r-1)^2=r^4-2r^3+2r^2-2r+1=0,$$
	由此易得原方程。
\end{frame}

\begin{frame}{选择}
	\linespread{1.3}
	\alert{例:}设$y_1(x),y_2(x),y_3(x)$为方程
	$$y''+p(x)y'+q(x)y=f(x)$$
	的三个线性无关的解,$C_1,C_2$为任意常数,则该非齐次线性微分方程的通解为
	(\underline{\uncover<2->{\;\b{C}}\;})
	\begin{enumerate}[(A)]
	  \item $(C_1+C_2)y_1+(C_2-C_1)y_2+(1-C_2)y_3$
	  \item $(C_1+C_2)y_1+(C_2-C_1)y_2+(C_1-C_2)y_3$
	  \item $C_1y_1+(C_2-C_1)y_2+(1-C_2)y_3$
	  \item $C_1y_1+(C_2-C_1)y_2+(C_1-C_2)y_3$
	\end{enumerate}
\end{frame}

\begin{frame}{选择}
	\linespread{1.5}
	\alert{例:}下列可能为方程$y''+4y=e^{3x}+x\sin 2x$的特解的是
	(\underline{\uncover<2->{\;\b{A}}\;})
	\begin{enumerate}[(A)]
	  \item $Ae^{3x}+x[(Bx+C)\cos2x+(Dx+E)\sin2x]$
	  \item $Ae^{3x}+(Bx+C)\cos2x+(Dx+E)\sin2x$
	  \item $Axe^{3x}+x[(Bx+C)\cos2x+(Dx+E)\sin2x]$
	  \item $Axe^{3x}+(Bx+C)\cos2x+(Dx+E)\sin2x$
	\end{enumerate}
	\pause\alert{提示:}\it\b 利用叠加原理,分别构造$e^3x$和$x\sin2x$
	对应的特解,再相加。
\end{frame}

\begin{frame}{选择}
	\linespread{1.3}
	\alert{例:}已知$xe^x+e^{2x}$和$xe^x+e^{-x}$是二阶常系数非齐次线性微分方程的两个解,
	则此方程为
	(\underline{\uncover<2->{\;\b{}}\;})
	\begin{enumerate}[(A)]
	  \item $y''-2y'+y=e^{2x}$
	  \item $y''-y'-2y=xe^{x}$
	  \item $y''-y'-2y=e^x-2xe^{x}$
	  \item $y''-y=e^{2x}$
	\end{enumerate}
	\pause\alert{提示:}\it\b 显然$e^{2x}-e^{-x}$为对应齐次方程的解,其中
	的两个函数线性无关,故分别为齐次方程的特解,进而特征根$r_1=2,r_2=-1$。
\end{frame}

\begin{frame}{选择}
	\linespread{1.3}
	\alert{例:}方程$y''+by'+y=0$的每个解都在$x>0$上有界,则实数$b$的取值范围是
	(\underline{\uncover<2->{\;\b{A}}\;})
	\begin{enumerate}[(A)]
	  \item $[0,+\infty)$
	  \item $(-\infty,0]$
	  \item $(-\infty,2)$
	  \item $(2,+\infty)$
	\end{enumerate}
	\pause\alert{提示:}\it\b 对照二阶常系数齐次线性方程的各种解的形式,可得
	当通解中的指数函数$e^{\lambda x}$满足$\lambda<0$时,即有所有的解有界。
	利用韦达定理可推出$b<0$。又$b=0$时,解方程可知也满足要求。
\end{frame}

\begin{frame}{解答题}
	\linespread{1.2}
	\alert{例:}求方程$y''+2y'+2y=2e^{-x}\cos^2\df x2$的通解.
	
	\pause\alert{提示:}\it\b 
	$$2e^{-x}\cos^2\df x2=e^{-x}+e^{-x}\cos x,$$
	利用叠加原理分别求解两个常系数非齐次线性微分方程。
\end{frame}

\begin{frame}{解答题}
	\linespread{1.5}
	\alert{例:}设二阶常系数线性微分方程
	$y''+\alpha y'+\beta y=\gamma e^x$
	的一个解为$y=e^{2x}+(1+x)e^x$,试确定其中的常数$\alpha,\beta,\gamma$.
	
	\pause\alert{提示:}\it\b 由解的结构,$e^{2x},e^x,xe^x$中有两个为齐次方程
	的特解,另一个为非齐次方程的特解。若$xe^x$为齐次方程的解,则$e^x$也是,此时非齐次
	方程的特解形如$Ax^2e^x$,显然$e^{2x}$不具有这种形式。故$xe^x$必为非齐次方程的特解,
	从而$e^{2x},e^x$为齐次方程的解。由此可得到对应的特征方程,进而
	进一步可解出$\alpha=-3,\beta=2,\gamma=-1$。
\end{frame}

\begin{frame}{解答题}
	\linespread{1.2}
	\alert{例:}利用变换$x=e^t$求解如下方程
	$$x^2\df{\d^2y}{\d x^2}+3x\df{\d y}{\d x}+5y=16x\ln x.$$
	
	\pause\alert{提示:}\it\b 
	$$y''_{tt}+2y'_t+5y=16te^{t}$$
	\pause \ba{请自行阅读教材第七章第九节“Euler方程”}
\end{frame}

\begin{frame}{解答题}
	\linespread{1.2}
	\alert{例:}令$t=\tan x$,将方程
	$$y''_{xx}\cos^4x+2\cos^2x(1-\sin x\cos x)y'_x+y=e^{-\tan x}$$
	变换为$y$关于$t$的微分方程,并求其通解。
	
	\pause\alert{提示:}\it\b 
	$$y''_{tt}+2y'_t+y=e^{-t}$$
	$$y=\left(C_1+C_2\tan x+\df12\tan^2x\right)e^{-\tan x}$$
\end{frame}

\begin{frame}{解答题}
	\linespread{1.2}
	\alert{课后作业:}试将$x=x(y)$所满足的微分方程
    $$\df{\d^2x}{\d y^2}+(y+\sin x)\left(\df{\d x}{\d y}\right)^3=0$$
    化为$y=y(x)$所满足的微分方程;
	
	\pause\alert{提示:}\it\b 
	$$\df{\d^2x}{\d y^2}=\df{\d x'}{\d y}
	=\df{\d(1/y')}{\d y}=
	\df{\df{\d\frac1{y'}}{\d x}}{\df{\d y}{\d x}}=-\df{y''}{(y')^3},$$
	原方程最终化为
	$y''-y=\sin x.$
\end{frame}

% \begin{frame}{选择}
% 	\linespread{1.3}
% 	5.设$y(x)$满足$x\d y+(x-2y)\d x=0$,且曲线$y=y(x)$与直线$x=1$
% 	及$x$轴所围平面图形绕$x$轴旋转所得旋转体的体积最小,则$y(x)=$
% 	(\underline{\uncover<2->{\;\b{C}}\;})
% 	\begin{enumerate}[(A)]
% 	  \item $x-\df14x^2$
% 	  \item $x+\df54x^2$
% 	  \item $x-\df54x^2$
% 	  \item $x+\df14x^2$
% 	\end{enumerate}
% \end{frame}

% \begin{frame}{解方程}
% 	\linespread{1.5}
% 	\begin{enumerate}
% 	  \item $xy'\ln x+y=\ln x$.\hfill \b$t=\ln x$
% 	\end{enumerate}
% \end{frame}

% \begin{frame}{解答题}
% 	\linespread{1.2}
% 	1.设$f(x)$为连续函数,且
% 	$$f(x)=e^{-x}+\dint_0^xf(t)\d t,$$
% 	求$f(x)$.
% 	
% 	\pause\alert{提示:}\it\b  \alert{积分方程通常自带初值条件!}
% 	\pause 在已知等式中令$x=0$,可得$f(0)=1$.\pause
% 	$$f'(x)-f(x)=-e^{-x}\quad\Rightarrow\quad 
% 	f(x)=\df12(e^x+e^{-x}).$$
% \end{frame}



% \begin{frame}{解答题}
% 	\linespread{1.2}
% 	3.函数$y(x)\;(x\geq 0)$二阶可导,$y'(x)>0$,
% 	$y(0)=1$,过其上任一点$(x,y)$作曲线的切线和至$x$轴的垂线,该两直线
% 	与$x$轴所围成的三角形面积记为$S_1(x)$,又区间$[0,\alert{x}]$
% 	(\alert{\it 此处教材印刷错误!})上以$y(x)$为曲边
% 	的曲边梯形面积记为$S_2(x)$。已知$2S_1-S_2=1$,求$y(x)$。
% 	
% 	\pause\alert{提示:}\it\b   
% 	$$\df{y^2}{y'}-\dint_0^xy(t)\d t=1\quad\Rightarrow\quad 
% 	yy''=(y')^2,\;y'(0)=1.$$
% 	\pause 结合$y(0)=1$,解得$y=e^x$.
% \end{frame}



% \begin{frame}{解答题}
% 	\linespread{1.2}
% 	某学生将乘积的导数公式错误地记作$(fg)'=f'g'$,然而在一次求导时
% 	居然得到了正确的结果。目前知道他使用的$f(x)=e^{x^2},(x>1/2)$,
% 	问他用到的$g(x)$可能是什么?
% 	
% 	\pause\alert{提示:}\it\b  
% 	$$\left(e^{x^2}g\right)'=\left(e^{x^2}\right)'g'$$
% 	从而$(2x-1)g'=2xg$,解得
% 	$$g=Ce^x\sqrt{2x-1}$$
% \end{frame}



\begin{frame}{解答题}
	\linespread{1.2}
	\alert{例:}求幂级数$\sumn[0]\df{x^{3n}}{(3n)!}$的和函数。
	
	\pause\alert{提示:}\it\b 
	收敛域为$(-\infty,+\infty)$,和函数满足如下的初值问题
	$$S^{(3)}(x)=S(x),\quad,S(0)=1,S'(0)=0,S''(0)=0.$$
	通解
	$$S(x)=C_1e^x+e^{-\frac x2}\left(C_2\cos\df{\sqrt3}2x
	+C_2\sin\df{\sqrt3}2x\right).$$
\end{frame}

% \begin{frame}{解答题}
% 	\linespread{1.2}
% 	\alert{(2003考研)} 设$y(x)$在$\mathbb{R}$上具有二阶连续导数,
% 	$y'\ne 0$,$x=x(y)$为其反函数。
% 	\begin{enumerate}
% 	  \item 试将$x=x(y)$所满足的微分方程
% 	  $$\df{\d^2x}{\d y^2}+(y+\sin x)\left(\df{\d x}{\d y}\right)^3=0$$
% 	  变换为$y=y(x)$所满足的微分方程;
% 	  \item 求变换后的微分方程满足初始条件$y(0)=0$和$y'(0)=1.5$的解。
% 	\end{enumerate}
% 	
% 	\pause\alert{提示:}\it\b 
% 	$y''-y=\sin x\quad\Rightarrow\quad y=e^x-e^{-x}-\df12\sin x $
% \end{frame}

\subsection{杂例}

\begin{frame}{解答题}
	\linespread{1.2}
	\alert{例:}设$f(x)$为连续函数,且
	$$f(x)=e^{2x}+\dint_0^xtf(x-t)\d t,$$
	求$f(x)$.
	
	\pause\alert{提示:}\it\b \ba{积分方程通常自带初值条件!}
	\pause 本例中,可得$f(0)=1,f'(0)=2$\pause  
	$$f''(x)-f(x)=4e^{2x}\quad\Rightarrow\quad 
	f(x)=-\df12e^x+\df16e^{-x}+\df43e^{2x}.$$
\end{frame}

\begin{frame}{解答题}
	\linespread{1.2}
	\alert{例:}设$f(x)$在$(-\infty,+\infty)$上处处可导,其反函数为$g(x)$,且
	$$\dint_0^{f(x)}g(t)\d t+\dint_0^xf(t)\d t=xe^x-e^x+1,$$
	求$f(x)$.
	
	\pause\alert{提示:}\it\b 两边求导,可得
	$$f'(x)+f(x)=xe^x,$$
	通解$y=Ce^{-x}+\df{2x-1}4e^x$。\pause 原
	等式左边即为$1\cdot f(1)-0\cdot f(0)$,故有初值条件$f(1)=1$。
\end{frame}

\begin{frame}{解答题}
	\linespread{1.2}
	\alert{例:}设对任意$x,y\in\mathbb{R}$
	$$f(x+y)=f(x)e^y+f(y)e^x,$$
	$f'(0)=a\ne 0$,求$f(x)$.
	
	\pause\alert{提示:}\it\b 必须用定义计算$f'(x)$,
	$$\lim\limits_{\Delta x\to 0}\df{f(x+\Delta x)-f(x)}{\Delta x}
	=f(x)+ae^x.$$
	\pause 类似题目:\alert{\bf 辅导书(下)-P256-例5}
\end{frame}

\begin{frame}{选择}
	\linespread{1.3}
	\alert{例:}设$y(x)$为方程$y''+py'+qy=e^{3x}$满足初始条件$y(0)=y'(0)=0$
	的解,则$\limx{0}\df{\ln(1+x^2)}{y(x)}$
	(\underline{\uncover<2->{\;\b{C}}\;})
	\begin{enumerate}[(A)]
	  \item 不存在
	  \item 等于$1$
	  \item 等于$2$
	  \item 等于$3$
	\end{enumerate}
	\pause\alert{提示:}\it\b 由已知可得$y''(0)=1$,利用L'Hospital法则
	求极限即可。
\end{frame}

\begin{frame}{选择}
	\linespread{1.3}
	\alert{例:}设$y=f(x)$为方程$y''-2y'+4y=0$
	的一个解,若$f(x_0)>0,f'(x_0)=0$,
	则函数$f(x)$在$x_0$
	(\underline{\uncover<2->{\;\b{A}}\;})
	\begin{enumerate}[(A)]
	  \item 取极大值
	  \item 取极小值
	  \item 的某个领域内单调增加
	  \item 的某个领域内单调减少
	\end{enumerate}
	\pause\alert{提示:}\it\b 由已知可得$f''(x_0)<0$,利用极值的判定条件即得结果。
\end{frame}

\subsection{微分方程的应用}

\begin{frame}{应用题}
	\linespread{1.4}
	\alert{例:}已知某凹曲线任一点处的曲率为$\df1{2y^2\cos\alpha}$,其中
	$\alpha$为该点处的切线倾角($\cos\alpha>0$),且曲线在
	点$(1,1)$处的切线是水平的,求该曲线的方程。
	
	\pause\alert{提示:}\it\b $\cos\alpha=\df1{\sqrt{1+(y')^2}}>0$,
	方程为
	$$2y^2y''=[1+(y')^2]^2,$$
	初值条件$y(1)=1$,最后解得
	$$4y=(x-1)^2+4$$
\end{frame}

\begin{frame}{应用题}
	\linespread{1.2}
	\alert{例:}一根挂在钉子上的链条,最初两端距离钉子
	分别为$8$m和$12$m,如不计钉子
	对链条产生的摩擦力,求链条从钉子上完全滑落所需的时间。
	
	\pause\alert{提示:}\it\b 设$x$为较长一端端点据钉子的距离
	$$\left\{\begin{array}{l}
		x''-\df g{10}x=-g\\
		x(0)=12\\
		x'(0)=0
	\end{array}\right.$$
% 	\pause
% 	\alert{思考:}若摩擦力等于$1$m长的链条的重量,模型又是怎样的?
\end{frame}

\begin{frame}{应用题}
	\linespread{1.2}
	\alert{例:}已知某车间容积$V$,其空气中CO$_2$的密度为$\rho_1$,现以CO$_2$浓度$\rho_2(<<\rho_1)$的
	新鲜空气输入,问每分钟应输入多少才能在$T$分钟后使车间中CO$_2$的含量不超过$\rho_0$。
	(注:假设新注入的空气能够与原有空气立即混合达到均匀,且空气不会被压缩。)

	\pause\alert{提示:}\it\b 设$t$分钟时的$CO_2$含量为$C(t)$,
	$$C'=\rho_2V_1-C\df{V_1}{V},C(0)=\rho_1V$$
\end{frame}