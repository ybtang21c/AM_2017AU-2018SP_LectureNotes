% !Mode:: "TeX:UTF-8"

\begin{frame}{《高等数学(下)》知识要点}
	\linespread{1.5}
% 	\begin{itemize}
% 	  \item {\bf 主要内容:}
	  \begin{enumerate}
		\item 常微分方程
		\item 空间解析几何
		\item 多元函数微分学
		\item 重积分
		\item 曲线与曲面积分
		\item 幂级数与Fourier级数
	  \end{enumerate}
% 	  \item {\bf  课后作业:}
% 	  \begin{itemize}
% 	    \item {\b 习题13.2:1(1,4),3(1),5,6(3)}
% 	  \end{itemize}
% 	\end{itemize}
\end{frame}

\begin{frame}{一、常微分方程}
	\linespread{1.2} 
	\ba{1、一阶方程}{\color{yellow!90!gray}\FiveStar} 
	
	{\b{\bf 关键词:} 变量分离、 常数变异、 变量替换、 全微分} 
	\begin{itemize}
	  \item 可分离变量方程: \ba{注意分母为零时对应的特解} 
	  \item \ba{齐次方程}: {\it 令\ba{$z=\df yx$},化为可分离变量方程}
	  $$
% 	  y'=\df{2x-5y+3}{2x+4y-6}, \quad 
	  xy'=\sqrt{x^2-y^2}+y $$
	  \vspace{-1em}
	  \item 一阶线性齐次/非齐次线性微分方程: \ba{常数变异法} 
	  \item Bernoulli方程: $y'+P(x)y=Q(x)y^n\,(n\ne 0,1)$ 
% 	  $$y'-\df y{2x}=\df{x^2}{2y}$$
		\ba{令$z=y^{1-n}$}
	\end{itemize}
\end{frame}

\begin{frame}
	\linespread{1.2}
	{\bf 解一阶方程的常用技巧:} 
	\begin{itemize}
	  \item {\b 变量替换:} \ba{“谁不好处理就代换谁!”}
	  $$xy'+y=y(\ln x+\ln y), \quad y'=(x+y)^2 $$
	  \vspace{-2em}
% 	  $$2x\ln xdy+y(y^2\ln x-1)dx=0,\quad y'=(x+y)^2$$
	  \item {\b 上下颠倒:} {\it 将$x$视为$y$的函数}
	  $$y'=\df 1{x+y^2}, \quad y'=\df{x}{x^2+y^2} $$
	  \vspace{-1em}
	  \item {\b 全微分方程:} {\ba{全微分的判定条件!}}
	  $$y'=-\df{\sin x+y}{x+\cos y}, \quad
	  xdx+ydy+\df{y\d y-x\d x}{x^2+y^2}=0 $$
	  \vspace{-1em}
	  \item {\color{gray!50!white} 幂级数法:\it 不推荐!} 
	\end{itemize}
\end{frame}

\begin{frame}
	\linespread{1.2}
	{\bf 2、二阶方程} 
	
	{\b{\bf 关键词:} 可降阶的方程、线性方程解的结构、叠加原理} 
	\begin{itemize}
	  \item \ba{$y''=f(x,y')$: {\it 令$y'=p(x)$,则$y''_{xx}=p'_x$}} 
	  \item \ba{$y''=f(y,y')$: {\it 令$y'=p(y)$,则$y''_{xx}=p'_yp$}}
	  \item \ba{齐次线性方程与非齐次线性方程解的结构:$y=Y+y^*$},{\it 其中:
	  	$Y$为对应齐次方程的通解,$y^*$为齐次方程的任一特解}
  	  \item \ba{叠加原理:习题7.4-11}
	  
	  \item {\color{gray!50!white} Liouville公式:\it 了解}
% 	  :已知二阶齐次线性方程的一个解,求另一个解,令:
% 	  $y_2=u(x)y_1$
	\end{itemize}
\end{frame}

\begin{frame}
	\linespread{1.2}
	\ba{3、二阶常系数线性微分方程}{\color{yellow!90!gray}\FiveStar} 
	
	\begin{itemize}
	  \item {\bf 齐次方程}:{\it 特征方程(根)法求通解},\ba{7.4节-表7.4.1}	  
	  \item {\bf 非齐次方程}:{\it 待定系数法求特解},\ba{教材7.4.2节-2} 
	  \item {\color{gray!50!white} Euler方程:\it 了解}
	\end{itemize}
\end{frame}

\begin{frame}{二、空间解析几何}
	\linespread{1.2} 
	{\bf 1、向量及其运算} 
	
	{\b{\bf 关键词:} 向量运算的几何意义} 
	\begin{itemize}
	  \item 向量运算的几何意义: {\it 投影、正交、面积、体积、共面的条件\ldots}
	  \item 平面方程: {\it 点法式、 一般式、 三点式、 截距式} 
	  \item 直线方程: {\it 点向式、 标准式、 一般式、 两点式及其相互转换} 
	  \item 空间对象的几何关系: {\it 距离、 夹角、 相交、{\color{red}异面}}
	  \item {\b 平面束}:{\it 过给定直线的全部平面}
	\end{itemize}
\end{frame}

\begin{frame}
	\linespread{1.2}
	{\bf 2、空间曲面} 
	
	{\b{\bf 关键词:} 旋转曲面,柱面} 
	\begin{itemize}
	  \item 二次曲面: {\it 根据方程判定曲面类型与几何特征} 
	  \item 曲面的参数方程: {\it 2个自由度$\Rightarrow $2个自变量(参数)} 
	  \item \ba{旋转曲面与柱面}:{\it 利用几何关系确定曲面方程}\\
	  {{\bf 例:}\it 过$(1,0,0),(0,1,1)$的直线绕$z$轴旋转所得曲面} \\
	  {{\bf 例:}\it 轴线平行于$x=y=z$,且经过$xOy$平面上的单位圆的柱面}
	\end{itemize}
\end{frame}

\begin{frame}
	\linespread{1.2}
	{\bf 3、空间曲线} 
	
	{\b{\bf 关键词:} 投影、 参数方程、向量值函数} 
	\begin{itemize}
	  \item 曲线的参数方程$\Leftrightarrow$向量值函数:$\bm{r}(t)$ 
	  \item 曲线的切线方向: $\bm{r}'(t)$,\alert{\it 指向参数
	  $t$增大的方向!}
	  \item \ba{投影法求曲线参数方程}
% 	  : \\
% 	  \centerline{\alert{投影(消元)$\to$参数化$\to$回代}}
	\end{itemize}
\end{frame}

\begin{frame}{三、多元函数微分学}
	\linespread{1.2} 
	{\bf 1、多元函数的连续性与可微性} 
	
	{\b{\bf 关键词:} 链式法则、自由度} 
	\begin{itemize}
	  \item 二重极限存在性的判定
	  \item \ba{多元函数连续、偏导存在、偏导连续和可微的关系} 
	  \item 用定义证明可微
	  \item 复合函数求偏导:\ba{1.链式法则;2.全微分}
	  \item 隐函数求偏导:\ba{自变量个数=变量个数-方程个数}{\color{yellow!90!gray}\FiveStar}
% 	  \item 方向导数与梯度: \ba{$D_uf=\bigtriangledown f\cdot\bm{e_u}$} 
% 	  \item 梯度: \ba{等值线(面)}的法方向 (函数值增加最快方向)
	\end{itemize}
\end{frame}

\begin{frame}
	\linespread{1.2}
	{\bf 2、方向导数、梯度、极值与条件极值} 
	
	{\b{\bf 关键词:} 几何意义、 Lagrange乘子法} 
	\begin{itemize}
	  \item 方向导数与梯度: \ba{$D_uf=\bigtriangledown f\cdot\bm{e_u}$} 
	  \item 梯度: \ba{等值线(面)}的法方向 (函数值增加最快方向)
% 	  {\bf 例:}曲面$F(x,y,z)=c$的法线方向:$\bigtriangledown F$ 
	  \item {\color{gray!50!white} Hessian矩阵与Taylor公式}
% 	  {\bf 例:}$x^4+xy+(1+y)^2$在原点处带Peano余项的一阶及二阶Taylor公式 \\
% 	  {\bf 例:}证明$x,y$很小时:$\ln(1+x)\ln(1+y)\approx xy$ 
	  \item \ba{极值与条件极值}{\color{yellow!90!gray}\mbox{\FiveStar}}
	  \begin{itemize}
	    \item 极值的判定:梯度与Hessian矩阵
	    \item Lagrange乘子法:灵活处理条件极值问题
	    \item 条件极值的几何意义:\ba{$\bigtriangledown f=\bigtriangledown g$}
	  \end{itemize}
	\end{itemize}
\end{frame}

\begin{frame}{四、重积分}
	\linespread{1.2} 
	{\b{\bf 关键词:} 微元、定限、坐标变换} 
	\begin{itemize}
	  \item 定限的次序与积分的次序相反
	  \item \ba{二重积分}{\color{yellow!90!gray}\FiveStar}
	  \begin{itemize}
	    \item \ba{极坐标下的应用}\\
	    \ba{1.$\d\sigma_{xy}=\rho\d\rho\d\theta$\\
	    2.积分区域为圆、扇形和圆环等是优先考虑;\\
	    3.熟练应用不同次序的定限方法}	    
	    \item {\color{gray!50!white} 在任意的坐标变换下计算二重积分}
	  \end{itemize}
	  \item \ba{三重积分}
	  \begin{itemize}
	    \item 按照“2+1”或“1+2”的方式定限
	    \item \ba{被积函数只与一个变量有关时,优先使用“1+2”}{\color{yellow!90!gray}\FiveStar}
	    \item \ba{柱坐标}与球坐标变换
	  \end{itemize}
	\end{itemize}
\end{frame}

\begin{frame}
	\linespread{1.2} 
	\begin{itemize}
	  \item 重积分的应用
	  \begin{itemize}
	    \item 质量
	    \item 质心{\color{yellow!90!gray}\FiveStar}
	    \item 转动惯量{\color{yellow!90!gray}\FiveStar}
	    \item {\color{gray!50!white} 万有引力}
	  \end{itemize}
	\end{itemize}
\end{frame}

\begin{frame}{五、曲线和曲面积分}
	\linespread{1.2}	
	{\b{\bf 关键词:} 弧长微元,面积微元,各种积分的相互转换、对称性} 
	
	\ba{1、对弧长的曲线积分} 
	\begin{itemize}
	  \item {\bf 应用:}{\it 曲线长度、质量、质心、转动惯量、{\color{gray!50!white} 引力} }
	 	$$\alert{\int_{L}f(\bm{x})\d s=\int_a^bf(\bm{r}(t))|\bm{r}'(t)|\d t
	 	{\color{yellow!90!gray}\mbox{\FiveStar}}}$$
	 {\it 其中$\bm{r}(t)$为曲线的参数方程,$a<b$}(\ba{化成定积分后必须确保上限大于下限})
	  \item {平面曲线弧微分的不同形式}\small
		$$\d s=\sqrt{(x'_t)^2+(y'_t)^2}\d t=\sqrt{1+(y')^2}\d x
		=\sqrt{\rho^2+(\rho')^2}\d\theta$$
	\end{itemize}
\end{frame}

\begin{frame}
	\linespread{1.2}
	\ba{2、对坐标的曲线积分} 
	\begin{itemize}
	  \item {\bf 应用:}变力做功、流(通)量、环量 
	  	$$\alert{\dint_L P\d x+Q\d y+Q\d z
	  	=\dint_L\bm{F}\d\bm{s}=\dint_L\bm{F}\cdot\bm{T}\d
	  	s}{\color{yellow!90!gray}\mbox{\FiveStar}}$$
	  \item 在平面向量场中: 
	  \begin{itemize}
	    \item \ba{流量:$\dint_L\bm{v}\cdot\bm{n}\d s=\dint_LP\d y-Q\d x$} 
	    \item \ba{环量:$\dint_L\bm{v}\cdot\bm{T}\d s=\dint_LP\d x+Q\d y$} 
	  \end{itemize} 
	  \item \ba{化成定积分后上、下限与分别为路径的起点和终点}
	  \item 有时可化为对弧长的曲线积分计算
	  \item \ba{如果积分与路径无关,沿着折线计算对坐标的曲线积分可以大大简化计算}
	\end{itemize}
\end{frame}

\begin{frame}
	\linespread{1.2}
	\ba{3、对面积的曲面积分} 
	\begin{itemize}
	  \item {\bf 应用:}曲面面积、质量、质心、转动惯量、引力\small
	 	$$\alert{\iint_{\Sigma}f(x,y,z)\d S
	 	=\iint_{D_{xy}}f(x,y,z(x,y))\sqrt{1+(f'_x)^2+(f'_y)^2}d\sigma_{xy}}
	 	{\color{yellow!90!gray}\mbox{\FiveStar}}$$ 
	  \item \normalsize 根据曲面特点选择合适的投影方向
		$$\d S=\df{\d\sigma_{yz}}{|\cos\alpha|}=
		\df{\d\sigma_{zx}}{|\cos\beta|}
		=\df{\d\sigma_{xy}}{|\cos\gamma|}$$ 
% 	  \item {\bf 计算步骤:}投影$\to$写出面积微元$\to$计算积分
% 	  \item \ba{注意:}\alert{化成定积分后必须确保上限大于下限}
	\end{itemize}
\end{frame}

\begin{frame}
	\linespread{1.2}
	\ba{4、对坐标的曲面积分} 
	\begin{itemize}
	  \item {\bf 应用:}流(通)量 
	 	$$\alert{\iint_{\Sigma}P\d y\d z+Q\d z\d x+R\d x\d y
	 	=\iint_{\Sigma}\bm{v}\cdot\bm{n}\d S}
	 	{\color{yellow!90!gray}\mbox{\FiveStar}}$$ 
	 {\it 其中$\bm{n}$为曲面正向对应的单位法向量}  
	  \item 计算对对坐标的曲线积分的三种常见思路
	  \begin{itemize}
	    \item 直接计算:\ba{注意曲面正向和投影方向的关系,准确添加正负号}
	    \item 利用Gauss公式计算{\color{yellow!90!gray}\mbox{\FiveStar}}:
	    \ba{注意“补全”}
	    \item 化为对面积的曲面积分计算:\ba{注意选择正确的法向量}
	  \end{itemize}	  
	\end{itemize}
\end{frame}

\begin{frame}
	\linespread{1.2}
	{\bf 5、Green公式、Gauss公式、\color{gray!50!white}Stokes公式} 
	\begin{itemize}
	  \item 流量问题:Green公式的流量形式\small
	    $$\alert{\oint_LP\d y-Q\d x=\iint_{D}
	    \left(\df{\p P}{\p x}+\df{\p Q}{\p
	    y}\right)\d\sigma_{xy}}{\color{yellow!90!gray}\mbox{\FiveStar}}$$ 
	    $$\oint_L\bm{v}\cdot\bm{n}\d s
	    =\iint_{D}{\mathrm{div}\,\bm{v}}{\d\sigma_{xy}}$$ 
	  \item \normalsize\ba{Green公式的“挖洞”问题}
	  {\color{yellow!90!gray}\mbox{\FiveStar}}
	  \item 注意区分Gauss公式的左端和Stokes公式的左端
	\end{itemize}
\end{frame}

\begin{frame}
	\linespread{1.2} 
	\begin{itemize}
	  \item 环量问题: Green公式的环量形式\small
	  $$\alert{\oint_LP\d x+Q\d y=\iint_{D}
	    \left(\df{\p Q}{\p x}-\df{\p P}{\p
	    y}\right)\d\sigma_{xy}}{\color{yellow!90!gray}\mbox{\FiveStar}}$$ 
	  $$\oint_{\p\Sigma}\bm{v}\cdot\bm{T}ds=
	  \iint_{\Sigma}{\mathrm{rot}\,\bm{v}}\cdot\bm{n}{dS}$$ 
	  \item \ba{$\mathrm{rot}\bm{v}=\bigtriangledown\times\bm{v}$} 
	  \item \ba{无旋场$\Leftrightarrow$保守场
	  $\Leftrightarrow$积分与路径无关$\Leftrightarrow$被积式为全微分}
	  \item 求原函数{\color{yellow!90!gray}\mbox{\FiveStar}}:折线法、\ba{凑微分法}、逐步积分法
	\end{itemize}
\end{frame}

\begin{frame}
	\linespread{1.5}
	{\bf 7、对称性在积分计算中的应用{\color{yellow!90!gray}\mbox{\FiveStar}}} 
% 	\begin{enumerate}
% 	  \item {\bf 关键词}
	  \begin{itemize}
		\item 区域的对称性 
		\item 函数的奇偶性
		\item 变量的对等性
	  \end{itemize}
	  \bigskip
	  \ba{{\bf 注意:}对坐标的曲线(面)积分与其他积分的对称性存在明显差异!} 
	  \vspace{1em}
% 	\end{enumerate}
	\hrule
	\bigskip
	\centerline{\ba{计算各类积分前,优先考虑对称性}}
\end{frame}

\begin{frame}{六、函数项级数}
	\linespread{1.2} 
	\ba{1、幂级数} 
	
	{\b{\bf 关键词:} 收敛域、 求和、 展开} 
	\begin{itemize}
	  \item 幂级数的收敛域: 对称性、\ba{区间端点须单独讨论} 
	  \item 幂级数求和{\color{yellow!90!gray}\mbox{\FiveStar}}:
	  {\it 逐项求导、积分,利用已知级数展开式}\\
	  {\bf 要点:} \ba{变量替换、 补全缺项} 
	  \item 函数的幂级数展开{\color{yellow!90!gray}\mbox{\FiveStar}}:
	  变量替换、逐项求导、积分
	  \item \ba{不论是求和还是展开的问题,都必须说明级数的收敛域!!!}
	\end{itemize}
\end{frame}

\begin{frame}
	\linespread{1.2}
	\ba{2、Fourier级数} 
	
	{\b{\bf 关键词:} 和函数、 延拓、 正(余)弦级数} 
	\begin{itemize}
	  \item Fourier展开:\ba{正确写出不同积分区间上$f(x)$的表达式} 
	  \item 和函数:$S(x)=\df12(f(x+0)+f(x-0))$
	  {\color{yellow!90!gray}\mbox{\FiveStar}}
	  \item \ba{函数的延拓与正(余)弦级数}:{\it 根据所求级数的类型确定延拓的方法}
	  \item 周期为$2l$的Fourier级数
	\end{itemize}
\end{frame}

% \begin{frame}{重点}
% 	\linespread{1.5} 
% 	{\bf 
% 	\begin{enumerate}
% 	  \item 各种类型的积分计算 
% 	  \item 积分与微分的应用 
% 	  \item 空间解析几何 
% 	  \item 函数项级数 
% 	  \item 常微分方程
% 	\end{enumerate}
% 	}
% \end{frame}
% 
% \begin{frame}{难点}
% 	\linespread{1.2} 
% 	\begin{enumerate}
% 	  \item 常微分方程 
% 	  \item 幂级数求和 
% 	  \item 各种积分的相互转换 
% 	  \item Green公式与Gauss公式 
% 	  \item 空间对象的几何关系 
% 	  \item 多元函数求导
% 	\end{enumerate}
% \end{frame}

%=====================================

% \begin{frame}{title}
% 	\linespread{1.2}
% 	\begin{exampleblock}{{\bf title}\hfill}
% 		123
% 	\end{exampleblock}
% \end{frame}
% 
% \begin{frame}{title}
% 	\linespread{1.2}
% 	\begin{block}{{\bf title}\hfill}
% 		123
% 	\end{block}
% \end{frame}