\begin{center}
	{\Large\bf 极限与函数的连续性单元测验}
	
	(闭卷考试,时间:150分钟)
\end{center}

{\bf 一、选择题(每题2分)}

(1)\;设$f(x)=u(x)+v(x),\,g(x)=u(x)-v(x)$,且$\limx{x_0}u(x)$和
$\limx{x_0}v(x)$都不存在,则下列命题中正确的是(\quad)%D
\begin{enumerate}[(A)]
  \setlength{\itemindent}{1cm}
  \item 若$\limx{x_0}f(x)$不存在,则$\limx{x_0}g(x)$存在
  \item 若$\limx{x_0}f(x)$存在,则$\limx{x_0}g(x)$存在
  \item 若$\limx{x_0}f(x)$不存在,则$\limx{x_0}g(x)$不存在
  \item 若$\limx{x_0}f(x)$存在,则$\limx{x_0}g(x)$不存在
\end{enumerate}

(2)\;函数$f(x)=x\sin x$(\quad)%A
\begin{tabbing}
	\hspace{8cm}\=\kill
	\quad\quad\quad(A)\;当$x\to+\infty$时为无穷大 \> 
	(B)\;在$(-\infty,+\infty)$内无界 \\ 
	\quad\quad\quad(C)\;仅当$x\to0$时为无穷小\>
	(D)\;当$x\to+\infty$时收敛
\end{tabbing}

(3)\;$\limx{\infty}\df{e^{\sin\frac1x}-1}{\left(1+\df1x\right)^a
-\left(1+\df1x\right)}=A\ne 0$的充要条件是(\quad)%B

\quad (A)\;$a>1$\quad\quad\quad(B)\;$a\ne 1$
\quad\quad\quad (C)\;$a>0$\quad\quad\quad(D)\;与$a$无关

(4)\;已知$\alpha(x),\beta(x)$均为$x\to x_0$时的无穷小$(\beta(x)\ne0)$,
则当$x\to x_0$时,以下不一定是无穷小的是(\quad)%A

\begin{tabbing}
	\hspace{8cm}\=\kill
	\quad\quad\quad(A)\;$\df{\alpha(x)}{\beta(x)}$ \> 
	(B)\;$\alpha^2(x)+\beta^2(x)\cdot\cos\df1x$ \\ 
	\quad\quad\quad(C)\;$\ln[1+\alpha(x)\beta(x)]$\>
	(D)\;$|\alpha(x)|+|\beta(x)|$ 
\end{tabbing}

(5)\;设$f(x)=\df1{e^{\frac{x}{x-1}}-1}$,则(\quad)%D
\begin{enumerate}[(A)]
  \setlength{\itemindent}{1cm}
  \item $x=0,x=1$都是$f(x)$的第一类间断点
  \item $x=0,x=1$都是$f(x)$的第二类间断点
  \item $x=0$都是$f(x)$的第一类间断点,$x=1$都是$f(x)$的第二类间断点
  \item $x=0$都是$f(x)$的第二类间断点,$x=1$都是$f(x)$的第一类间断点
\end{enumerate}

(6)\;设$a,b>0$,则$\limn\sqrt[n]{a^n+b^n}=$(\quad)%D

\begin{tabbing}
	\hspace{8cm}\=\kill
	\quad\quad\quad(A)\;$a$ 
	\> (B)\;$\df12(a+b-|a-b|)$\\ 
	\quad\quad\quad(C)\;$b$
	\>	(D)\;$\df12(a+b+|a-b|)$ 
\end{tabbing}

% (6)\;设$f(x)=\df{\ln|x|}{|x-1|}\sin x$,则$f(x)$有(\quad)%C
% 
% \begin{tabbing}
% 	\hspace{8cm}\=\kill
% 	\quad\quad\quad(A)\;1个可去间断点,1个跳跃间断点 
% 	\> (B)\;1个跳跃间断点,一个无穷间断点 \\ 
% 	\quad\quad\quad(C)\;2个可去间断点
% 	\>	(D)\;2个无穷间断点 
% \end{tabbing}

{\bf 二、填空题(每题3分)}

(1)\;设$\limx{\infty}\df{(x-1)(x-2)(x-3)(2x-5)}{(2x-1)^{\alpha}}=\beta\ne 0$,
则$\beta=$\underline{\hspace{4cm}}.

(2)\;若$x\to0$时,$\ln\df{1-ax^2}{1+ax^2}\sim\df1{10000}x^4+\sin^2(\sqrt6x)$,
则$a=$\underline{\hspace{4cm}}.

(3)\;已知$F_n=\df1{\sqrt5}\left[\left(\df{1+\sqrt5}2\right)^{n+1}
-\left(\df{1-\sqrt5}2\right)^{n+1}\right]$,则$\limn\df{F_n}{F_{n+1}}=$
\underline{\hspace{4cm}}.

(4)\;若$x\to\pi$时,$\sqrt[4]{\sin\df x2}-1\sim A(x-\pi)^k$,则$A=$
\underline{\hspace{4cm}}.

\bigskip

{\bf 三、计算以下极限(每题6分)}
\begin{enumerate}[(1)]
  \setlength{\itemindent}{1cm}
  \item $\limn n^2\left(\df
  kn-\df1{n+1}-\df1{n+2}-\ldots-\df1{n+k}\right),\quad(k\in\mathbb{Z}_+)$
%   \item $\limn\df{1+2^3+\ldots+n^3}{n^4}$
%   \item $\limn\left(1+\df1n+\df 1{n^2}\right)^n$
  \item $\limn\df{n^5}{2^n}$
  \item $\limn\df{1+\sqrt[n]2+\sqrt[n]3+\ldots+\sqrt[n]n}n$
%   \item $\limn n(\sqrt[n]a-1),\quad(a>0)$
%   \item $\limn\left[\df1{1\cdot 2\cdot 3}+\df1{2\cdot 3\cdot 4}+\ldots+
%   \df1{n\cdot (n+1)\cdot (n+2)}\right]$
%   \item $\limn\df{n^5}{2^n}$
%   \item 
%   \item $\limx{0}\df{\tan x-\sin x-x^3}{\sin^3x}$
  \item $\limx{\infty}\left[\df{x^2}{(x-a)(x-b)}\right]^x$
  \item $\limx{+\infty}(\sqrt{x^2+x}-\sqrt[3]{x^3+x^2})$ 
  \item $\limx{0}\df{1-\cos x\cos 2x}{x^2}$
%   \item $\limx{+\infty}\df{\ln(2+3e^{3x})}{\ln(3+2e^{2x})}$
  \item $\limx{0}\df1{x^3}\left[\left(\df{2+\cos x}3\right)^x-1\right]$
%   \item $\limx{\pi/4}(\tan x)^{\tan 2x}$
%   \item $\limx{0}\df{\arctan x-\sin x}{x^3}$
  \item $\limx{0}\df{\df1{1+x^2}-\cos x}{x^2}$
\end{enumerate}

% {\bf 四、判定以下级数的敛散性(每题4分)}
% \begin{enumerate}[(1)]
%   \setlength{\itemindent}{1cm}
% %   \item $\sum\limits_{n=2}^{\infty}\df1{\ln n^{\ln n}}$
% % %   \item $\sumn\sin(\pi\sqrt{n^2-1})$
% %   \item $\sumn(-1)^n\df{n^{n+\frac1n}}{\left(n+\df1n\right)^n}$
%   \item $\sumn\df{n!}{{(\sqrt5+1)}{(\sqrt5+2)}\ldots{(\sqrt5+n)}}$
%   \item $\sumn(-1)^n\df{n^{n+\frac1n}}{\left(n+\df1n\right)^n}$
% %   \item $\sumn\left(\cot\df{n\pi}{4n-2}-\sin\df{n\pi}{2n+1}\right)$
%   \item $\sum\limits_{n=2}^{\infty}\sin\left(n\pi+\df1{\ln n}\right)$
%   \item $\sumn(-1)^n\ln\left(1+\df1n\right)$
% \end{enumerate}

% {\bf 三、用定义证明以下极限(每题5分):}
% \begin{enumerate}[(1)]
%   \setlength{\itemindent}{1cm}
%   \item $(n+1)^k-n^k\to 0,\;(n\to\infty)$,其中$0<k<1$
%   \item $\limx{1}\df1{x}=1$
% \end{enumerate}

{\bf 四(6分)、}设
$$f(x)=\lim\limits_{t\to x}\left(\df{x-1}{t-1}\right)^{\frac{t}{x-t}},$$
求$f(x)$的连续区间和间断点,并指出其间断点的类型。

{\bf 五(6分)、}用数列极限的定义证明:若$\limn a_n=a\ne0$,则
$$\limn\df1{a_n}=\df1a.$$

% 证明不等式形式的比值判别法:
% 设$a_n>0\;(n=1,2,\ldots)$,记$r_n=\df{a_{n+1}}{a_n}$,证明:
% \begin{enumerate}[(1)]
%   \setlength{\itemindent}{1cm}
%   \item 若对充分大的$n$,总有$r_n\geq 1$,则$\sumn a_n$发散;
%   \item 若存在常数$r\in(0,1)$,使对充分大的$n$,总有$r_n\leq r$,则$\sumn a_n$收敛。
% \end{enumerate}

{\bf 六(6分)、}设$f(x)$为三次多项式,$a\ne 0$,且
$\limx{2a}\df{f(x)}{x-2a}=\limx{4a}\df{f(x)}{x-4a}=1$,
求$\limx{3a}\df{f(x)}{x-3a}$.

{\bf 七(10分)、}证明:
\begin{enumerate}[(1)]
  \setlength{\itemindent}{1cm}
  \item 方程$x^n+x^{n-1}+\ldots+x=1\;(n=2,3,\ldots)$在$\left(\df12,1\right)$
  内有且仅有一个实根;
  \item 记(1)中的实根为$x_n$,证明$\limn x_n$存在,并求此极限。
\end{enumerate}

\newpage

\begin{center}
	{\Large\bf 解答与评分标准}\ps{\b 1.若解法与参考答案不同,参照本评分标准的特点分段给分\\
	2.计算题只写结果,缺少计算过程,最多得1分}
\end{center}

{\bf 一、选择题(每题2分):}\quad D\quad B\quad B\quad A\quad D\quad D

[提示]:

(3)
$$\mbox{原式}=\limx{\infty}\df{\sin\df1x}{\left(1+\df1x\right)^{a-1}-1}
=\lim\limits_{y\to0}\df{\sin y}{(1+y)^{a-1}-1}
==\lim\limits_{y\to0}\df{y}{(a-1)y}=\df1{a-1}$$

(6)
$\df12(a+b+|a-b|)=\max\{a,b\}$

{\bf 二、填空题(每题3分)}

(1)\;$\df18$\quad\quad(2)\;$-3$\quad\quad
(3)\;$\df2{1+\sqrt5}$\quad\quad(4)\;$-\df1{32}$

[提示]:(1)该极限存在且不为零,当且仅当分子分母的最高次幂相同,故$\alpha=4$,
相应地极限的结果就是分子分母最高次幂的系数的比值,也即$\df2{2^4}=\df18$;

(2)$x\to0$时,
$$\ln\df{1-ax^2}{1+ax^2}\sim\df{1-ax^2}{1+ax^2}-1
=\df{-2ax^2}{1+ax^2}\sim-2ax^2,$$
$$\df1{10000}x^4+\sin^2{\sqrt6x}\sim6x^2,$$
故$a=-3$;

(3)
\begin{align*}
	\limn\df{F_n}{F_{n+1}}
	&=\limn\df{\left(\df{1+\sqrt5}2\right)^{n+1}
	-\left(\df{1-\sqrt5}2\right)^{n+1}}
	{\left(\df{1+\sqrt5}2\right)^{n+2}
	-\left(\df{1-\sqrt5}2\right)^{n+2}}\\
	&=\limn\df{1-\left(\df{1-\sqrt5}{1+\sqrt5}\right)^{n+1}}
	{\df{1+\sqrt5}2-\df{1-\sqrt5}2\left(\df{1-\sqrt5}{1+\sqrt5}\right)^{n+1}}\\
	&=\df2{1+\sqrt5}.
\end{align*}

(4)记$y=x-\pi$,则$x\to\pi$等价于$y\to0$,此时
$$\sqrt[4]{\sin\df x2}-1=\sqrt[4]{\sin\left(\df y2+\df{\pi}2\right)}-1
=\sqrt[4]{\cos\df y2}-1\sim\df14\left(\cos\df y2-1\right)
\sim-\df1{32}y^2.$$
$$A(x-\pi)^k=Ay^k.$$
故$A=-\df1{32}.$

{\bf 三、计算以下极限(每题4分)}\ps{\b 用L'Hospital法则求解一律不得分!}

(1)
\begin{align}
	\mbox{原式}&=\limn n^2\left[\left(\df1n-\df1{n+1}\right)+
	\left(\df1n-\df1{n+2}\right)+\ldots+
	\left(\df1n-\df1{n+k}\right)\right]\notag\\
	&=\limn\df{n^2}{n(n+1)}+\limn\df{2n^2}{n(n+2)}
	+\ldots+\limn\df{kn^2}{n(n+k)}\tag{+4分}\\
	&=1+2+\ldots+k=\df{k(k+1)}2\tag{+2分}
\end{align}

(2)\;记$a_n=\df{n^5}{2^n}$,显然$a_n>0\,(n\in\mathbb{N})$。又当$n>\df{1}{\sqrt[5]2-1}$时,
  $$\df{a_{n+1}}{a_n}=\left(1+\df1{n}\right)^5\cdot\df 12<1,
  \eqno{(+3\;\mbox{分})}$$
  故由单调有界原理,$\{a_n\}$收敛,不妨设其极限为$a$。对递推式
  $$a_{n+1}=\left(1+\df1{n}\right)^5\cdot\df 12a_n$$
  两端同时取极限,可得$a=\df 12a$,从而可知$\limn a_n=a=0$。\hfill(+3分)

% 注意到
%   $$\left(1+\df 1n\right)^n<\left(1+\df1n+\df 1{n^2}\right)^n<\left(1+\df
%   1{n-1}\right)^n,\eqno{(+4\;\mbox{分})}$$
%   而$\limn\left(1+\df 1n\right)^n=\limn\left(1+\df1{n-1}\right)^n=e$,故有夹逼定理:
%   原式$=e$。\hfill(+2分)
% 
% 由Stolz定理,
%   \begin{align}
%   	\mbox{原式}&=\limn\df{n^3}{n^4-(n-1)^4}\tag{{+2\;\mbox{分}}}\\
%   	&=\limn\df{n^3}{C_4^1n^3-C_4^2n^2+C_4^3n-1}=\df14\tag{{+2\;\mbox{分}}}
%   \end{align}
  
(3)\ps{用Stolz定理不得分}
$$1=\df{1+1+\ldots+1}n<\df{1+\sqrt[n]2+\ldots+\sqrt[n]n}n<\df{n\cdot\sqrt[n]n}n,$$
注意到$\limn\sqrt[n]n=1$,故由夹逼定理,原式$=1$。\hfill{(+6分)}
  
% (3)\;由Henie定理,
% $$\mbox{原式}=\limx{+\infty}x\left(a^{\frac1x}-1\right)
% =\lim\limits_{y\to0^+}\df{a^y-1}y=\ln a\eqno{(+4\mbox{分})}$$

(4)
\begin{align}
  	\mbox{原式}&=\limx{\infty}\left\{\left[1+\df{(a+b)x-ab}
  	{(x-a)(x-b)}\right]^{\frac{(x-a)(x-b)}{(a+b)x-ab}}\right\}
  	^\frac{[(a+b)x-ab]x}{(x-a)(x-b)\tag{{+3\;\mbox{分}}}}\\
  	&=\left\{\limx{\infty}\left[
  	1+\df{(a+b)x-ab}{(x-a)(x-b)}\right]^{\frac{(x-a)(x-b)}
  	{(a+b)x-ab}}\right\}^{\limx{\infty}\frac{[(a+b)x-ab]x}
  	{(x-a)(x-b)}}=e^{a+b}\tag{{+3\;\mbox{分}}}
  \end{align}

(5)
  \begin{align}
  	\mbox{原式}&=\limx{+\infty}(x^3+x^2)^{\frac13}
  	\left[\left(\df{x+1}x\right)^{\frac16}-1\right]\tag{{+2\;\mbox{分}}}\\
  	&=\limx{+\infty}\left(1+\df1x\right)^{\frac13}\df16
  	\ln\left(\df{x+1}x\right)^x\tag{{+3\;\mbox{分}}}\\
  	&=\df16\tag{{+1\;\mbox{分}}}
  \end{align}

(6)
  \begin{align}
  	\mbox{原式}&=\limx{0}\df{(1-\cos x)+\cos
  	x(1-\cos2x)}{x^2}\tag{{+3\;\mbox{分}}}\\
  	&=\limx{0}\df{1-\cos x}{x^2}+
  	\limx{0}\cos x\df{1-\cos2x}{x^2}
  	=\df12+\limx{0}\df{2x^2}{x^2}=\df52\tag{{+3\;\mbox{分}}}
  \end{align}

(7)
  \begin{align}
  	\mbox{原式}&=\limx{0}\df1{x^3}\left[\exp\left(
  	x\ln\df{2+\cos x}{3}\right)-1\right]\notag\\
  	&=\limx{0}\df{x\ln\df{2+\cos x}3}{x^3}\tag{{+2\;\mbox{分}}}\\
  	&=\limx{0}\df{\cos x-1}{3x^2}\tag{{+2\;\mbox{分}}}\\
  	&=-\df16\tag{{+2\;\mbox{分}}}
  \end{align}

% (8)\;令$y=\arctan x$,
%   \begin{align}
%   	\mbox{原式}&=\lim\limits_{y\to 0}\df{y-\df
%   	y{\sqrt{1+y^2}}}{y^3}\tag{{+2\;\mbox{分}}}\\
%   	&=\lim\limits_{y\to 0}\df{(\sqrt{1+y^2}-1)}{y^2\sqrt{1+y^2}}
%   	=\lim\limits_{y\to 0}\df1{\sqrt{1+y^2}+1}=\df12\tag{{+2\;\mbox{分}}}
%   \end{align}

(8)\;
\begin{align}
	\mbox{原式}&=\limx0\df{\df1{1+x^2}-1}{x^2}+\limx0\df{1-\cos
	x}{x^2}\tag{{+3\;\mbox{分}}}\\
	&=\limx0\df{-x^2}{x^2}+\limx0\df{\df12x^2}{x^2}=-1+\df12=-\df12
	\tag{{+3\;\mbox{分}}}
\end{align}

% {\bf 四、判定以下级数的敛散性(每题4分)}
% 
% (1)\;注意到
% $$\df{n!}{{(\sqrt5+1)}{(\sqrt5+2)}\ldots{(\sqrt5+n)}}
% <\df{n!}{3\cdot4\cdot5\cdots(2+n)}=\df2{(n+1)(n+2)}<\df2{n^2},$$
% 级数$\sumn\df1{n^2}$收敛,由比较判别法,原级数收敛。\hfill{{(+4分)}}
% 
% (2)\;注意到$\limn\sqrt[n]n=1$,且
% $$\limn\left(1+\df1{n^2}\right)^n=\limn\exp\left[\df1n
% \ln\left(1+\df1{n^2}\right)^{n^2}\right]
% =\exp\left[\limn\df1n\cdot
% \ln\limn\left(1+\df1{n^2}\right)^{n^2}\right]
% =1\eqno{(+2\;\mbox{分})}$$
% 故
% $$\df{n^{n+\frac1n}}{\left(n+\df1n\right)^n}=\df{n^{\frac1n}}
% {\left(1+\df1{n^2}\right)^n}\to 1,\;(n\to\infty).$$
% 不满足级数收敛的必要条件,故原级数发散。\hfill{{(+2分)}}
% 
% (3)\;注意到$\sin\left(n\pi+\df1{\ln n}\right)=(-1)^n\sin\df1{\ln n}$,
% 显然$\left\{\sin\df1{\ln n}\right\}$单调递减趋于零,故由Leibniz判别法,原级数收敛。
% \hfill{{(+2分)}}
% 
% 又
% $$\limn\df{\sin\df1{\ln n }}{\df1{\ln n}}=1,$$
% 且$\df1{\ln n}>\df1n$,故由比较判别法可知级数$\sumn\sin\df1{\ln n }$发散。
% 
% 综上,原级数条件收敛。\hfill{{(+2分)}}
% 
% (4)\;显然$\left\{\ln\left(1+\df1n\right)\right\}$单调递减趋于零,故由Leibniz
% 判别法,原级数收敛。\hfill{{(+2分)}}
% 
% 又
% $$\limn\df{\ln\left(1+\df1n\right)}{\df1n}=\ln\limn\left(1+\df1n\right)^n=1,$$
% 级数$\sumn\df1n$发散,故由比较判别法,级数$\sumn\ln\left(1+\df1n\right)$发散。
% 
% 综上,原级数条件收敛。\hfill{{(+2分)}}

{\bf 四(6分)、}解:由已知
$$
	f(x)=\lim\limits_{t\to x}\left[\left(1+\df{x-t}{t-1}\right)
	^{\frac{t-1}{x-t}}\right]^{\frac{t}{t-1}}
	=\left[\lim\limits_{t\to x}\left(1+\df{x-t}{t-1}\right)
	^{\frac{t-1}{x-t}}\right]^{\lim\limits_{t\to x}\frac{t}{t-1}}\notag\\
	=e^{\frac{x}{x-1}}\eqno{(+3\mbox{分})}.
$$
显然$f(x)$在$x=1$处无定义,故易知其连续区间为$(-\infty,1)$和$(1,+\infty)$。又
$$\limx{1^+}e^{\frac{x}{x-1}}=+\infty,\quad\quad\limx{1^-}e^{\frac{x}{x-1}}=0,$$
故$x=1$为$f(x)$的无穷间断点。\hfill{{(+3分)}}

{\bf 五(6分)、}证:\ps{未用定义证明,此题得零分}
由极限的保号性可知,存在$N_1$,对任意$n>N_1$,使得
$$|a_n|>\df{|a|}2.\eqno{(+2\;\mbox{分})}$$
任取$\e>0$,由$\limn a_n=a$,对$\e_1=\df{a^2}2\e$,存在$N_2$,对任意$n>N_2$,恒有
$$|a_n-a|<\e_1.\eqno{(+2\;\mbox{分})}$$
从而,对任意$n>N=\max\{N_1,N_2\}$,总有
$$\left|\df1{a_n}-\df1a\right|=\df{|a_n-a|}{|a_na|}
<\df{\e_1}{\df{a^2}2}=\e.\eqno{(+2\;\mbox{分})}$$
即证。

% (1)\;由已知,存在$N\in\mathbb{N}$,对$\forall n>N$,都有$r_n\geq1$,从而
% $$a_n\geq a_{n-1}\geq\ldots\geq a_N.\eqno{(+2\;\mbox{分})}$$
% 由数列极限的保号性,易知$\{a_n\}$不可能以$0$为极限,不满足级数收敛的必要条件,故原级数发散。
% \hfill{{(+1分)}}
% 
% (2)\;由已知,存在$N\in\mathbb{N}$,对$\forall n>N$,
% 都有$r_n\leq r$,从而
% $$a_n<ra_{n-1}<\ldots<r^{n-N}a_N.\eqno{(+2\;\mbox{分})}$$
% 当$r<1$时,几何级数$\sum\limits_{n=N}^{\infty}r^{n-N}a_N$收敛,故
% 由比较判别法,原级数收敛。\hfill{{(+1分)}}

{\bf 六(6分)、}解:由已知可设
$$f(x)=A(x-B)(x-C)(x-D),$$
其中$A,B,C,D$均为常数,$A\ne 0$。\hfill{{(+1分)}}

由$\limx{2a}\df{f(x)}{x-2a}=1$,可以断言$B,C,D$之中有且仅有一个为$2a$。事实上,若
$B,C,D$均不等于$2a$,则显然(由于分子趋于有限值,分母趋于$0$)该极限不存在;又若
$B,C,D$中有不少于两个等于$2a$,不妨设$B=C=2a$,则必有
$$\limx{2a}\df{A(x-2a)(x-2a)(x-D)}{x-2a}=\limx{2a}A(x-2a)(x-D)
=0\ne1.$$
以下不妨设$B=2a$。

同理由$\limx{4a}\df{f(x)}{x-4a}=1$可得,$B,C,D$之中有且仅有一个为$4a$,不妨设$C=4a$。
\hfill{{(+3分)}}

至此,可得$f(x)=A(x-2a)(x-4a)(x-D)$,其中$D$不等于$2a$和$4a$,于是
$$\limx{2a}\df{f(x)}{x-2a}=\limx{2a}A(x-4a)(x-D)=A(-2a)(2a-D)=1,$$
$$\limx{4a}\df{f(x)}{x-4a}=\limx{4a}A(x-2a)(x-D)=A(2a)(4a-D)=1,$$
由此可解得$D=3a,A=\df1{2a^2}$。从而
$$\limx{3a}\df{f(x)}{x-3a}=\limx{3a}\df1{2a^2}(x-2a)(x-4a)=-\df12.
\eqno{(+2\mbox{分})}$$

{\bf 七(10分)、}证:(1)\;设
  $$P_n(x)=x^n+x^{n-1}+\ldots+x-1,\;(n=2,3,\ldots).$$
  注意到$P_n(x)$在$\left[\df12,1\right]$上连续,且$P_n(1)=n-1>0$,
  $$P_n\left(\df12\right)=\df1{2^n}+\df1{2^{n-1}}+\ldots+\df12-1
  =-\df1{2^n}<0,$$
  故由介值定理,必存在$x_n\in\left(\df12,1\right)$,使得$P_n(x_n)=0$。\hfill{{(+2分)}}
  
  显然$P_n(x)$在$\left(\df12,1\right)$内严格单调递增,故以上的$x_n$唯一。
  \hfill{{(+2分)}}

(2)\;由(1)可知,$0<\df12<x_n<1,\;n=2,3,\ldots$。又由$P_n(x_n)=
  P_{n+1}(x_{n+1})=0$,\ps{$P_{n+1}(x_{n+1})=x_{n+1}^{n+1}+P_n(x_{n+1})=0$}
  $$P_n(x_n)-P_n(x_{n+1})=(x_n^n+x_n^{n-1}+\ldots+x_n-1)
  -(x_{n+1}^n+x_{n+1}^{n-1}+\ldots+x_{n+1}-1)=x_{n+1}^{n+1}>0,$$
  由$P_n(x)$的单调性可知$\{x_n\}$是单调递减的。由单调有界原理,
  $\{x_n\}$收敛。\hfill{{(+3分)}}
  
  设$a=\limn x_n$。注意到$n>2$时,$0<x_n<x_2<1$,所以$0<x_n^n<x_2^n$,由夹逼定理可知,
  $\limn x_n^n=0$。又等式\ps{\b 由$0<x_n<1$直接推出$\limn x_n^n=0$,本步不得分!}
  $x_n^n+x_n^{n-1}+\ldots+x_n=1$
  即为
  $$\df{x_n(1-x_n^n)}{1-x_n}=1,$$
  两边同时取极限,可得$\df a{1-a}=1$,进而解得$a=\df12$,即为所求。\hfill{{(+3分)}}