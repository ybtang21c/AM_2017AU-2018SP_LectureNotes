% \begin{tikzpicture}
% 	\tikzstyle{level 1}=[sibling angle=120]
% 	\tikzstyle{level 2}=[sibling angle=60]
% 	\tikzstyle{level 3}=[sibling angle=30]
% 	\tikzstyle{every node}=[fill]
% 	\tikzstyle{edge from parent}=[snake=expanding waves,segment length=1mm,
% 		segment angle=10,draw]
% 	%grow cyclic无法被识别!
% 	\tikz [grow cyclic,shape=circle,very thick,level distance=13mm,cap=round]
% 	  \node {} child [color=\A] foreach \A in {red,green,blue}
% 	    { node {} child [color=\A!50!\B] foreach \B in {red,green,blue}
% 	      { node {} child [color=\A!50!\B!50!\C] foreach \C in {black,gray,white}
% 			{ node {} } 
% 		  }
% 		}
% \end{tikzpicture}

%inline graph
% 行内的正弦 \tikz \draw[x=1ex,y=1ex] (0,0) sin (1.57,1); 曲线!

% 圆角折线
% \tikz \draw[thick,rounded corners=8pt]
% (0,0) -- (0,2) -- (1,3.25) -- (2,2) -- (2,0) 
% -- (0,2) -- (2,2) -- (0,0) -- (2,0);
% 
% 利用控制点画曲线
% \begin{tikzpicture}
%   \filldraw [gray] (0,0) circle (2pt)
%                    (1,1) circle (2pt)
%                    (2,1) circle (2pt)
%                    (2,0) circle (2pt);
%   \draw (0,0) .. controls (1,1) and (2,1) .. (2,0);
% \end{tikzpicture}
% 
% \bigskip

% 带网格的坐标系
% 
% \begin{tikzpicture}[scale=3, >=stealth]
% 	%剪切绘图
% % 	\clip[draw] (-0.1,-0.2) rectangle (1.1,0.75);
% 	%自定义样式
% % 	\tikzstyle help lines=[color=blue!50,very thin]
% % 	\draw[step=.5cm, help lines] (-1.4,-1.4) grid (1.4,1.4);
%  	\draw[step=.5cm,gray,very thin] (-1.4,-1.4) grid (1.4,1.4);	
%   	\draw[->] (-1.5,0) -- (1.5,0);
%   	\draw[->] (0,-1.5) -- (0,1.5);
%   	\draw (0,0) circle (1cm);
%   	\fill[fill=green!20!white, draw=green!50!black] (0,0) -- (3mm,0mm) arc (0:30:3mm) -- cycle;
%   	%渐变填充
% %   \shadedraw[left color=gray,right color=green, draw=green!50!black]
% %   (0,0) -- (3mm,0mm) arc (0:30:3mm) -- cycle;
% 	\draw[red,very thick]
% 		(30:1cm) 
% 			--node[left=1pt,fill=white] {$\sin \alpha$} 
% 			(30:1cm |- 0,0);
% 	\draw[blue,very thick] 
% 		(30:1cm) ++(0,-0.5) 
% 			--node[below=2pt,fill=white] {$\cos\alpha$} (0,0);
% 	\draw[very thick,orange]
% 		(1,0) --node [right=1pt,fill=white]
%     		{$\displaystyle \tan \alpha \color{black}=
%       		\frac{{\color{red}\sin \alpha}}{\color{blue}\cos \alpha}$} 
% 		 	(intersection of 1,0--1,1 and 0,0--30:1cm) coordinate (t);
% 	\draw (0,0) -- (t);%t为上一行记录下来的坐标
% 	
% 	\foreach \x/\xtext in {-1, -0.5/-\frac{1}{2},1}
%     	\draw (\x cm,1pt) -- (\x cm,-1pt) node[anchor=north east] {$\xtext$};
%   	\foreach \y/\ytext in {-1, -0.5/-\frac{1}{2}, 0.5/\frac{1}{2}, 1}
%     	\draw (1pt,\y cm) -- (-1pt,\y cm) node[anchor=north east] {$\ytext$};
% \end{tikzpicture}

% \bigskip

% 部分图形的抽离
% \begin{tikzpicture}[scale=2]
%   \draw (0,0) -- (90:1cm) arc (90:360:1cm) arc (0:30:1cm) -- cycle;
%   \draw (60:5pt) -- +(30:1cm) arc (30:90:1cm) -- cycle;
%   \draw (3,0)  +(0:1cm) -- +(72:1cm) -- +(144:1cm) -- +(216:1cm) --
%                +(288:1cm) -- cycle;
% \end{tikzpicture}

% 交叠的图形——sth wrong
% \begin{tikzpicture}[even odd rule,rounded corners=2pt,x=10pt,y=10pt]
%   \filldraw[fill=examplefill] (0,0)   rectangle (1,1)
%     [xshift=5pt,yshift=5pt]   (0,0)   rectangle (1,1)
%                 [rotate=30]   (-1,-1) rectangle (2,2);
% \end{tikzpicture}

% 循环语句
% \begin{tikzpicture}
%   \foreach \x in {1,2,...,5,7,8,...,12}
%     \foreach \y in {1,...,5}
%     {
%       \draw (\x,\y) +(-.5,-.5) rectangle ++(.5,.5);
%       \draw (\x,\y) node{\x,\y};
%     }
% \end{tikzpicture}

% 带标记的曲线
% \begin{tikzpicture}
%   \draw (0,0) .. controls (6,1) and (9,1) ..
%     node[near start,sloped,above] {near start}
%     node {midway}
%     node[very near end,sloped,below] {very near end} (12,0);
% \end{tikzpicture}

% 精细版本的图形示例
% 
% \begin{tikzpicture}[scale=3,cap=round]
% 	%定义常数:30度角的余弦值
%   	\def\costhirty{0.8660256}
% 	%定义会用到的配色
%   	\colorlet{anglecolor}{green!50!black}%角度的颜色
%   	\colorlet{sincolor}{red}%正弦的颜色
%   	\colorlet{tancolor}{orange!80!black}%正切的颜色
%   	\colorlet{coscolor}{blue}%余弦的颜色
% 	%定义样式
%   	\tikzstyle{axes}=[]%坐标轴
%   	\tikzstyle{important line}=[very thick]%重要的线条
%   	\tikzstyle{information text 1}%说明文字
%   		=[rounded corners,fill=red!10,inner sep=1ex]%圆角,淡红填充,内空1ex
%   	\tikzstyle{information text 2}%说明文字
%   		=[rounded corners,fill=green!10,inner sep=1ex]%圆角,淡绿填充,内空1ex
% 	%开始绘图
%   	%辅助网格
%   	\draw[style=help lines,step=0.5cm] (-1.4,-1.4) grid (1.4,1.4);
%   	%单位圆
%   	\draw (0,0) circle (1cm);
%   	%样式参数统一为axes
%   	\begin{scope}[style=axes]
%     	%x轴
%     	\draw[->] (-1.5,0) -- (1.5,0) node[right] {$x$} coordinate(x axis);
%     	%y轴
%     	\draw[->] (0,-1.5) -- (0,1.5) node[above] {$y$} coordinate(y axis);
%     	%标记坐标值,循环语句
%     	\foreach \x/\xtext in {-1, -.5/-\frac{1}{2}, 1}
%       		\draw[xshift=\x cm] (0pt,1pt) 
%       			-- (0pt,-1pt) node[below,fill=white]{$\xtext$}; 
%       	\foreach \y/\ytext in {-1, -.5/-\frac{1}{2}, .5/\frac{1}{2}, 1}
%       		\draw[yshift=\y cm] (1pt,0pt) 
%       			-- (-1pt,0pt) node[left,fill=white]{$\ytext$};
% 	\end{scope}
%   	%角,填充
%   	\filldraw[fill=green!20,draw=anglecolor] 
%   		(0,0) -- (3mm,0pt) arc (0:30:3mm);
%   	%角标记
%   	\draw (15:2mm) node[anglecolor] {$\alpha$};
%   	%正弦、余弦和正切的长度及标记
%   	\draw[style=important line,sincolor]
%     	(30:1cm) 
%     		-- node[left=1pt,fill=white] {$\sin \alpha$}(30:1cm |- x axis);
%   	\draw[style=important line,coscolor]
%     	(30:1cm |- x axis) 
%     		-- node[below=2pt,fill=white] {$\cos \alpha$}(0,0);
%   	\draw[style=important line,tancolor] 
%   		(1,0) -- node[right=1pt,fill=white] 
%   			{$\displaystyle \tan \alpha \color{black}=
%     		\frac{{\color{sincolor}\sin \alpha}}{\color{coscolor}\cos \alpha}$}
%     		(intersection of 0,0--30:1cm and 1,0--1,1) coordinate (t);
%   	%利用记录坐标值,画斜线
%   	\draw (0,0) -- (t);
%   	%辅助文字
%   	\draw[xshift=1.85cm,yshift=0.7cm]
%     	node[right,text width=6cm,style=information text 1]
%     	{
%       		The {\color{anglecolor} angle $\alpha$} is $30^\circ$ in the
%       		example ($\pi/6$ in radians). The {\color{sincolor}sine of
%         	$\alpha$}, which is the height of the red line, is
%       		\[
%       			{\color{sincolor} \sin \alpha} = 1/2.
%       		\]
%       		By the Theorem of Pythagoras ...
%     	};
%     \draw[xshift=1.85cm,yshift=-0.6cm]
%     	node[right,text width=6cm,style=information text 2]
%     	{
%     		\bf 这个例子很好地展示了:\it\\
%     		\quad $\cdot\;$常用的画图语句\\
%     		\quad $\cdot\;$各种样式选项\\
%     		\quad $\cdot\;$基本的循环控制语句\\
%     		\quad $\cdot\;$美观的图形布局\\
%     		\quad $\cdot\;$enjoy it !
%     	};
% \end{tikzpicture}

% Petri网示例
% 
% %定义tikz风格,place是petri类型中的node自定义风格
% \tikzstyle{every place}=[minimum size=6mm,thick,draw=blue!75,fill=blue!20]
% \tikzstyle{every transition}=[thick,draw=black!75,fill=black!20]
% \tikzstyle{red place}=[place,draw=red!75,fill=red!20]
% \tikzstyle{every label}=[red]
% %进入绘图
% %设定node间距,箭头类型,转弯角度
% %auto用于实现node自动安放
% \begin{tikzpicture}
% 	[node distance=1.3cm, >=stealth', bend angle=45, auto]
% 	%变迁前的状态图
% 	%根据位置关系自动确定node的位置
% 	\node [place,tokens=1] (w1){};%token表示node中的点数
% 	\node [place] (c1) [below of=w1]{};
% 	\node [place] (s) [below of=c1,label=above:$s\le 3$]{};
% 	\node [place] (c2) [below of =s]{};
% 	\node [place,tokens=1] (w2) [below of=c2]{};
% 	%划线,pre和post分别代表向后和向前,bend指明偏转的方向
% 	\node [transition] (e1) [left of=c1] {}
% 	  	edge [pre,bend left] (w1)
% 	  	edge [post,bend right]  (s)
% 	  	edge [post] (c1);
% 	\node [transition] (e2) [left of=c2] {}
% 	  	edge [pre,bend right] (w2)
% 	  	edge [post,bend left] (s)
% 	  	edge [post] (c2);
% 	\node [transition] (l1) [right of=c1] {}
% 	  	edge [pre] (c1)
% 	  	edge [pre,bend left] (s)
% 	  	edge [post,bend right] node[swap] {2} (w1);%swap用于把node放在默认的反方向
% 	\node [transition] (l2) [right of=c2] {}
% 	  	edge [pre] (c2)
% 	  	edge [pre,bend right] (s)
% 	  	edge [post,bend left]  node {2} (w2);
% 	%变迁后的状态图,设置公共参数
% 	\begin{scope}[xshift=6cm]
%   		%画节点
%   		\node [place,tokens=1] (w1'){};
%   		\node [place](c1')[below of=w1']{};
%   		\node [red place](s1') 
%   			[below of=c1',xshift=-5mm][label=left:$s$]{};
% 		\node [red place,tokens=3] (s2') 
% 			[below of=c1',xshift=5mm][label=right:$\bar s$]{};
% 		\node [place] (c2') [below of=s1',xshift=5mm]{};
% 		\node [place,tokens=1] (w2') [below of=c2']{};
% 		%画线
% 		\node [transition] (e1') [left of=c1'] {}
% 			edge [pre,bend left](w1')
% 			edge [post](s1')
% 			edge [pre](s2')
% 			edge [post](c1');
% 		\node [transition] (e2') [left of=c2'] {}
% 			edge [pre,bend right](w2')
% 			edge [post](s1')
% 			edge [pre](s2')
% 			edge [post](c2');
% 		\node [transition] (l1') [right of=c1'] {}
% 		  	edge [pre](c1')
% 		  	edge [pre](s1')
% 		  	edge [post](s2')
% 		  	edge [post,bend right] node[swap] {2} (w1');
% 		\node [transition] (l2') [right of=c2'] {}
% 		    edge [pre] (c2')
% 		    edge [pre] (s1')
% 		    edge [post] (s2')
% 		    edge [post,bend left]  node {2} (w2');
% 	\end{scope}
% 	%画蛇形线,标记文字
% 	%如下两种不同的写法皆可
% % 	\draw [-to ,decorate,decoration={snake,amplitude=.4mm,
% % 		segment length=2mm,post length=1mm}]
% 	\draw [->,thick,snake=snake,
% 		segment amplitude=.4mm,
% 		segment length=2mm,line
% 		after snake=1mm]
% 		 ([xshift=5mm]s -| l1) -- ([xshift=-5mm]s1' -| e1')
%     	node [above=1mm,midway,text width=3cm,text centered]
%       		{replacement of the \textcolor{red}{capacity}
%       		by \textcolor{red}{two places}};
%   	
%   	%画背景框
%   	\begin{pgfonlayer}{background}%置于底层
%     	\filldraw [line width=4mm,join=round,black!10]
%       		(w1.north  -| l1.east)  rectangle (w2.south  -| e1.west)
%       		(w1'.north -| l1'.east) rectangle (w2'.south -| e1'.west);
%   	\end{pgfonlayer}
% \end{tikzpicture}

自定义node,画图

\tikzstyle{int}=[ultra thick, rounded corners,
	draw=black!40,fill=red!10!white,
 	inner sep=1ex,
	text width=2.5cm,
% 	text height=5ex,
	minimum height=1.8cm,
	text centered]
\tikzstyle{mint}=[ultra thick, rounded corners,
	draw=black!40,fill=yellow!10!green!10,
 	inner sep=1ex,
	minimum width=3.5cm,
% 	text height=5ex,
	minimum height=1.5cm,
	text centered]
\tikzstyle{arrow}=[thick,blue!80]
\begin{tikzpicture}
 	[node distance=4cm, >=stealth', thick, bend angle=45, auto]
	\node [int] (iiint) 
		{\large 三重积分\\
		\small$\ds\iiint\limits_{\Omega}f(x,y,z)\d V$
	};
	\node [int, below of=iiint] (iint) 
		{\large 二重积分\\
		\small$\ds\iint\limits_Df(x,y)\d\sigma$
	};
	\node [int, below of=iint] (int) 
		{\large 定积分\\
		\small$\ds\int_a^bf(x)\d x$
	};
	\node [mint, right of = int, xshift=2.5cm] (NL)
% 		{\color{red}Netwon-Lebniz\it 公式};
		{\small $\dint_A^Bf(\bm{x})\d\bm{x}=F(\bm{x})|_A^B$};
% 	\node [right of = int, xshift = 3cm, minimum width=5cm,
% 		minimum height = 1.5cm] (NL)
% 		{\small$\ds\int_a^bf(x)\d x=F(b)-F(a)$};
	\begin{scope}[xshift=5.5cm]
		\node [int] (sint) 
			{\large 曲面积分
			};
		\node [int, below of = sint] (cint)
			{\large 曲线积分
			};
	\end{scope}
	\begin{scope}[xshift=11cm,yshift=1.6cm]
		\node [mint] (sint1) 
			{\small$\ds\iint\limits_{\Sigma}f(x,y,z)\d S$};
		\node [mint, below of = sint1, yshift=1.5cm] (sint2)
			{\small$\ds\iint\limits_{\Sigma}f(x,y,z)\d x\d y$};
		\node [mint, below of = sint2, yshift=2cm] (cint1)
			{\small$\ds\int_Lf(\bm{x})\d s$};
		\node [mint, below of = cint1, yshift=1.5cm] (cint2)
			{\small$\ds\int_LP\d x+Q\d y+R\d z$};
	\end{scope}
	\path (iiint) edge[arrow,<->]
		node [text width=3em]{\it 柱坐标 球坐标} 
		node[swap, text width=3em,text centered] {“1+2”  “2+1”} (iint)
		edge[arrow,<->] 
		node {\color{red}Gauss\it 公式}(sint);
	\path (iint) edge[arrow,<->] 
		node[swap, text width=3em,text centered] {“1+1”  \it 定限} 
		node[text width=4em] {\it 极坐标} (int)
		edge[arrow,<->] 
		node {\color{red}Green\it 公式} (cint);
	\path (sint) edge [arrow,->] 
		node [sloped,pos=0.2,swap]{\it 投影/曲面方程}(iint)
		edge[arrow,<->] 
		node {\color{red}Stokes\it 公式}(cint)
		edge[arrow] node[sloped,pos=0.8]{\it 对面积} (sint1)
		edge[arrow] node[sloped,pos=0.2]{\it 对坐标}(sint2);
	\path (cint) edge [arrow, ->] 
		node [sloped,near start,swap]{\it 曲线参数化} (int)
		edge [arrow] node[sloped,pos=0.8]{\it 对弧长}(cint1)
		edge [arrow] node[sloped,pos=0.2]{\it 对坐标}(cint2);
	\path (sint1) edge[arrow, <->] 
		node {\color{red}$\d\sigma_{xy}=|\cos\gamma|\d S$} (sint2);
	\path (cint1) edge[arrow, <->] 
		node {\color{red}$\d s=|\bm{r}'(t)|\d t$} (cint2);
	\path (cint2) edge [arrow, ->, out=-90, in=0, dashed] 
		node {\it 积分与路径无关条件} (NL);
	\path (int) edge [arrow, ->, dashed]
		node {{\color{red}Netwon-Lebniz}}
		node [swap]{\color{red}\it 公式} (NL);
% 	\path (int) edge[<->] (NL);
\end{tikzpicture}