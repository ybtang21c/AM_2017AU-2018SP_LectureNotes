\setcounter{chapter}{5}

\chapter{定积分及其应用}

\section{定积分的概念与性质}

\subsection{定义}

{\bf\b 曲边梯形的面积:}由$x$轴,$x=a,x=b$以及曲线$y=f(x)$
所围平面区域的面积

\begin{center}
	\resizebox{!}{5.2cm}{\includegraphics{./images/ch6/integral.pdf}}
	
	$$S=\dint_a^bf(x)\d x$$
\end{center}

{\bf 教材-例1:}求由曲线$y=x^2$以及直线$x=0,x=1$和$y=0$所围成的曲边梯形的面积。

{\bf 定义:}函数$y=f(x)$在区间$[a,b]$上(Riemann)可积:
\ps{重点理解构造思想:{\b\bf “分割取近似,做和求极限”}}
$$\dint_a^bf(x)\d x=\lim\limits_{\lambda\to
0}\sum\limits_{k=1}^nf(\xi_k)\Delta x_k$$
其中,对区间$[a,b]$做任意划分,$\xi_k$取自第$k$个区间的任一点,$\Delta_k$
为第$k$个区间的长度,$\lambda$为最大的区间的长度,如果以上右侧极限对任意划分和
任意点的取法都一样,则称$f(x)$在$[a,b]$上(Riemann){\it 可积},该极限的值称为
$f(x)$在$[a,b]$上的{\it 定积分}

{\bf 注:}
\begin{enumerate}[(1)]
  \setlength{\itemindent}{1cm}
  \item {\b 定积分是一个数值,不定积分是一个函数}
  \item 两个“任意”
  \begin{itemize}
    \item 对$[a,b]$的任意分法
    \ps{事实上,等分亦可,因为后一个“任意”已经足够了,参见同济教材的定义}
    \item {\b 给定分法,$\xi_k$的取法应该是任意的}
  \end{itemize}
  \item $\lambda\to 0$用于保证分割得足够“细”
\end{enumerate}

{\bf 定理6.1.1:}
\begin{enumerate}[(1)]
  \setlength{\itemindent}{1cm}
  \item $f(x)$在$[a,b]$上可积,则一定在$[a,b]$上有界 
  \item $f(x)$在$[a,b]$上连续,则一定在$[a,b]$上可积 
\end{enumerate}

{\bf 注:}分段连续(只有有限个第一类间断点)的函数是可积的

{\bf 教材-例2:}证明Dirichlet函数在任意区间$[a,b]$上不可积。

\subsubsection{【用定积分的定义计算极限】}

{\bf 教材-习题7:}将下列求和式表示成定积分\ps{\b 常考的内容,注意和
Stolz定理和夹逼定理适用的问题作区分!}
\begin{enumerate}[(1)]
  \setlength{\itemindent}{1cm}
  \item $\limn\df 1n\left[\sin\df{\pi}{n}+\sin\df{2\pi}{n}+\ldots
  +\sin\df{(n-1)\pi}{n}\right]$ 
  \item $\limn\left(\df{1}{n^2}+\df{2}{n^2}+\ldots+\df{n-1}{n^2}\right)$
\end{enumerate}

{\bf 例:}求$\limn\sum\limits_{k=1}^n\df{k}{n^3}\sqrt{n^2-k^2}$。

{\bf 例:}利用定积分定义证明以下极限:
$$\limn\cos\df x2\cos\df x4\cos\df x8\cdot\cdots\cdot\cos\df x{2^n}
=\df{\sin x}x,$$
并证明Vieta's Formula(韦达公式)
$$\df2\pi=\df{\sqrt2}2\cdot\df{2+\sqrt2}2\cdot\df{\sqrt{2+\sqrt{2+\sqrt2}}}2
\cdot\cdots$$

[提示]:
\begin{align}
	&\cos\df x2\cos\df x4\cos\df x8\cdots\cos\df x{2^n}\notag\\
	&=\df12\left(\cos\df34x+\cos\df14x\right)\cos\df x8\cdots\cos\df
	x{2^n}\notag\\
	&=\df1{2^2}\left(\cos\df78x+\cos\df58x+\cos\df38x+\cos\df18x\right)
	\cos\df x{16}\cdots\cos\df x{2^n}\notag\\
	&=\ldots\notag\\
	&=\df1{2^{n-1}}\left(\cos\df{2^n-1}{2^n}x+\cos\df{2^n-3}{2^n}x+
	\ldots+\cos\df3{2^n}x+\cos\df1{2^n}x\right)\notag\\
	&=\dint_0^1\cos(xt)\d t=\df{\sin x}x\notag
\end{align}

{\bf 例:}$G_n=\sqrt[n]{(n+1)(n+2)\cdots(n+n)},(n=1,2,\ldots)$,求
$\limn\df{G_n}n$.

[提示]:
$$\ln\df{G_n}n=\limn\df1n\sum\limits_{k=1}^n\ln\left(1+\df kn\right)
=\dint_0^1\ln(1+x)\d x=2\ln2-1$$

\subsection{几何意义}

\begin{center}
	\resizebox{!}{3.12cm}{\includegraphics{./images/ch6/IS1.pdf}} 
	\resizebox{!}{3.12cm}{\includegraphics{./images/ch6/IS2.pdf}} 
	\resizebox{!}{3.12cm}{\includegraphics{./images/ch6/IS3.pdf}}
	
	{\b\it “带符号的面积” }
\end{center}

{\bf 约定:}
\begin{enumerate}[(1)]
  \setlength{\itemindent}{1cm}
  \item $\dint_a^af(x)\d x=0$
  \item $\dint_b^af(x)\d x=-\dint_a^bf(x)\d x$ 
\end{enumerate}

\begin{shaded}

{\bf 例:}$f(x)\in C[0,1]$且严格单调递增,$f(0)=0,f(1)=1$,已知
$\dint_0^1f(x)\d x=\df23$,求$\dint_0^1f^{-1}(y)\d y$

[提示]:作图易知
$$\dint_0^1f(x)\d x+\dint_0^1f^{-1}(y)\d y=1$$

\begin{center}
	\resizebox{!}{5cm}{\includegraphics{./images/ch6/iff-1.pdf}}  
\end{center}

{\bf 例:}$a>1$,证明:$\dint_1^a\ln x\d x+\dint_0^{\ln a}e^y\d y=a\ln a$

[提示]:与上题同理。

{\bf 命题:}$f(x)\in C[a,b]$严格单调增加,则
$$\dint_a^bf(x)\d x+\dint_{f(a)}^{f(b)}f^{-1}(y)\d y=bf(b)-af(a)$$
例如:
$$\dint_0^{\pi/2}\sin x\d x+\dint_0^1\arcsin x\d x=\df{\pi}2$$
\begin{center}
	\resizebox{!}{5cm}{\includegraphics{./images/ch6/iff-1ab.pdf}}  
\end{center}

{\bf 例:}$f(x)\in C[0,+\infty)$且严格单调递增,$f(0)=0,a>0,b>0$,则
成立如下的Young{\it 不等式}
$$ab\leq\dint_0^af(x)\d x+\dint_0^bf^{-1}(y)\d y$$
\begin{center}
	\resizebox{!}{4cm}{\includegraphics{./images/ch6/iff-ab.pdf}}  
\end{center}
利用该不等式,可以证明Minkowski{\it 不等式}:设$p>1,q>1,\df1p+\df1q=1$,则
$$ab\leq\df{a^p}p+\df{b^q}q$$
[提示]:令$f(x)=x^{p-1}$,则$x=y^{q-1}$(注意到$\df1{p-1}=q-1$)

{\bf 例:}计算$\dint_0^1\left(\sqrt[3]{1-x^7}-\sqrt[7]{1-x^3}\right)\d x$

[提示]:$y=\sqrt[3]{1-x^7}$和$x=\sqrt[7]{1-y^3}$互为反函数!

\end{shaded}

\subsection{基本性质}

\begin{enumerate}[(1)]
  \setlength{\itemindent}{1cm}
  \item {\bf 线性性:}
  $$\dint_a^b[\alpha
  f(x)+\beta g(x)]\d x=\alpha\dint_a^bf(x)\d x+\beta\dint_a^bg(x)\d x$$
  \item {\bf 区间可加性:}
  $$\dint_a^bf(x)\d x=\dint_a^cf(x)\d x+\dint_c^bf(x)\d x$$
  \item {\bf 保号性:}\ps{不定积分不具有保号性!}
  \begin{itemize}
    \item $f(x)$在$[a,b]$上可积,且$f(x)\geq 0$,则$\dint_a^bf(x)\d x\geq 0$ 
    \item $f(x)$在$[a,b]$上连续,非负且不恒为零,则$$\dint_a^bf(x)\d x> 0$$ 
    \item {\bf 保序性:}设$f(x),g(x)$在$[a,b]$上可积,且$f(x)\leq g(x)$,则
    $$\dint_a^bf(x)\d x\leq \dint_a^bg(x)\d x$$
    \item {\bf 绝对值不等式:}$f(x)$在$[a,b]$上可积,则
    $${\b \dint_a^bf(x)\d x\leq\dint_a^b|f(x)|\d x}$$ 
    \item {\bf 积分估值:}$f(x)$在$[a,b]$上可积,且$m\leq f(x)\leq M$,则
    $$m(b-a)\leq\dint_a^bf(x)\d x\leq M(b-a)$$
  \end{itemize}
\end{enumerate}

\subsection{定积分中值定理}

{\bf 定理6.1.2:}若函数$f(x)$在区间$[a,b]$上连续,
则在$[a,b]$上至少存在一点$\xi$,使得
$$\dint_a^bf(x)\d x=f(\xi)(b-a)$$

{\it\b $f(\xi)$相当于函数$f(x)$
在$[a,b]$上的均值(注意不是中值)},这一概念在概率统计中得到了广泛应用!

{\bf 注:}由连续函数的变限积分的可导性(参见6.2.2节),利用Lagrange中值
定理可以证明定积分中值定理,因此{\it 定积分中值定理与微分中值定理本质上是一样的!}

{\bf 例:}计算如下极限
$$\limx{+\infty}\sqrt[3]x\dint_x^{x+1}\df{\sin t}{t+\cos t}\d t$$

[提示]:不妨设$x>1$,由定积分中值定理,存在$\xi\in(x,x+1)$
$$\left|\sqrt[3]x\dint_x^{x+1}\df{\sin t}{t+\cos t}\d t\right|
=\sqrt[3]x\df{|\sin\xi|}{|\xi+\cos\xi|}\leq\sqrt[3]x\df{1}{x-1}
\to0\;(x\to\infty)$$


\section{微积分基本公式与定积分的计算}

\subsection{微积分基本定理}

{\it 在Newton和Leibniz之前,微分和积分的许多个别结果已经得到,但只有当
他们发现了这个公式之后,微积分才统计成了一个整体!}

{\bf 定理6.2.1}({\it 微积分基本定理})设函数$f(x)$在区间$[a,b]$上可积,$F(x)$是
$f(x)$在$[a,b]$上的一个原函数,则
$$\dint_a^bf(x)\d x=F(b)-F(a).$$

{\bf 教材-例1:}计算定积分
\begin{enumerate}[(1)]
  \setlength{\itemindent}{1cm}
  \item $\dint_{0}^{1}x^2 \d x$
  \item $\dint_{-1}^{\sqrt 3}\df 1{1+x^2}\d x$
  \item $\dint_{-1}^{-2}\df 1x\d x$
\end{enumerate}

{\bf 例:}如下的计算过程有什么错误?
$$\dint_0^{3\pi/4}\df{\sin x}{1+\cos^2x}\d x
=\left.\arctan(\sec x)\right|_0^{3\pi/4}=-\arctan\sqrt2-\df{\pi}4$$
正确的方法
$$\dint_0^{3\pi/4}\df{\sin x}{1+\cos^2x}\d x
=-\dint_0^{3\pi/4}\df{\d\cos x}{1+\cos^2x}=\arctan\df{\sqrt2}2
+\df{\pi}4$$

{\bf 注:}{\b 定积分计算中使用换元法时,必须同时将积分上下限更改为对应的值,且必须保证
变换后的变量是连续变化的!!}

{\bf 例:}如下的计算过程有什么错误?
$$\dint_0^{\pi}\df{\d x}{2+\cos2x}=\df1{\sqrt3}\arctan
\left(\df{\tan x}{\sqrt3}\right)_0^{\pi}=0$$
错误:原函数在$x=\pi/2$处有间断点!正确的过程
\begin{align}
	\dint_0^{\pi}\df{\d x}{2+\cos2x}&=\dint_0^{\pi}\df{\d x}
	{3-2\sin^2x}=\dint_0^{\pi}\df{\cot^2x\d x}{3\csc^2x-2}\notag\\
	&=-\dint_0^{\pi}\df{\d\cot x}{3\cot^2x+1}=\df{\pi}{\sqrt3}\notag
\end{align}

{\bf 注:}
\begin{itemize}
  \setlength{\itemindent}{1cm}
  \item {\b 可积函数未必有原函数},例如,有第一类间断点的函数
  \item {\b 有原函数的函数未必可积},例如
  $$f(x)=\left\{\begin{array}{ll}
  x^2\sin\df1x,& x\ne0,\\ 0,& x=0
  \end{array}\right.$$
\end{itemize}

{\bf 例:}设$f(x)$满足方程
$$f(x)=3x-\sqrt{1-x^2}\dint_0^1f\,^2(x)\d x,$$
求$f(x)$。

{\bf 例:}求$F(x)=\dint_0^x[\,t\,]\d t,\;(x>0)$的表达式,其中$[\,x\,]$为下取整函数。

\subsection{变限积分}

{\bf 定理6.2.2:}若$f(x)$在$[a,b]$上可积,则{\it 变上限积分}
$$\Phi(x)=\dint_a^xf(t)\d t$$
在$[a,b]$上连续。

{\bf 定理6.2.3:}若$f(x)$在$[a,b]$上连续,则$\Phi(x)=\dint_a^xf(t)\d t$在$[a,b]$可导,且
$$\Phi'(x)=f(x)$$

{\bf 注:}若$f(x)$连续,则变上限积分$\Phi(x)=\dint_a^xf(t)\d t$是$f(x)$的一个原函数

{\bf 思考:}为什么$\Phi(x)$是$f(x)$的“一个”原函数?

$${\b\left[\dint_{\varphi(x)}^{\psi(x)}f(t)\d t\right]'_x
	=f[\psi(x)]\psi'(x)-f[\varphi(x)]\varphi'(x)}$$

{\bf 教材-例6:}求下列函数的导函数
\begin{enumerate}[(1)]
  \setlength{\itemindent}{1cm}
  \item $y=\dint_0^{x}e^{-t^2}\d t$
  \item $y=\dint_0^{x^2}\sqrt{1+t^4}\d t$
  \item $y=\dint_{-x}^{\sqrt x}\sin t^2\d t$
\end{enumerate}

{\bf 教材-例13:}设$f(x)$在$[0,+\infty)$内连续,且$f(x)>0$,证明:
$$F(x)=\df{\dint_0^xtf(t)\d t}{\dint_0^xf(t)\d t}$$
在$(0,+\infty)$内单调递增。

{\bf 教材-例14:}设$f(x)$在$[0,+\infty)$上可导,$f(0)=0$,且存在反函数$g(x)$,已知
$$\dint_0^{f(x)}g(t)\d t=(x-1)e^x+x^2+1,$$
求$f(x)$。

{\bf 例:}设$\limx{0}\df{1}{bx-\sin
x}\dint_0^x\df{t^2}{\sqrt{a+t}}\d t=1$,求$a,b$。

{\bf 例:}设$\phi(x)=\dint_a^x(x+t)\varphi(t)\d t$,其中$\varphi(x)$可导,求
$\phi''(x)$。

{\bf 思考:}若$\phi(x)=\dint_a^x(x+t)\varphi(x+t)\d t$,结果如何?

{\bf 例:}$x>0$时,$\cos x\leq 1$,该式从$0$到$x$积分可得
$$\sin x\leq x$$
进而
$$1-\cos x<\df{x^2}2$$
$$\sin x>x-\df{x^3}6$$
$$\ldots$$
最终
$$\left|\sin x-\sum\limits_{k=0}^n(-1)^k\df{x^{2k+1}}{(2k+1)!}\right|
\leq\df{|x|^{2n+3}}{(2n+3)!}$$
$$\left|\cos x-\sum\limits_{k=0}^n(-1)^k\df{x^{2k}}{(2k)!}\right|
\leq\df{|x|^{2n+2}}{(2n+2)!}$$

\begin{shaded}
	{\bf 积分中值定理的证明及其应用}
	
	利用变限积分的性质和Cauchy中值定理,可以证明定积分中值定理。

	{\bf 例:}设$f(x)$在$[a,b]$上连续,且$g(x)$在$[a,b]$上可积且不变号,证明:存在
	$\xi\in(a,b)$,使得
	$$\dint_a^bf(x)g(x)\d x=f(\xi)\dint_a^bg(x)\d x$$
	
	[提示]:令$F(t)=\dint_a^tf(x)\d x,\;G(t)=\dint_a^tg(x)\d x$,则
	由Cauchy中值定理
	$$\df{F(b)-F(a)}{G(b)-G(a)}=\df{F'(\xi)}{G'(\xi)}=f(\xi)$$
	
	{\bf 例:}计算极限$\limn f(n)\sin\df 1n$,其中
	  $$f(x)=\dint_x^{x^2}\left(1+\df 1{2t}\right)^t\sin\df{1}{\sqrt t}\d t$$
	
	[提示]:
	\begin{align}
		\limn f(n)\sin\df 1n&=\limn\df1n\dint_n^{n^2}\left(1+\df
		1{2t}\right)^t\sin\df{1}{\sqrt t}\d t\notag\\
		&=\limn\df1n\left(1+\df1{2\xi}\right)^{\xi}\dint_n^{n^2}
		\sin\df1{\sqrt t}\d t\quad(\xi\in(n,n^2))\notag\\
		&=2e^{1/2}\limn\df1n\dint_{\sqrt n}^n2x\sin\df1x\d x\notag\\
		&=2e^{1/2}\limn\eta\sin\df1{\eta}\df{n-\sqrt n}n\quad(\eta\in(\sqrt
		n,n))\notag\\
		&=2e^{1/2}\notag
	\end{align}
	
	{\bf 例:}设$f(x)$在$[0,2\pi]$上连续,证明:
	$$\limn\dint_0^{2\pi}f(x)|\sin nx|\d x=\df 2{\pi}\dint_0^{2\pi}f(x)\d x$$
	
	{\bf [提示]:}
	\begin{align}
	\dint_0^{2\pi}f(x)|\sin nx|\d x
	&=\sum\limits_{k=1}^n\dint_{2(k-1)\pi/n}^{2k\pi/n}f(x)|\sin nx|\d x\notag\\
	&=\sum\limits_{k=1}^nf(\xi_k)\dint_{2(k-1)\pi/n}^{2k\pi/n}|\sin nx|\d x\notag
	\end{align}
	and
	$$\dint_{2(k-1)\pi/n}^{2k\pi/n}|\sin nx|\d x=\df4n$$
	hence
	$$\dint_0^{2\pi}f(x)|\sin nx|\d x=\df4n\sum\limits_{k=1}^nf(\xi_k)
	=\df2{\pi}\left(\sum\limits_{k=1}^nf(\xi_k)\df{2\pi}n\right)$$
\end{shaded}

\subsection{定积分的计算}

{\bf 定理6.3.3}(换元法)
设$f(x)$在$[a,b]$上连续,函数$x=\varphi(t)$满足:
\begin{enumerate}[(1)]
  \setlength{\itemindent}{1cm}
  \item $\varphi(\alpha)=a,\;\varphi(\beta)=b$;
  \ps{换元后一定要将积分上下限做对应的修改}
  \item $\varphi(t)$在$[\alpha,\beta]$上连续可导,且$\varphi(t)\in[a,b]$,则
\end{enumerate}
$${\dint_a^bf(x)\d
x=\dint_{\alpha}^{\beta}f[\varphi(t)]\varphi'(t)\d t}$$

{\bf 定理6.3.4}(分部积分法)
$${\dint_a^bu(x)\d v(x)=u(x)v(x)|_a^b-\dint_a^bv(x)\d u(x)}$$

{\bf 例:}计算积分
\begin{enumerate}[(1)]
  \setlength{\itemindent}{1cm}
  \item  $\dint_9^4\df 1{\sqrt x}\d x$ 
  \item $\dint_0^1\sqrt{(1-x^2)^3}\d x$ 
  \item $\dint_0^{\ln 2}\sqrt{1-e^{-2x}}\d x$
  \item $\dint_0^{2\pi}\sqrt{1+\cos\theta}\d\theta$ 
  \item $\dint_0^{\pi}(x\sin x)^2\d x$ 
  \item $\dint_0^{n\pi}x|\sin x|\d x$ 
  \item $\dint_0^1\df{\ln(1+x)}{1+x^2}\d x$
\end{enumerate}

{\bf 例:}设$f(x)=\dint_1^x\df{\ln t}{1+t}\d t$,求$f(x)+f
\left(\df1x\right)$。

{\bf 例:}设$f(\pi)=2$,$\dint_0^{\pi}[f(x)+f''(x)]\sin x\d x=5$,求$f(0)$。

\begin{shaded}
	{\bf 用定积分的性质证明不等式}
	
	{\bf 教材-习题10}({\it Schwarz积分不等式})设$f(x),g(x)$均在$[a,b]$上连续,证明:
	$$\left[\dint_a^bf(x)g(x)\d x\right]^2
	\leq\dint_a^bf\,^2(x)\d x\dint_a^bg^2(x)\d x$$
	
	[提示]:不妨设$\dint_a^bf^2(x)\d x>0$,对任意$t\in\mathbb{R}$
	$$\dint_a^b(tf-g)^2\d x=t^2\dint_a^bf^2\d x-2t\dint_a^bfg\d x
	+\dint_a^bg^2\d x\geq 0$$
	故必有
	$$\Delta=\left(\dint_a^bfg\d x\right)^2-\dint_a^bf^2\d x\dint_a^b
	g^2\d x<0$$
	若$\dint_a^bf^2\d x=\dint_a^b g^2\d x=0$,则
	$$\left|\dint_a^bfg\d x\right|\leq\dint_a^b|fg|\d x
	\leq\dint_a^b\df12(f^2+g^2)\d x=0$$
	
	{\bf 注:}类似地方法可证明{\it Cauchy不等式}
	$$\left(\sum\limits_{i=1}^na_ib_i\right)\leq
	\sum\limits_{i=1}^na_i^2\sum\limits_{i=1}^nb_i^2$$
	
	{\bf 例:}$f(x)\in C[a,b]$,且$f(x)>0$,则
	$$\dint_a^bf(x)\d x=\dint_a^b\df1{f(x)}\d x\geq(b-a)^2$$
	
	{\bf 例:}设$f(x)$在$[a,b]$上连续可微,$f(a)=0$,证明:
	$$\dint_a^b[f(x)]^2\d x\leq\df{(b-a)^2}{2}\dint_a^b[f\,'(x)]^2\d x$$
\end{shaded}

{\bf 例:}证明:
$$\dint_1^af\left(x^2+\df{a^2}{x^2}\right)\df{\d x}{x}
=\dint_1^af\left(x+\df{a^2}{x}\right)\df{\d x}{x}$$

[提示]:令$y=\df ax$,则
$$\mbox{left}=\df12\dint_1^{a^2}f\left(y+\df{a^2}y\right)\df{\d y}y$$
令$y=\df{a^2}x$,则
$$\dint_a^{a^2}f\left(x+\df{a^2}x\right)\df{\d x}x
=\dint_1^af\left(y+\df{a^2}y\right)\df{\d y}y$$

\subsection{定积分的一些特殊计算方法}

\subsubsection{【对称区间上的定积分】}

{\bf 命题:}$f(x)$连续,则
$$\dint_0^af(x)\d x=\dint_0^af(a-x)\d x$$
利用该结论可以较为简便地计算如下的积分:
\begin{enumerate}[(1)]
  \setlength{\itemindent}{1cm}
  \item $\dint_0^a\df{f(x)\d x}{f(x)+f(a-x)}$
  \item $\dint_0^3\df{\sqrt x\d x}{\sqrt x+\sqrt{3-x}}$
  \item $\dint_0^1\df{x\d x}{e^x+e^{1-x}}$
  \item $\dint_0^{\pi/2}\df{\sin x\d x}{\sin x+\cos x}$
  \ps{在$[0,\pi/2]$上的积分中$\cos x$和$\sin x$可以相互替换}
  \item $\dint_0^{\pi/2}\df{\sin^nx\d x}{\sin^nx+\cos^nx}$
  \item $\dint_0^{\pi}\df{\cos x}{\sqrt{a^2\sin^2x+b^2\cos^2x}}\d
  x\;\;(a^2+b^2\ne 0)$
  \item $\dint_0^{\pi}\df{a^n\sin^2x+b^n\cos^2x}
  {a^{2n}\sin^2x+b^{2n}\cos^2x}\d x$
  \item $\dint_0^{\pi}\df{x\d x}{1+\cos^2x}$
  \item $\dint_0^{\pi}\df{x\sin x\d x}{1+\cos^2x}=\df{\pi^2}4$
  \item $\dint_0^1\df{\ln(1+x)\d x}{1+x^2}=\dint_0^{\pi/4}\ln(1+\tan x)\d x$
  
  [提示]:
  $$\ln\left[1+\tan\left(\df{\pi}4-t\right)\right]=\ln2-\ln(1+\tan t)$$
  故$\ln(1+\tan t)-\df{\ln2}2$是以$\df{\pi}8$为中心的奇函数。于是$I=\df{\pi}8\ln2$
\end{enumerate}

{\bf 例:}利用对称性计算下列各题:
\begin{enumerate}[(1)]
  \setlength{\itemindent}{1cm}
  \item $\dint_{-\pi/2}^{\pi/2}\df{\sin^2x}{1+e^{-x}}\d x$
  \item $\dint_{-1}^1\df{x+\cos x}{1+\sin^2x}\d x$
  \item $\dint_{0}^{\pi}|\cos x|\sqrt{1+\sin^2x}\d x$
  \item $\dint_{-2}^2x\ln(1+e^x)\d x$
\end{enumerate}

\subsubsection{【周期函数的定积分】}

$$\dint_a^{a+T}f(x)\d x=\dint_0^{T}f(x)\d x$$

\subsubsection{【用几何意义计算定积分】}

{\bf 例:}$2\dint_{-1}^1\sqrt{1-x^2}\d x=
\dint_{-1}^1\df{\d x}{\sqrt{1-x^2}}$

[提示]:左边是面积,右边为周长

\begin{shaded}
	{\bf 勾股定理的推广}
	
	{\bf 命题:}若$a^2+b^2=c^2$,$f(x)$非负连续,
	$$g(x)=\df acf\left(\df cax\right),\quad 
	h(x)=\df bcf\left(\df cbx\right)$$
	则
	$$\dint_0^xf(x)\d x=\dint_0^xg(x)\d x+\dint_0^xh(x)\d x$$
	\begin{center}
		\resizebox{!}{5cm}{\includegraphics{./images/ch6/ggNew.pdf}}  
	\end{center}
\end{shaded}

\section{定积分的应用}

{\bf 定积分的构造思路──微元法:}

\begin{enumerate}
  \item {\bf 分割:}沿$x$轴方向分割曲边梯形 
  \item {\bf 取近似:}用小矩形的面积$\Delta S$近似小曲边梯形面积 
  \item {\bf 做和:}求所有小矩形的面积总和$\sum \Delta S$ 
  \item {\bf 求极限:}$\sum \Delta S\to S$
\end{enumerate}

\begin{center}
	\resizebox{!}{5.5cm}{\includegraphics{./images/ch6/sumSSS.pdf}}
\end{center}

\subsection{几何应用}

\subsubsection{【面积】}

{\bf 例:}计算两条抛物线$y^2=x$和$y=x^2$所围图形的面积。

{\bf 例:}计算两条抛物线$y^2=2x$与直线$y=x-4$所围图形的面积。

{\bf 例:}推导圆和扇形的面积公式。

\subsubsection{【极坐标下的面积】}

{\bf 例:}计算Archimedes螺线$\rho=a\theta\;(a>0)$
相应于$\theta$从$0$变到$2\pi$的一段弧与极轴所围的图形面积。

\begin{center}
	\resizebox{!}{4cm}{\includegraphics{./images/ch6/Achimedes.pdf}}
\end{center}

{\bf 例}(老羊三问题)
\begin{enumerate}[(1)]
  \setlength{\itemindent}{1cm}
  \item 半径为$a$的圆形水池,一只老羊被长为$ka(0<k\leq2)$的绳子拴在水池边缘,求
  老羊能吃到的草地面积
  \item 如果将水池围起来,老羊吃草时,绳子因为被绕在栅栏上而不能穿过水池,再求面积
  \item 在草场中央围一个半径为$a$的区域,要使内外的老羊吃到的草面积一样,$k$如何取
\end{enumerate}

[提示]:(1)
$$A=\pi k^2a^2-2\dint_{\arccos\frac k2}^{\pi}\df12(2a\cos\varphi)^2
\d\varphi$$
$$A=a^2\left[(k^2-1)\pi+(2-k^2)\arccos\df k2+\df k2
\sqrt{4-k^2}\right]$$
(2)
$$A=\df{\pi k^2a^2}2+2\dint_0^k\df12(ka-a\varphi)^2\d\varphi=k^2a^2
\left(\df{\pi}2+\df k3\right)$$
以上的$\varphi$为小园的圆心角
(3)$k=1.26$

\subsubsection{【平面曲线的弧长】}

{\bf 例:}求以下曲线的弧长 
\begin{enumerate}[(1)]
  \setlength{\itemindent}{1cm}
  \item $y=\df 23x^{3/2}\;(a\leq x\leq b)$ 
  \item $x=a(t-\sin t),y=a(1-\cos t)\;(0\leq t\leq 2\pi)$ 
  \item $\rho=a\theta\;(a>0,\;0\leq t\leq 2\pi)$
\end{enumerate}

\subsubsection{【旋转体的体积】}

圆盘法:曲线$y=f(x)\;(x\in[a,b])$与$x=a,x=b,y=0$所围区域绕$x$轴旋转一周:
$$V=\dint_a^b\pi[f(x)]^2\d x$$

{\bf 例:}求椭圆$\df{x^a}{a^2}+\df{y^2}{b^2}=1$绕$x$轴旋转一周所得立体的体积。

圆柱壳法:$y=f(x)\;(x\in[a,b])$与$x=a,x=b,y=0$所围区域绕$y$轴旋转一周
$$V=2\pi\dint_a^bx|f(x)|\d x$$

{\bf 例:}计算由圆滚线$$x=a(t-\sin t),y=a(1-\cos t)\;(0\leq t\leq 2\pi)$$
和直线$y=0$所围图形分别绕$x$轴、$y$轴旋转一周所得立体的体积。

\subsubsection{【旋转体的表面积】}
曲线$y=f(x)\;(x\in[a,b])$绕$x$轴旋转一周:
$$A=2\pi \dint_a^by\d s=2\pi \dint_a^by\sqrt{1+(y')^2}\d x$$

[提示]:圆台的侧面积
$$A=\pi(r_1+r_2)l$$
其中$r_1,r_2$为上下底的半径,$l$为斜高(或母线长度)
$$\Delta A=2\pi(y+(y+\Delta y))\sqrt{(\Delta x)^2+(\Delta y)^2}
=2\pi y\sqrt{1+(y')^2}\Delta x+\circ(\Delta x)$$

\subsubsection{【已知截面积的立体体积】}

{\bf 例:}一平面经过半径为$R$的圆柱体的底面圆心,并与底面交角为$\alpha$
(如图所示),求该平面所截圆柱体

\begin{center}
	\resizebox{!}{4.5cm}{\includegraphics{./images/ch6/bucket.pdf}}
\end{center}

{\bf 例:}半径不等的两个木质球体,分别中间凿去一个以直径为轴的正圆柱体以及柱体下方的球冠,
剩下的环状部分高度均为$h$,问哪一个环状物体的体积更大。

[提示]:
$$V=\dint_{\sqrt{R^2-h^2}}^R2\pi x\sqrt{R^2-x^2}\d x=\df23\pi h^3$$
结果表明:剩余部分的体积与球半径无关!

{\bf 例:}试推导球的体积公式和球面上一块面积为$S$的球心锥的体积公式($rS/3$)

{\bf 例:}开口容器内的水在单位时间内的蒸发量与水的表面积成正比,证明:水的深度以常速率
下降且与容器形状无关。

[提示]:设水的表面积为$S$,则
$$\Delta V\approx S\Delta h\Rightarrow
\df{\Delta V}{\Delta t}\approx S\df{\Delta h}{\Delta t}$$
即
$$\df{\d V}{\d t}=S\df{\d h}{\d t}$$

\subsection{物理应用}

\subsubsection{【平面物体的质心】}
$y=f(x)(a<x<b)$密度均匀,其质心的横坐标为
$$\bar{x}=\df{\dint_a^bxf(x)\d x}{\dint_a^bf(x)\d x}$$

{\bf 例:}求右半圆的质心坐标。

{\bf 例:}$f(x)\in C[a,b]$且单调递增,证明:
$$\dint_a^bxf(x)\d x\geq\df{a+b}2\dint_a^bf(x)\d x$$
[提示]:物理意义:递增函数的形成的曲边梯形质心位于右侧。

证1:由$f(x)$单增,
$$\left(x-\df{a+b}2\right)\left[f(x)-f\left(
\df{a+b}2\right)\right]\geq0$$
两边积分即得。

注:$f\left(\df{a+b}2\right)\dint_a^b\left(x-\df{a+b}2\right)=0$

证2:记$c=\df{a+b}2$
\begin{align}
	\dint_a^b&(x-c)f(x)\d x\notag\\
	&=\left(\dint_a^c+\dint_c^b\right)(x-c)f(x)\d x\notag\\
	&=f(\xi_1)\dint_a^c(x-c)\d x+f(\xi_2)\dint_c^b(x-c)\d x\notag\\
	&=[f(\xi_1)-f(\xi_2)]\df{(b-a)^2}2\geq0
\end{align}

\subsubsection{【变力沿直线做功】}

{\bf 例:}一圆柱形的水桶高$5m$,底半径$3$m,桶内装满水,问要将桶内的水全部抽出,
需要做多少功(设重力加速度$g=10N/kg$)。

{\bf 例:}在由电量为$+q$的电荷产生的电场中,沿某一轴向将一个单位正电荷由距离
$a$移动到距离$b$的位置,问在该过程中,电场力共对该电荷做了多少功?

{\bf 电场力:}
$$F=k\df{q_1q_2}{r^2}$$

\subsubsection{【水压】}

{\bf 例:}某个圆柱形油桶底半径为$R$,所装油密度为$\rho$,
现将其装满油后横放,求其端面承受的压力。

\subsubsection{【万有引力】}

{\bf 例:}设有一长度为$l$,线密度为$\mu$的均匀细直棒,在其中垂线上距离棒$a$
处有一质量为$m$的质点$M$,求该棒对质点$M$的引力。

[提示]:
\begin{align}
	F&=-\dint_{-\frac l2}^{\frac l2}\df{Gam\mu}{(a^2+y^2)^{\frac32}}\d y
	=-\dint_{-\arctan\frac l{2a}}^{\arctan\frac l{2a}}\df{Gam\mu}{(a^2+(a\tan
	t)^2)^{\frac32}}\d (a\tan t)\notag\\
	&=-\df{Gm\mu}a\dint_{-\arctan\frac l{2a}}^{\arctan\frac
	l{2a}}\cos t\d t=-\df{Gm\mu}a\left.\sin t\right|_{-\arctan\frac
	l{2a}}^{\arctan\frac l{2a}}\notag\\
	&=-\df{2Gm\mu}a\sin\left(\arctan\df l{2a}\right)\notag
\end{align}
注意到$\sin x=\df{\tan x}{\sec x}=\df{\tan x}{\sqrt{1+\tan^2 x}}$,故
$$\sin\left(\arctan\df l{2a}\right)=\df{\frac{l}{2a}}
{\sqrt{1+\left(\frac{l}{2a}\right)^2}}=\df{l}{\sqrt{4a^2+l^2}}$$
代回前式即得$F=-\df{2Gm\mu l}{a\sqrt{4a^2+l^2}}$。

\section{反常积分}

{\bf 定积分存在的条件:}

$${I=\dint_a^bf(x)\d x}$$

\begin{enumerate} [(1)]
  \setlength{\itemindent}{1cm}
  \item {\bf 有限区间:}$a,b\ne \infty$
  \item {\bf 分段连续:}$f(x)$最多有限多个第一类间断点 
  \item {\bf 必要条件:}$f(x)$有界
\end{enumerate}

\subsection{无穷限的反常积分}

{\bf 定义6.5.1:}
$${\dint_a^{+\infty}f(x)\d x}
=\lim\limits_{t\to+\infty}\dint_a^tf(x)\d x$$
类似地,可以定义${\dint_{-\infty}^af(x)\d x}$和
${\dint_{-\infty}^{+\infty}f(x)\d x}$

Newton-Leibnitz公式 
  $$\dint_a^{+\infty}f(x)\d x=F(x)|_{a}^{+\infty}
  =\limx{+\infty}F(x)-F(a)$$
  
{\bf 教材-例1:}计算以下无穷积分
$$\dint_{-\infty}^{+\infty}\df{1}{1+x^2}\d x$$

\begin{center}
	\resizebox{!}{5cm}{\includegraphics{./images/ch6/infint.pdf}}
\end{center}

{\bf 教材-例3-5:}
\begin{enumerate}[(1)]
  \setlength{\itemindent}{1cm}
  \item 讨论$\dint_a^{+\infty}\df{\d x}{x^p},\;a>0, p>0$的敛散性; 
  \item 证明$$\dint_1^{n+1}\df 1{x^p}\d x<\sum\limits_{k=1}^n\df 1{k^p}
  <1+\dint_1^n\df{1}{x^p}\d x$$ 
  \item 根据以上结论给出$p$级数的判敛条件。
\end{enumerate}

\begin{shaded}
	{\bf Gabriel's Horn Paradox}
	
	\begin{center}
		\resizebox{!}{5cm}{\includegraphics{./images/ch6/gabrielshorn.jpg}}
	\end{center}
	
	Gabriel是圣经故事中的大天使(Saint Gabriel the Archangel)之一
	(其他的还有Michael、Raphael等)
	,他吹奏一支长喇叭让魔鬼坠入地狱。关于长喇叭有如下的一个悖论
	(据说由Torricelli提出):设Gabriel喇叭由曲线$y=\df1x(x\geq 1)$
	围绕$x$轴旋转而成,证明:
	\begin{enumerate} [(1)]
  	  \setlength{\itemindent}{1cm}
	  \item 喇叭的体积是一个有限值
	  \item 喇叭的表面积为无穷大
	  \item 有前面的结论应该可以推论出这样的结果:Gabriel喇叭的内部可以被有限数量
	  的油漆填满,但同时,任何有限数量的漆都无法刷遍其内表面!该如何解释这个悖论呢?
	\end{enumerate}
	[提示]:
	(1)体积为
	$$V=\dint_1^{+\infty}\pi\left(\df1x\right)^2\d x=\pi$$
	(2)表面积
	$$A=2\pi\dint_1^{+\infty}y\sqrt{1+(y')^2}\d x=2\pi\dint_1^{+\infty}
	\df{\sqrt{x^4+1}\d x}{x^3}>2\pi\dint_1^{+\infty}\df1x\d x=+\infty$$
	(3)如果涂在表面上的漆的厚度可以是任意小,则任意有限数量的漆都可以刷遍整个喇叭。
	
	一个类似的论题:对于平面区域$D:x\geq0,y\in[0,1]$,给定一个单位质量的金箔,
	只要我们允许金珀铺的任意薄(厚度$h(x)=e^{-x}$),则可以将整个区域$D$铺满。
\end{shaded}

\subsection{无界函数的反常积分}

{\bf 定义6.5.2:}
\begin{enumerate} [(1)]
  \setlength{\itemindent}{1cm}
  \item {\it 瑕点:}无穷间断点 
  \item 设$a$为$f(x)$的瑕点,{\it 瑕积分:}
  $$\dint_a^bf(x)\d x=\limx{a^+}\dint_t^bf(x)\d x$$
\end{enumerate}

{\bf 教材-例6:}计算
$$\dint_0^a\df{1}{\sqrt{a^2-x^2}}\d x,\;a>0$$

{\bf 教材-例7:}讨论瑕积分的收敛性
\begin{enumerate}[(1)]
  \setlength{\itemindent}{1cm}
  \item $\dint_{-1}^1\df{\d x}{x^2}$ 
  \item $\dint_{-1}^1\df{\d x}{x}$
\end{enumerate}

{\bf 例:}计算
$$\dint_0^{+\infty}\df{\d x}{\sqrt{x(x+1)^3}}$$

\subsection{反常积分收敛性的判定}

{\bf 定理6.5.3:}(比较判别法)
设$f(x),g(x)$在$[a,+\infty)$上连续,且
$$0\leq f(x)\leq g(x),\quad x\in[a,+\infty)$$ 
\begin{enumerate}[(1)]
  \setlength{\itemindent}{1cm}
  \item
  $\dint_a^{+\infty}g(x)\d x$收敛$\Rightarrow\dint_a^{+\infty}f(x)\d x$收敛 
  \item $\dint_a^{+\infty}f(x)\d x$发散$\Rightarrow\dint_a^{+\infty}g(x)\d x$发散
\end{enumerate}

{\bf 教材-例1:}讨论以下无穷积分的敛散性
\begin{enumerate}[(1)]
  \setlength{\itemindent}{1cm}
  \item $\dint_2^{+\infty}\df{1+\sin x}{x^2}\d x$ 
  \item $\dint_1^{+\infty}\df{\d x}{\sqrt[3]{x^4+1}}$
\end{enumerate}

{\bf 定理}(极限判别法)
设$f(x),g(x)$在$[a,+\infty)$上连续,$0\leq f(x)\leq g(x)$,且
$$\limx{+\infty}\df{f(x)}{g(x)}=l$$ 
\begin{enumerate}[(1)]
  \setlength{\itemindent}{1cm}
  \item $0<l<+\infty$,两个无穷积分同敛散 
  \item
  $l=0$,$\dint_a^{+\infty}g(x)\d x$收敛
  $\Rightarrow\dint_a^{+\infty}f(x)\d x$收敛 
  \item
  $l=+\infty$,$\dint_a^{+\infty}f(x)\d x$发散
  $\Rightarrow\dint_a^{+\infty}g(x)\d x$发散
\end{enumerate}

{\bf 例:}判定以下反常积分的收敛性
\begin{enumerate}[(1)]
  \setlength{\itemindent}{1cm}
  \item $\dint_1^{+\infty}\df{\d x}{x\sqrt{1+x^2}}$ 
  \item $\dint_1^{+\infty}\df{x^{3/2}}{1+x^2}\d x$ 
  \item $\dint_1^{+\infty}\df{\arctan x}{x}\d x$
\end{enumerate}

{\bf 定理}(绝对收敛性)设$f(x)$在$[a,+\infty)$上连续,
若$\dint_a^{+\infty}|f(x)|\d x$收敛,则
$\dint_a^{+\infty}f(x)\d x$收敛。

{\bf 例:}讨论反常积分$\dint_0^{+\infty}e^{-ax}\sin bx\d x\;(a>0)$的敛散性。

{\bf 例:}判定下列积分的敛散性
\begin{enumerate}[(1)]
  \setlength{\itemindent}{1cm}
  \item $\dint_1^3\df{\d x}{\ln x}$ 
  \item $\dint_0^1\df{\d x}{\sqrt{(1-x^2)(1-k^2x^2)}}\;(k^2<1)$ 
  \item $\dint_0^1\df 1{\sqrt{x}}\sin\df 1x\d x$
\end{enumerate}

\subsection{$\Gamma$函数}

$${\Gamma(s)=\dint_0^{+\infty}e^{-x}x^{s-1}\d x\;(s>0)}$$

\begin{enumerate}[(1)]
  \setlength{\itemindent}{1cm}
  \item $\Gamma(s+1)=s\Gamma(s)$ 
  \item $\lim\limits_{s\to 0+}\Gamma(s)= +\infty$ 
  \item $\Gamma(s)\Gamma(1-s)=\df{\pi}{\sin \pi s}\;(0<s<1)$ 
  \item $2\dint_0^{+\infty}e^{-x^2}\d x=\Gamma\left(\df 12\right)=\sqrt{\pi}$
\end{enumerate}

{\bf 例:}证明$\Gamma(s)=\dint_0^{+\infty}e^{-x}x^{s-1}\d x\;(s>0)$是收敛的。

[证]:显然$s=1$时,积分收敛。

当$s\ne1$时,$e^{-x}x^{s-1}<x^{s-1}\;(x>0)$,又$\dint_0^1x^{s-1}\d x=\df1s$,
故由比较判别法,$I_1=\dint_0^1e^{-x}x^{s-1}\d x$收敛。

由$\limx{+\infty}e^{-\frac x2}x^{s-1}=0$,存在$X>1$,对任意$x>X$,有
$e^{-\frac x2}x^{s-1}<1$,进而$e^{-x}x^{s-1}<e^{-\frac x2}$,又
$\dint_X^{+\infty}e^{-\frac x2}\d x=2e^{\frac X2}$,故由比较判别法,
$I_3=\dint_X^{+\infty}e^{-x}x^{s-1}\d x$收敛。

$I_2=\dint_1^Xe^{-x}x^{s-1}\d x$为定积分。

综上,$\Gamma(s)=I_1+I_2+I_3$收敛。

\newpage

\section*{课后作业}
\addcontentsline{toc}{section}{课后作业}

{\bf 【必作题】}

\begin{itemize}
  \item 习题6.1:3(2,4),4,5(4,5,6),7
  \item 习题6.2:1(5-8),4,6,8,11,12,13
  \item 习题6.3:4(3,4,5,11,12,15,16),5,6(6,7,8,9),7(5,8,12,16),
  8,11,12(3,4),16
  \item 习题6.4:2,5,6(1),8,10,11,15,16,19
  \item 习题6.5:1(4-6),2,3,6,8
\end{itemize}

\bigskip

\hrule

\bigskip
\bigskip

{\bf 【思考题】}

\begin{itemize}
  \item 习题6.1:9,10
  \item 习题6.2:3,7,9,10
  \item 习题6.3:9,17,18
  \item 习题6.4:1-30其余各题
  \item 习题6.5:4,5
\end{itemize}

% {\bf 例:}$a>0,c\in(0,e^a)$,证明:
% $$\dint_0^a|e^x-c|\dx\geq(e^{\frac a2}-1)^2$$

\newpage

\section*{习题参考解答}
\addcontentsline{toc}{section}{习题参考解答}

{\bf 习题6.5-8.} 长度为$l$的细杆,均匀带电,总电量为$q\,(q<0)$,若在杆的延长线上,
距离杆一端$x_0$处有一单位正电荷。现将单位正电荷从$x_0$处移动到无穷远,试求克服
电场力所作的功。

解:如图,
\begin{center}
	\resizebox{!}{2cm}{\includegraphics{./images/ch6/eMove.pdf}}
\end{center}

当单位正电荷位于$x(x>l)$处时,细杆上位置$t$处长度为$\d t$的一段对其产生的电场(吸引)力为
$$\d F=\df{kq\d t}{l(x-t)^2},\quad t\in[0,l]$$
故此时单位正电荷所受的总电场力为
$$F=\dint_0^l\df{kq\d t}{l(x-t)^2}=\df{kq}l\left(\df1{x-l}-\df1x\right).$$
克服该力的作用,单位正电荷向远处移动$\d x$,需要做的功为
$$\d W=F\d x=\df{kq}l\left(\df1{x-l}-\df1x\right)\d x,
\quad x\in[l+x_0,+\infty)$$
故所求克服电场力所需做的总功为
$$W=\dint_{l+x_0}^{+\infty}\df{kq}l\left(\df1{x-l}-\df1x\right)\d x
=\df{kq}l\ln\left(1+\df{l}{x_0}\right).$$